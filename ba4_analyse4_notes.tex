\documentclass[a4paper, 11pt]{report}

\usepackage[french]{babel}
\usepackage[a4paper]{geometry}
\usepackage{amsmath}
\usepackage{amsthm}
\usepackage{amssymb}
\usepackage{mathtools}
\usepackage{enumitem}
\usepackage{graphicx}
\usepackage[hidelinks]{hyperref}


\newtheorem*{theorem}{Théorème}
\newtheorem*{corollary}{Corollaire}

\theoremstyle{definition}
\newtheorem*{motivation}{Motivation}
\newtheorem*{hypothesis}{Hypothèses}
\newtheorem*{result}{Résultat}
\newtheorem*{definition}{Définition}
\newtheorem*{idea}{Idée}
\newtheorem*{convention}{Convention}
\newtheorem*{method}{Méthode}

\theoremstyle{remark}
\newtheorem*{terminology}{Terminologie}
\newtheorem*{note}{Note}
\newtheorem*{remark}{Remarque}
\newtheorem*{example}{Exemple}
\newtheorem*{illustration}{Illustration}
\newtheorem*{constatation}{Constatation}

\newcommand{\N}{\mathbb{N}}
\newcommand{\Z}{\mathbb{Z}}
\newcommand{\R}{\mathbb{R}}
\newcommand{\Cx}{\mathbb{C}}
\newcommand{\dd}{\mathrm{d}}
\newcommand{\dl}{\mathrm{d}\ell}
\newcommand{\ds}{\mathrm{d}s}
\newcommand{\dt}{\mathrm{d}t}
\newcommand{\dx}{\mathrm{d}x}
\newcommand{\dy}{\mathrm{d}y}
\newcommand{\dz}{\mathrm{d}z}
\newcommand{\dal}{\mathrm{d}\alpha}
\newcommand{\dth}{\mathrm{d}\theta}
\newcommand{\dxi}{\mathrm{d}\xi}
\newcommand{\TF}{\mathfrak{F}}
\newcommand{\TFi}{\TF^{-1}}
\newcommand{\TL}{\mathcal{L}}
\newcommand{\TLi}{\TL^{-1}}
\newcommand{\ostpi}{\frac{1}{\sqrt{2 \pi}}} % one (over) sqrt of two Pi
\newcommand{\oipi}{\frac{1}{2 \pi i}} % one (over) i times two Pi
\newcommand{\cd}{\cdot}
\newcommand{\Rep}{\operatorname{Re}}
\newcommand{\Imp}{\operatorname{Im}}
\newcommand*\conj[1]{\overline{#1}}
\newcommand*\module[1]{\left|#1\right|}
\newcommand{\Arctg}{\textrm{Arctg}\,}
\newcommand{\inte}{\textrm{int}\,}
\newcommand{\rot}{\textrm{rot}\,}
\newcommand{\bord}{\partial}
\newcommand*\adh[1]{\conj{\inte #1}}
\newcommand{\Res}{\textrm{Rés}}


\title{\textbf{Analyse IV} \\
Transcript du cours du Pr. Michel \bsc{Cibils}}
\author{Robin \bsc{Mamié}}
\date{Printemps 2018}


\begin{document}

\maketitle

\tableofcontents

% !TeX spellcheck = fr_FR
\chapter{Transformées de Fourier}


\section{Introduction}


\subsection{Définitions et résultats préliminaires}

\begin{definition}
    Une fonction $f: \R \rightarrow \R$ est dite \textbf{T-périodique} s'il existe $T > 0$ tel que $f(x + T) = f(x) \enspace \forall x \in \R$.
    
    L'intervalle $[0, T]$ caractérise complètement la fonction.
\end{definition}


\begin{definition}[14.1.i, p.103]

    Une fonction $f: \R \rightarrow \R$ est dite \textbf{continue par morceaux} sur l'intervalle $[a, b]$ s'il existe des points $ \{x\}_{i = 0}^{n + 1} \subset [a, b]$ avec $a = x_0 < x_1 < \cdots < x_n < x_{n + 1} = b$ tels que pour $i = 0, 1, \ldots, n$ on ait:
    
    \begin{enumerate}

        \item
        $f$ est continue sur chaque intervalle ouvert $]x_i, x_{i + 1}[$

        \item
        la limite à droite $f(x_i + 0) := \lim\limits_{\substack{t \rightarrow x_i \\ t > x_i}} f(t)$ et la limite à gauche $f(x_{i + 1} - 0) := \lim\limits_{\substack{t \rightarrow x_{i+1} \\ t < x_{i+1}}} f(t)$ existent et sont finies.

    \end{enumerate}

\end{definition}


\begin{terminology}
    On dit qu'une fonction T-périodique est continue par morceaux si elle l'est sur l'intervalle $[0, T]$ qui la caractérise.
\end{terminology}


\begin{definition}[14.2, p.104]
    Soit $f: \R \rightarrow \R$ une fonction T-périodique continue par morceaux.
    Pour $N \in \mathbb{N}$, la \textbf{série de Fourier partielle d'ordre N} de $f$ est:
    
    \[
    F_N f(x) = \sum_{n = -N}^{N} c_n e^{i \frac{2 \pi n}{T} x}
    \]
    
    où les \textbf{coefficients de Fourier} $c_n$ sont des nombres complexes donnés par:
    
    \[
    c_n = \frac{1}{T} \int_{0}^{T} f(x) e^{-i \frac{2 \pi n}{T} x} \dx
    \]
    
    On appelle \textbf{série de Fourier de f} (en notation complexe) la limite lorsque $N \longrightarrow \infty$ de la série de $F_N f(x)$.
    On écrit:
    
    \[ F f(x) := \lim_{N \rightarrow +\infty} F_N f(x) = \sum_{-\infty}^{\infty} c_n e^{i \frac{2 \pi n}{T} x} \]
\end{definition}


\begin{theorem}[de Dirichlet -- Résultat de convergence; 14.3, p.104]
    Soit $f: \R \rightarrow \R$ une fonction T-périodique telle que $f$ et $f'$ soient continues par morceaux.
    Alors $\forall x \in \R$:
    
    \[
    F f(x) = \lim\limits_{N \rightarrow \infty} F_N f(x) \textrm{ existe et } F f(x) = \frac{f(x + 0) + f(x - 0)}{2}
    \]
    
    En particulier, si $f$ est continue en $x$, alors $f(x + 0) = f(x - 0) = f(x)$ et on a $F f(x) = f(x)$.
\end{theorem}


\begin{note}
    Utilisation de la formule d'Euler $e^{ix} = \cos x + i \sin x$ \textit{(cf. ex. 1-2, série 1)}.
\end{note}


\subsection{Motivation}

\begin{description}

    \item[Série de Fourier]
    développement des fonctions \textit{périodiques} comme somme infinie de fonctions trigonométriques.
    
    \item[Transformée de Fourier]
    étude de fonctions \textit{non} périodiques.

\end{description}


\begin{idea}
    Soit $T > 0$ et $f_T$ une fonction T-périodique définie par
    
    \[
    f_T (x) =
    \left \{
    \begin{array}{r c l}
        0 & \mathrm{si} & x \in \enspace ]-\frac{T}{2}, -1[ \\
        1 & \mathrm{si} & x \in [-1, 1] \\
        0 & \mathrm{si} & x \in \enspace ]1, \frac{T}{2}[ \\
    \end{array}
    \right .
    \]
    
    Lorsque la période $T \rightarrow \infty$, on a:
    
    \[
    \lim_{T \rightarrow \infty} f_T (x) = f(x) =
    \left \{
    \begin{array}{r c l}
        1 & \mathrm{si} & x \in [-1, 1] \\
        0 & \mathrm{si} & x \notin [-1, 1] \\
    \end{array}
    \right .
    \]
    
    qui n'est plus une fonction périodique.
    
\end{idea}


\begin{idea}
    considérer des fonctions comme limites de fonctions périodiques dont la période $T$ tend vers $+\infty$.
\end{idea}


\subsection{Raisonnement heuristique}

Soit $f_T: \R \rightarrow \R$ une fonction \textit{continue}, T-périodique telle que $f'_T$ soit continue par morceaux.
Alors la série de Fourier de $f_T$ est:

\[
F f_T(y) = \sum_{-\infty}^{+\infty} c_n e^{i \frac{2 \pi n}{T} y}
\]

pour $y \in \R$, où

\[
c_n = \frac{1}{T} \int_{0}^{T} f_T(x) e^{-i \frac{2 \pi n}{T} x} \dx
= \frac{1}{T} \int_{-\frac{T}{2}}^{\frac{T}{2}} f_T(x) e^{-i \frac{2 \pi n}{T} x} \dx
\]

En écrivant
$\Delta \alpha = \frac{2 \pi}{T}$
et
$\alpha_n = n \cdot \Delta \alpha$,
on a
$\frac{1}{T} = \frac{\Delta \alpha}{2 \pi}$.

\[
c_n =
\frac{\Delta \alpha}{2 \pi}
\int_{-\frac{T}{2}}^{\frac{T}{2}} f_T(x) e^{-i \alpha_n x} \dx
\]

\[
\Rightarrow
F f_T(y) =
\sum_{-\infty}^{+\infty}
\left[
\frac{\Delta \alpha}{2 \pi}
\int_{-\frac{T}{2}}^{\frac{T}{2}}
f_T(x) e^{-i \alpha_n x} \dx
\right]
e^{i \alpha_n y}
\]

Échange de la somme infinie et de l'intégrale:

\[
F f_T(y) =
\frac{1}{2 \pi}
\int_{-\frac{T}{2}}^{\frac{T}{2}}
f_T(x)
\left[
\Delta \alpha
\sum_{-\infty}^{+\infty}
e^{-i \alpha_n (x - y)}
\right]
\dx
\]

On découvre une somme de Riemann qui permet de définir une intégrale.
En effet:

\[
\Delta \alpha \sum_{-\infty}^{+\infty}
e^{-i \alpha_n (x - y)}
\underbrace{=}_{\Delta\alpha = \alpha_n - \alpha_{n - 1}}
\sum_{-\infty}^{+\infty}
e^{-i \alpha_n (x - y)} (\alpha_n - \alpha_{n - 1})
\]

\[
=
\int_{-\infty}^{+\infty}
e^{-i \alpha (x - y)}
\dal
\]

Donc on obtient:

\begin{align*}
F f_T(y) &=
\frac{1}{2 \pi}
\int_{-\frac{T}{2}}^{\frac{T}{2}}
f_T(x)
\left[
\int_{-\infty}^{+\infty}
e^{-i \alpha (x - y)}
\dal
\right]
\dx
\\&=
\frac{1}{\sqrt{2 \pi}}
\int_{-\infty}^{+\infty}
\left[
\frac{1}{\sqrt{2 \pi}}
\int_{-\frac{T}{2}}^{\frac{T}{2}}
f_T(x) e^{-i \alpha x}
\dx
\right]
e^{i \alpha y}
\dal
\end{align*}

Comme $f_T$ est continue, alors on a $f_T(y) = F f_T(y)$ et donc lorsque $T$ tend vers $+\infty$, on a
$\lim\limits_{T \rightarrow +\infty} f_T(y) =
\lim\limits_{T \rightarrow +\infty} F f_T(y) \iff
f(y) = \lim\limits_{T \rightarrow +\infty} F f_T(y)$.

\[
\iff
f(y) =
\frac{1}{\sqrt{2 \pi}}
\int_{-\infty}^{+\infty}
\underbrace{
\left[
\frac{1}{\sqrt{2 \pi}}
\int_{-\infty}^{+\infty}
f(x) e^{-i \alpha x}
\dx
\right]
}_{\substack{
\textrm{Nouvelle fonction qui dépend de la}\\
\textrm{variable $\alpha$, qui est appelée la}\\
\textrm{transformée de Fourier de $f$ et}\\
\textrm{notée $\TF(f)$ ou $\hat{f}$}
}}
e^{i \alpha y}
\dal
\]

On écrit:

\[
\TF(f)(\alpha) = \hat{f}(\alpha) =
\frac{1}{\sqrt{2 \pi}}
\int_{-\infty}^{+\infty}
f(x) e^{-i \alpha x}
\dx
\]

\begin{remark}
    On a que:
    
    \[
    f(y) =
    \frac{1}{\sqrt{2 \pi}}
    \int_{-\infty}^{+\infty}
    \hat{f}(\alpha)
    e^{i \alpha y}
    \dal
    \]
\end{remark}



\section{Transformée de Fourier d'une fonction}


\subsection{Définition}

\begin{definition}[15.1, p.113]
    Soit $f: \R \rightarrow \R$ une fonction continue par morceaux et telle que $\int_{-\infty}^{+\infty} | f(x) | \dx < \infty$.
    
    La \textbf{transformée de Fourier} de $f$ est la fonction notée $\TF(f)$ ou $\hat{f}: \R \rightarrow \Cx$ définie par:
    
    \[
    \alpha \longmapsto
    \TF(f)(\alpha) = \hat{f}(\alpha) =
    \frac{1}{\sqrt{2 \pi}}
    \int_{-\infty}^{+\infty}
    f(x) e^{-i \alpha x}
    \dx
    \]
\end{definition}


\subsection{Exemples}

\begin{example}
    Calculer la transformée de Fourier de la fonction:
    
    \[
    f:
    \begin{array}{r c l}
        \R & \rightarrow & \R_+^* \\
        x  & \mapsto & f(x) = e^{-|x|} =
        \left \{
        \begin{array}{r c l}
            e^{-x} & \textrm{si} & x \geq 0 \\
            e^x    & \textrm{si} & x < 0
        \end{array}
        \right.
    \end{array}
    \]
    
    \begin{align*}
    \hat{f}(\alpha) & =
    \frac{1}{\sqrt{2 \pi}}
    \int_{-\infty}^{+\infty}
    f(x)
    e^{-i \alpha x}
    \dx
    =
    \frac{1}{\sqrt{2 \pi}}
    \int_{-\infty}^{+\infty}
    e^{-|x|}
    e^{-i \alpha x}
    \dx
    \\&=
    \frac{1}{\sqrt{2 \pi}}
    \left[
    \int_{-\infty}^{0}
    e^{x}
    e^{-i \alpha x}
    \dx
    +
    \int_{0}^{+\infty}
    e^{-x}
    e^{-i \alpha x}
    \dx
    \right]
    \\&=
    \frac{1}{\sqrt{2 \pi}}
    \left[
    \int_{-\infty}^{0}
    e^{(1 - i \alpha) x}
    \dx
    +
    \int_{0}^{+\infty}
    e^{-(1 + i \alpha) x}
    \dx
    \right]
    \\&=
    \frac{1}{\sqrt{2 \pi}}
    \left[
    \left.
    \frac{e^{(1 - i \alpha) x}}
    {1 - i \alpha}
    \right|_{-\infty}^0
    -
    \left.
    \frac{e^{-(1 + i \alpha) x}}
    {1 + i \alpha}
    \right|_0^{+\infty}
    \right]
    \\&=
    \frac{1}{\sqrt{2 \pi}}
    \left[
    \frac{1}
    {1 - i \alpha}
    \left(
    1 - \lim\limits_{x \rightarrow -\infty} e^{(1 - i \alpha) x}
    \right)
    \right.
    -
    \left.
    \frac{1}
    {1 + i \alpha}
    \left(
    \lim\limits_{x \rightarrow + \infty} e^{-(1 + i \alpha) x} - 1
    \right)
    \right]
    \\&=
    \frac{1}{\sqrt{2 \pi}}
    \left(
    \frac{1}{1 - i \alpha}
    +
    \frac{1}{1 + i \alpha}
    \right)
    \\&=
    \frac{1}{\sqrt{2 \pi}}
    \frac{1 + i \alpha + 1 - i \alpha}
    {(1 - i \alpha) (1 + i \alpha)}
    =
    \frac{1}{\sqrt{2 \pi}}
    \frac{2}{1 + \alpha^2}
    \end{align*}
    
    \textbf{Résultat:}
    
    \[
    \hat{f}:
    \begin{array}{r c l}
    \R & \rightarrow & \R_+^* \\
    \alpha & \mapsto & \hat{f}(\alpha) =
    \sqrt{\frac{2}{\pi}} \frac{1}{1 + \alpha^2}
    \end{array}
    \]
\end{example}

\newpage

\begin{remark}
    Pour calculer
    $\lim\limits_{x \rightarrow -\infty} e^{(1 - i \alpha) x}$:
    
    \begin{align*}
    \left|e^{(1 - i \alpha) x}\right|
    &= \left|e^{- i \alpha x} e^x\right|
    = \underbrace{\left|e^{- i \alpha x}\right|}_{= 1} \left|e^x\right|
    = e^x\\
    \Rightarrow \lim\limits_{x \rightarrow -\infty} \left|e^{(1 - i \alpha) x}\right|
    &= \lim\limits_{x \rightarrow -\infty} e^x
    = 0\\
    \Rightarrow \lim\limits_{x \rightarrow -\infty} e^{(1 - i \alpha) x}
    &= 0 + i0
    = 0
    \end{align*}
    
    On a aussi $\lim\limits_{x \rightarrow +\infty} e^{-(1 + i \alpha) x} = 0$.
\end{remark}


\begin{example}
    Soit la fonction 
    
    \[
    f:
    \begin{array}{r c l}
    \R & \rightarrow & \R \\
    x  & \mapsto & f(x) = 
    \left \{
    \begin{array}{r c l}
    1 & \textrm{si} & x \in [-1, 1] \\
    0   & \textrm{sinon} &
    \end{array}
    \right.
    \end{array}
    \]
    
    Calcul de la transformée de Fourier de $f$.
    
    \begin{align*}
    \hat{f}(\alpha) & =
    \frac{1}{\sqrt{2 \pi}}
    \int_{-\infty}^{+\infty}
    f(x)
    e^{-i \alpha x}
    \dx
    =
    \frac{1}{\sqrt{2 \pi}}
    \int_{-1}^{1}
    e^{-i \alpha x}
    \dx
    \\&\overbrace{=}^{\alpha \neq 0}
    -
    \frac{1}{\sqrt{2 \pi}}
    \left.
    \frac{e^{-i \alpha x}}{i \alpha}
    \right|_{-1}^{1}
    =
    \frac{1}{\sqrt{2 \pi}}
    \frac{e^{i \alpha} - e^{-i \alpha}}{i \alpha}
    \\&=
    \frac{2}{\sqrt{2 \pi}}
    \frac{1}{\alpha}
    \frac{e^{i \alpha} - e^{-i \alpha}}{2i}
    =
    \sqrt{\frac{2}{\pi}}
    \frac{\sin \alpha}{\alpha}
    \quad
    \text{si } \alpha \neq 0
    \end{align*}
    
    Pour $\alpha = 0$ :
    
    \begin{align*}
    \hat{f}(0) &=
    \frac{1}{\sqrt{2 \pi}}
    \int_{-\infty}^{+\infty}
    f(x)
    e^{-i 0 x}
    \dx
    =
    \frac{1}{\sqrt{2 \pi}}
    \int_{-1}^{1}
    \dx
    =
    \left.
    \frac{1}{\sqrt{2 \pi}}
    x
    \right|_{-1}^{1}
    =
    \sqrt{\frac{2}{\pi}}
    \end{align*}
    
    \textbf{Résultat:}
    
    \[
    \hat{f}:
    \begin{array}{r c l}
    \R & \rightarrow & \R \\
    \alpha & \mapsto & \hat{f}(\alpha) =
    \left\{
    \begin{array}{l c l}
    \sqrt{\frac{2}{\pi}}
    \frac{\sin \alpha}{\alpha} & \mathrm{si} & \alpha \neq 0\\
    \sqrt{\frac{2}{\pi}} & \mathrm{si} & \alpha = 0
    \end{array}
    \right.
    \end{array}
    \]
    
    On remarque que $\lim\limits_{\alpha \rightarrow 0} \hat{f}(\alpha) = \lim\limits_{\alpha \rightarrow 0} \sqrt{\frac{2}{\pi}} \frac{\sin \alpha}{\alpha} = \sqrt{\frac{2}{\pi}} = \hat{f}(0)$.
    
    $\implies \hat{f}$ est aussi continue en $\alpha = 0$.
    
    \textit{Autres exemples: ex. 3-4, série 1}
    
\end{example}



\section{Transformée de Fourier inverse}


\subsection{Définition}

\begin{definition}
    Soit $g: \R \rightarrow \Cx$ une fonction continue par morceaux telle que $\int_{-\infty}^{+ \infty} |g(t)| \dt < \infty$.
    La \textbf{transformée de Fourier inverse} de $g$ est notée:
    
    \[
    \TFi(g):
    \begin{array}{r c l}
    \R & \rightarrow & \Cx \\
    x & \rightarrow & \TFi(g)(x)
    \end{array}, \textrm{ où}
    \enspace
    \TFi(g)(x) =
    \frac{1}{\sqrt{2 \pi}} \int_{-\infty}^{+\infty} g(t) e^{itx} \dt.
    \]
\end{definition}


\subsection{Théorème de réciprocité (formule d'inversion)}

\begin{theorem}[15.3.i, p.115]
    Soit $f: \R \rightarrow \R$ une fonction telle que $f$ et $f'$ soient continues par morceaux avec $\int_{-\infty}^{+ \infty} |f(x)| dx < \infty$ et $\int_{-\infty}^{+ \infty} |\hat{f}(\alpha)| \dal < \infty$.
    Alors $\forall x \in \R$, on a:
    
    \[
    \TFi (\hat{f})(x) =
    \frac{1}{\sqrt{2 \pi}}\int_{-\infty}^{+ \infty} \hat{f}(\alpha) e^{i\alpha x}\dal =
    \frac{f(x+0)+f(x-0)}{2}
    \]
    
    En particulier si $f$ est continue en $x$, on a $\frac{1}{2}[f(x+0) + f(x-0)] = f(x)$ et alors:
    
    \[
    f(x) = \TFi (\hat{f})(x) =
    \frac{1}{\sqrt{2 \pi}}
    \int_{-\infty}^{+ \infty}
    \hat{f}(\alpha) e^{i\alpha x}
    \dal
    \]
    
    Autrement dit, on a $\TFi(\TF(f)) = f$.
    La transformée de Fourier peut être vue comme une \og transformation $\TF$ \fg{} inversible (une bijection) qui \og agit\fg{} sur la fonction $f$:
    
    \[
    f \enspace
    \xrightarrow{\enspace \TF \enspace}
    \enspace \hat{f} \enspace
    \xrightarrow{\TFi}
    \enspace f
    \]
\end{theorem}


\subsection{Exemple d'utilisation}

\begin{example}
    Soit $
    f:
    \begin{array}{r c l}
    \R & \rightarrow & \R_+^* \\
    x  & \mapsto & f(x) = e^{-|x|} =
    \left \{
    \begin{array}{r c l}
    e^{-x} & \textrm{si} & x \geq 0 \\
    e^x    & \textrm{si} & x < 0
    \end{array}
    \right.
    \end{array}
    $.
    
    La transformée de Fourier de $f$ est (\textit{exemple 1, §1.2.2})
    
    \[
    \hat{f}:
    \begin{array}{r c l}
    \R & \rightarrow & \R_+^* \\
    \alpha & \mapsto & \hat{f}(\alpha) =
    \sqrt{\frac{2}{\pi}} \frac{1}{1 + \alpha^2}
    \end{array}
    \]

    On remarque que $f'(x) =
    \left\{
    \begin{array}{r c l}
    -e^{-x} & \textrm{si} & x \geq 0 \\
    e^x    & \textrm{si} & x < 0
    \end{array}
    \right.$
    est continue par morceaux car:
    
    \[\lim\limits_{\substack{x \rightarrow 0 \\ x > 0}} f'(x) = -1 \textrm{ et } \lim\limits_{\substack{x \rightarrow 0 \\ x < 0}} f'(x) = 1\]
    
    \newpage
    
    De plus, $\int_{-\infty}^{+\infty} |\hat{f}|\dx = \sqrt{\frac{2}{\pi}} \int_{-\infty}^{+\infty} \frac{\dal}{1 + \alpha^2} \dal < \infty$.
    $f$ est continue $\forall x \in \R \implies$ en appliquant le \textit{théorème de réciprocité}, on a que $f(x) = \frac{1}{\sqrt{2 \pi}} \int_{-\infty}^{+\infty} \hat{f}(\alpha) e^{i\alpha x}$, i.e.:
    
    \[
    e^{-|x|} =
    \frac{1}{\sqrt{2 \pi}} \sqrt{\frac{2}{\pi}} \int_{-\infty}^{+\infty} \frac{e^{i\alpha x}}{1 + \alpha^2}\dal
    \quad \forall x \in \R
    \]
    
    En particulier, lorsque $x = 0$, on trouve:
    
   \[1 = \frac{1}{\pi} \int_{-\infty}^{+\infty} \frac{\dal}{1 + \alpha^2} \implies \int_{-\infty}^{+\infty} \frac{\dal}{1 + \alpha^2} = \pi.
   \]
    
    En particulier, lorsque $x = 1$, on trouve:
    
    \begin{align*}
    e^{-1} &=
    \frac{1}{\pi} \int_{-\infty}^{+\infty} \frac{e^{i \alpha}}{1 + \alpha^2}\dal =
    \frac{1}{\pi} \left[ \int_{-\infty}^{+\infty} \frac{\cos \alpha}{1 + \alpha^2} \dal +
    i \overbrace{\int_{-\infty}^{+\infty} \frac{\sin \alpha}{1 + \alpha^2} \dal}^{\substack{\textrm{fonction impaire} \\ \textrm{intégrée sur tout l'axe} \\ \textrm{réel } = \> 0}} \right]
    \\&\implies
    \int_{-\infty}^{+\infty} \frac{\cos \alpha}{1 + \alpha^2} \dal = \frac{\pi}{e}
    \end{align*}

\end{example}

\textbf{Conclusion:} Le \textit{théorème de réciprocité} permet de calculer la valeur d'intégrales généralisées.

\textit{Autre exemple: ex. 1, série 2}



\section{Propriétés de la transformée de Fourier}


On considère $f$ et $g: \R \rightarrow \R$ continues par morceaux telles que $\int_{-\infty}^{+\infty} |f(x)| \dx < \infty$ et $\int_{-\infty}^{+\infty} |g(x)| \dx < \infty$.
On note indifféremment $\TF(f) \doteq \hat{f}$ et $\TF(g) \doteq \hat{g}$ les transformées de Fourier de $f$ et de $g$.

\begin{note}
    Les prochains résultats sont décrits dans les \textbf{théorèmes 15.2 et 15.3} aux pages 113 à 115 du livre du cours.
\end{note}


\subsection{Continuité et linéarité}

\begin{itemize}
\item
$\TF(f)$ est continue $\forall \alpha \in \R$ et $\lim\limits_{\alpha \rightarrow \pm \infty} |\TF(f)(\alpha)| = 0$.
\item
$\TF$ linéaire: $\TF(af + bg) = a\,\TF(f) + b\,\TF(g) \quad \forall a,b \in \R$.
\end{itemize}


\subsection{Transformée de Fourier du produit de convolution}

\begin{definition}
    Le \textbf{produit de convolution} de deux fonctions $f$ et $g$ est la fonction notée $f \ast g: \R \rightarrow \R$ définie par:
    
    \[(f \ast g)(x) := \int_{-\infty}^{+\infty} f(x - t) g(t) \dt\]
\end{definition}

\begin{remark}
    On peut aussi écrire $(f \ast g)(x) := \int_{-\infty}^{+\infty} f(t') g(x - t') \dt'$, via un changement de variable.
\end{remark}

\textbf{Résultat:} on a que $\TF(f \ast g) = \sqrt{2\pi} \enspace \TF(f) \cdot \TF(g)$.

La transformée de Fourier du produit de convolution de deux fonctions est égale au \textbf{produit} des transformées de Fourier de chaque fonction.

\textit{Exemples: ex. 2-3, série 2}


\subsection{Transformée de Fourier de la dérivée d'une fonction}

Si de plus $f \in C^1(\R)$ et $\int_{-\infty}^{+\infty} |f'(x)| \dx < \infty$, alors on a :

\[\TF(f')(\alpha) = i\alpha \enspace \TF(f)(\alpha) \quad \forall \alpha \in \R\]

On écrit aussi $\widehat{f'}(\alpha) = i\alpha \enspace \hat{f}(\alpha)$.

La transformée de Fourier de la dérivée de $f$ s'obtient en \textbf{multipliant par} $i\alpha$ la transformée de Fourier de $f$.

Plus généralement, si $f\in C^n(\R)$ et $\int_{-\infty}^{+\infty} |f^{(k)}(x)| \dx < \infty$ pour $k=1,2,\ldots,n$, alors on a:

\[\TF(f^{(k)})(\alpha) = (i\alpha)^k \enspace \TF(f)(\alpha) \quad \forall \alpha \in \R, k = 1,2,\ldots,n\]

On écrit aussi $\widehat{f^{(k)}}(\alpha) = (i\alpha)^k \enspace \hat{f}(\alpha)$.

\textit{Exemple: ex.3, série 2}


\subsection{Décalage}

Si $a \in \R^*, b \in \R$ et $h(x) = e^{-ibx}f(ax)$, alors:

\[\TF(h)(\alpha) = \frac{1}{|a|} \TF(f)\left(\frac{\alpha + b}{a}\right)\]


\subsection{Identité de Plancherel}

Si de plus $\int_{-\infty}^{+\infty} [f(x)]^2 \dx < \infty$, alors on a:

\[\int_{-\infty}^{+\infty} [f(x)]^2 \dx = \int_{-\infty}^{+\infty} |\TF(f)(\alpha)|^2 \dal\]


\subsection{Transformée de Fourier en sinus et cosinus}

Si la fonction $f$ est paire (i.e. $f(-x) = f(x) \enspace \forall x \in \R$), alors on a:

\[
\TF(f)(\alpha) =
\sqrt{\frac{2}{\pi}}
\int_{0}^{+\infty} f(x)\cos(\alpha x) \dx
\]

qui est la transformée de Fourier en \textbf{cosinus} de $f$.

Si la fonction $f$ est impaire (i.e. $f(-x) = -f(x) \enspace \forall x \in \R$), alors on a:

\[
\TF(f)(\alpha) =
- i
\sqrt{\frac{2}{\pi}}
\int_{0}^{+\infty} f(x)\sin(\alpha x) \dx
\]

qui est la transformée de Fourier en \textbf{sinus} de $f$.

\textit{Exemples: ex.4, série 2 et ex.1, série 3}

\begin{remark}
    Si de plus $f'$ est continue par morceaux et $\int_{-\infty}^{+\infty} |\hat{f}(\alpha)| \dal < \infty$, alors, d'après le \textit{théorème de réciprocité}, on a:
    
    \begin{align*}
    f(x) &=
    \sqrt{\frac{2}{\pi}}
    \int_{0}^{+\infty} \hat{f}(\alpha)\cos(\alpha x) \dal
    \quad
    \textrm{ lorsque } f \textrm{ est paire}\\
    f(x) &=
    i
    \sqrt{\frac{2}{\pi}}
    \int_{0}^{+\infty} \hat{f}(\alpha)\sin(\alpha x) \dal
    \quad
    \textrm{ lorsque } f \textrm{ est impaire}\\
    \end{align*}
\end{remark}


\section{Démonstrations de quelques propriétés}

\subsection{Transformée de Fourier de la dérivée d'une fonction}

\begin{proof}
    On a
    
    \[
    \TF(f')(\alpha) =
    \ostpi \int_{-\infty}^{+\infty} f'(x) e^{-i\alpha x} \dx
    \]
    
    On intègre par parties, avec $u' = f' \rightarrow u = f$ et $v = e^{-i\alpha x} \rightarrow v' = -i \alpha e^{-i\alpha x}$.
    
    \[
    \implies \ostpi \left[
    f(x) e^{-i\alpha x} \bigg|_{-\infty}^{+\infty} +
    i\alpha \int_{-\infty}^{+\infty} f(x)e^{-i\alpha x} \dx
    \right]
    \]
    
    Or, $\lim\limits_{x \rightarrow \pm \infty} \left| f(x) e^{-i\alpha x}\right| = \lim\limits_{x \rightarrow \pm \infty} \left|f(x)\right| = 0$ car $\left|e^{-i\alpha x}\right| = 1 \enspace \forall x\in \R$ et $\int_{-\infty}^{+\infty} |f(x)| \dx < \infty$ ($f$ est sommable).
    
    \[
    \implies \TF(f')(\alpha) = i\alpha \ostpi \int_{-\infty}^{+\infty} f(x) e^{-i\alpha x} \dx = i\alpha \, \TF(f)(\alpha)
    \]
\end{proof}

\begin{remark}
    Formule pour la dérivée de la transformée de Fourier d'une fonction:
    
    \[\TF'(f)(\alpha) = (\hat{f})'(\alpha) = -i \, \TF\left(xf(x)\right)(\alpha)\]
    
    \textit{Cf. ex. 2, série 3}
\end{remark}


\subsection{Transformée de Fourier du produit de convolution}

\begin{proof}
    On a
    
    \begin{align*}
    \TF(f\ast g)(\alpha) &=
    \ostpi \int_{-\infty}^{+\infty} (f\ast g)(x) e^{-i\alpha x} \dx
    \\&=
    \ostpi \int_{-\infty}^{+\infty} \left[\int_{-\infty}^{+\infty} f(x-t)g(t)\dt\right] e^{-i\alpha x} \dx
    \\&=
    \ostpi \int_{-\infty}^{+\infty} \left[\int_{-\infty}^{+\infty} f(x-t)e^{-i\alpha x}\dx\right] g(t) \dt
    \\&\overbrace{=}^{\textrm{cdv}}
    \ostpi \int_{-\infty}^{+\infty} \left[\int_{-\infty}^{+\infty} f(y)e^{-i\alpha (y+t)}\dy\right] g(t) \dt
    \\&=
    \ostpi \int_{-\infty}^{+\infty} f(y)e^{-i\alpha y}\dy \int_{-\infty}^{+\infty} g(t)e^{-i\alpha t}\dt
    \\&=
    \sqrt{2 \pi} \> \TF(f)(\alpha) \cdot \TF(g)(\alpha)
    \end{align*}
    
\end{proof}


\subsection{Identité de Plancherel}

\begin{proof}
    Soit $t\in \R$.
    On remarque que
    
    \begin{align*}
    \int_{-\infty}^{+\infty} f(x) g(x+t) \dx &\overbrace{=}^{\footnotemark}
    \int_{-\infty}^{+\infty}
    f(x)
    \left[
    \frac{1}{\sqrt{2 \pi}} \int_{-\infty}^{+\infty} \TF(g)(\alpha) e^{i\alpha (x + t)} \dal
    \right]
    \dx
    \\&=
    \int_{-\infty}^{+\infty}
    \TF(g)(\alpha)
    \left[
    \frac{1}{\sqrt{2 \pi}} \int_{-\infty}^{+\infty}
    f(x) e^{i\alpha x} \dx
    \right]
    e^{i\alpha t}
    \dal
    \\&=
    \int_{-\infty}^{+\infty}
    \TF(g)(\alpha)
    \conj{\TF(f)(\alpha)}
    e^{i\alpha t}
    \dal
    \end{align*}
    \footnotetext{Par le \textbf{théorème de réciprocité}, §1.3.2}
    
    En posant $t = 0$ et en choisissant $g = f$, on obtient:
    
    \[
    \int_{-\infty}^{+\infty} \left[f(\alpha)\right]^2 \dx =
    \int_{-\infty}^{+\infty} \left|\TF(f)(\alpha)\right|^2 \dal
    \]
\end{proof}

\newpage

\section{Exemples d'utilisations de la transformée de Fourier}

\begin{note}
    Cette application est issue du \textbf{Chapitre 17.3} page 132 du livre de référence.
\end{note}

\begin{enumerate}[label=\alph*)]
    \item
    Trouver une solution particulière $y(x)$ d'une équation différentielle du type:
    
    \[\lambda\, y''(x) + \omega\, y(x) = f(x) \quad \forall x \in \R \quad \lambda, \omega \in \R \quad f: \R \rightarrow \R\]
    
    qui est l'équation de l'oscillateur forcé ($f$ est une fonction donnée).
    
    \textbf{Méthode:} on écrit la transformée de Fourier de l'équation différentielle:
    
    \begin{align*}
    &\TF(\lambda y'' + \omega y)(\alpha) = \TF(f)(\alpha)
    \\\iff&
    \lambda \, \TF(y'')(\alpha) + \omega \, \TF(y)(\alpha) = \TF(f)(\alpha)
    \\\iff&
    \lambda \left(-\alpha^2\right) \TF(y)(\alpha) + \omega \, \TF(y)(\alpha) = \TF(f)(\alpha)
    \\\iff&
    \left(\omega - \lambda \alpha^2\right) \TF(y)(\alpha) = \TF(f)(\alpha)
    \\\iff&
    \TF(y)(\alpha) = \frac{\TF(f)(\alpha)}{\omega - \lambda \alpha^2}
    \end{align*}
    
    On utilise le \textbf{théorème de réciprocité} (§1.3.2): la solution particulière $y(x)$ s'obtient en calculant la transformée de Fourier inverse de la fonction de la variable $\alpha$ définie par $\frac{\TF(f)(\alpha)}{\omega - \lambda \alpha^2}$.
    
    
    \item
    On utilise la transformée de Fourier pour résoudre des équations intégrales du type produit de convolution (\textit{cf. ex. 2 et 3, série 3}).
    
    
    \item
    Problèmes statistiques avec loi normale.
    
    fonction gaussienne $f(x) = e^{-\frac{x^2}{2}} \quad \forall x\in \R$
    
    \textit{Ex. 4, série 1}: $\hat{f}(\alpha) = e^{-\frac{\alpha^2}{2}}$ est aussi une fonction gaussienne.
    
    
    \item
    Mécanique quantique
    
    $f(x)$: position de la particule quantique\\
    $\hat{f}(p)$: impulsion de la particule quantique
    
    
    \item
    Résolution de l'équation de la chaleur pour une barre conductrice de longueur \textbf{infinie}.
    
    \begin{description}
    \item[Rappel:] cas d'une barre de longueur \textbf{finie}
    
    Soit une barre de longueur $0 < L < \infty$.
    On note $u(x,t)$ la fonction qui décrit la température de la barre au point $x$ et à l'instant $t$.
    
    L'évolution de la température $u(x,t)$ le long de la barre est modélisée par l'équation de la chaleur donnée par:
    
    \[
    \frac{\partial}{\partial t} u(x,t) = a^2 \frac{\partial^2}{\partial x^2} u(x,t) \quad \textrm{pour } x \in ]0,L[ \textrm{ et } t > 0, \quad a \neq 0
    \]
    
    où $a$ est un coefficient thermique.
    On impose:
    \begin{itemize}
    \item \textbf{deux} conditions limites $u(0,t) = u(L,t) = 0 \quad \forall t > 0$,
    \item une condition initiale $u(x,0) = f(x)$ pour $x \in ]0,L[$. \footnote{Dans le milieu de l'ingénierie, on sous-entend souvent qu'une condition \textbf{limite} l'est aux positions, et qu'une condition \textbf{initiale} l'est au temps}
    \end{itemize} 

    \textbf{Problème:} trouver une solution $u(x,t)$ satisfaisant ces conditions.
    
    \item[Cas d'une barre de longueur infinie]\hfill
    
    Il n'y a plus de conditions aux limites concernant les extrémités de la barre.
    Le problème est:
    
    \[
    \left\{
    \begin{array}{rcl}
    \frac{\partial}{\partial t} u(x,t) &=& a^2 \frac{\partial^2}{\partial x^2} u(x,t)\\
    u(x,0) &=& f(x)
    \end{array}
    \right.
    \]
    
    Pour $x\in \R$ et $t > 0$, avec \textbf{une} condition initiale, valable $\forall x \in \R$ où $f: \R \rightarrow \R$ est $C^1$ telle que $f$ et la transformée de Fourier de $f$ soient sommables.
    
    \item[Résolution]\hfill
    
    \textbf{1\iere{} étape:} on écrit la transformée de Fourier de l'équation de la chaleur en considérant $u(x,t)$ comme fonction de la variable $x$ ($t$ joue le rôle d'un paramètre).
    On obtient:
    
    \[
    \TF\left(\frac{\partial u}{\partial t}\right)(\alpha,t) = a^2 \TF\left(\frac{\partial^2 u}{\partial x^2}\right) (\alpha,t) \textrm{ avec } \TF(u)(\alpha,0) = \TF(f)(\alpha)
    \]
    
    Avec la notation:
    
    \[
    v(\alpha,t) = (\TF u)(\alpha,t) = \ostpi \int_{-\infty}^{+\infty} u(x,t) e^{-i\alpha x} \dx
    \]
    
    On obtient à gauche:
    
    \begin{align*}
    \TF\left(\frac{\partial u}{\partial t}\right)(\alpha,t) &=
    \ostpi \int_{-\infty}^{+\infty} \frac{\partial u}{\partial t}(x,t) e^{-i\alpha x} \dx
    \\&=
    \frac{\partial}{\partial t}
    \left[
    \ostpi \int_{-\infty}^{+\infty} u(x,t) e^{-i\alpha x} \dx
    \right]
    \\&=
    \frac{\partial}{\partial t}
    v(\alpha,t)
    \end{align*}
    
    \newpage
    
    Et à droite:
    
    \begin{align*}
    \TF\left(\frac{\partial^2 u}{\partial x^2}\right) (\alpha,t) &\overbrace{=}^{\footnotemark} (i\alpha)^2 \TF(u)(\alpha,t)
    \\&=
    - \alpha^2 v(\alpha,t)
    \end{align*}
    \footnotetext{Par la propriété du §1.4.3, concernant la \textbf{transformée de Fourier de la dérivée} d'une fonction}
    
    L'équation devient:
    
    \[
    \frac{\partial}{\partial t} v(\alpha,t) = - a^2 \alpha^2 v(\alpha,t)
    \]
    
    C'est une équation différentielle du \textbf{premier ordre} pour la fonction $v(\alpha,t)$ par rapport à \textbf{la variable} $t$ ($\alpha$ joue le rôle de paramètre).
    
    La solution est:
    
    \[
    v(\alpha,t) = v(\alpha,0) e^{- a^2 \alpha^2 t} = \TF(f)(\alpha) e^{- a^2 \alpha^2 t}
    \]
    
    \textbf{2\ieme{} étape:} pour obtenir la solution $u(x,t)$, on calcule la transformée de Fourier inverse de $v(\alpha,t)$ en considérant $t$ comme paramètre.
    On obtient:
    
    \begin{align*}
    u(x,t) = \TFi(v)(x,t) &=
    \ostpi \int_{-\infty}^{+\infty} v(\alpha,t) e^{i\alpha x} \dal
    \\&=
    \ostpi \int_{-\infty}^{+\infty} \TF(f)(\alpha) e^{-a^2 \alpha^2 t} e^{i\alpha x} \dal
    \end{align*}
    
    comme solution de l'équation de la chaleur pour une barre de longueur infinie.
    
    \textit{Exemple: ex. 3, série 3}
    \end{description}
\end{enumerate}

% !TeX spellcheck = fr_FR
\chapter{Fonctions holomorphes et équations de Cauchy-Riemann}


\section{Introduction}

\subsection{Motivation}

\begin{description}
    \item[But:] étendre l'étude de fonctions réelles (du type $f: \R \rightarrow \R$) à des fonctions qui dépendent d'\textbf{une} variable complexe qui sont à valeurs complexes (du type $f: \Cx \rightarrow \Cx$ où $\Cx$ est l'ensemble des nombres complexes).
    
    \item[Rôle:] établir les notions de limite, de continuité, de dérivabilité et d'intégration dans $\Cx$.
    
    \item[Intérêt:] méthodes puissantes qui permettent de calculer facilement des intégrales \textbf{ré\-elles} compliquées.
\end{description}

\textit{Cf. ex. 4, série 3}

\subsection{Rappel sur les nombres complexes}

\begin{itemize}
    \item 
    $\Cx$ désigne l'ensemble des nombres complexes
    \item 
    $z \in \Cx \iff z = x + iy$ avec $x = \Rep z \in \R$ et $y = \Imp z \in \R$ et $i^2 = -1$
    \item 
    $\Cx^* = \Cx \setminus \{0\}$ où $0 = 0 + i0$
    \item 
    complexe conjugué de $z \quad \conj{z} = x - iy$ 
    \item 
    module de $z \in \Cx \quad \module{z} = \sqrt{x^2 + y^2} \in \R_+$
    \item 
    représentation polaire de $z \in \Cx^* \quad z = \module{z} e^{i\theta} = \module{z} (\cos \theta + i \sin \theta)$
    \item 
    $\theta$ est appelé l'argument de $z$ et est noté $\arg z$
\end{itemize}

\begin{remark}
    Pour $z \in \Cx^*$:
    \begin{itemize}
        \item 
        L'argument de $z$ est défini à $2k\pi$ près avec $k \in \Z$
        \item 
        Par convention, la \textbf{valeur (détermination) principale} de l'argument de $z$ est l'unique angle $\theta \in ] -\pi ; \pi ]$ tel que $\frac{z}{\module{z}} = \cos \theta + i \sin \theta$
    \end{itemize}
\end{remark}


\section{Fonctions complexes}

\subsection{Définitions}

\begin{definition}
    Une \textbf{fonction d'une variable complexe} à valeur dans $\Cx$ s'écrit:
    
    \[
    f: \begin{array}{{rcl}}
    \Cx & \longrightarrow & \Cx \\
    z = x+iy & \longmapsto & f(z) = u(x,y) + i v(x,y) \\
    \end{array}
    \]
    
    où
    
    \[
    u: \begin{array}{{rcl}}
    \R^2 & \longrightarrow & \R \\
    (x,y) & \longmapsto & u(x,y) \\
    \end{array}
    \textrm{ et }\enspace
    v: \begin{array}{{rcl}}
    \R^2 & \longrightarrow & \R \\
    (x,y) & \longmapsto & v(x,y) \\
    \end{array}
    \]
    
    sont deux fonctions à valeurs réelles qui s'appellent respectivement la partie réelle de $f$ (on note $u = \Rep f$) et la partie imaginaire de $f$ (on note $v = \Imp f$).
\end{definition}

\begin{remark}
    Les variables $x \in \R$ et $y \in \R$ des fonctions $u$ et $v$ sont les parties réelles et imaginaires de la variable $z \in \Cx$ de la fonction $f$.
\end{remark}

\subsection{Exemples}

\begin{example}\hfill
\begin{enumerate}[label=\arabic{enumi})]
    \item 
    \[
    f: \begin{array}{{rcl}}
    \Cx & \longrightarrow & \Cx \\
    z = x+iy & \longmapsto & f(z) = \conj{z} = x - iy \\
    \end{array}
    \]
    
    On a $u(x,y) = x \enspace$ et $v(x,y)=-y$.
    
    \item 
    \[
    f: \begin{array}{{rcl}}
    \Cx & \longrightarrow & \Cx \\
    z = x+iy & \longmapsto & f(z) = z^2 = (x + iy)^2 = x^2 - y^2 + 2ixy \\
    \end{array}
    \]
    
    On a $u(x,y) = x^2 - y^2 \enspace$ et $v(x,y)= 2xy$.
    
    \item 
    \[
    f: \begin{array}{{rcl}}
    \Cx^* & \longrightarrow & \Cx \\
    z = x+iy & \longmapsto & f(z) = \frac{1}{z}
    = \frac{1}{x + iy}
    = \frac{x - iy}{(x + iy)(x - iy)}
    = \frac{x - iy}{x^2 + y^2} \\
    \end{array}
    \]
    
    On a $u(x,y) = \frac{x}{x^2 + y^2} \enspace$ et $v(x,y)= -\frac{y}{x^2 + y^2}$.
    
    \item 
    Pour $z = x + iy \in \Cx$, la fonction exponentielle est définie par:
    
    \[e^z = e^{x + iy} := e^x(\cos y + i \sin y) \in \Cx^*\]
    
    On a $u(x,y) = e^x \cos y \enspace$ et $v(x,y)= e^x \sin y$.
    
    \begin{remark}
    Contrairement au cas réel, $e^z$ n'est pas bijective sur $\Cx$ car $e^{z + 2ik\pi} = e^z$ $\forall z \in \Z$ \textit{(ex.4, série 3)}.
    En choisissant $y$ tel que $-\pi < y \leq \pi$, la fonction $e^z$ est bijective sur l'ensemble $\left\{z \in \Cx : \Imp z \in \enspace ] -\pi; \pi]\right\}$.
    Avec cette convention, c'est la \textbf{\og restriction bijective de l'exponentielle complexe\fg{}}.
    \end{remark}

    \item 
    Pour $z \in \Cx^*$, la fonction logarithme est définie par:
    
    \[
    \log z := \ln \module{z} + i \arg z
    \]
    
    avec le choix de la valeur principale $\arg z \in \enspace ]-\pi;\pi[$. Avec cette convention, c'est la \textbf{\og détermination principale du logarithme complexe \fg} (correspond à la fonction réciproque de la restriction bijective de l'exponentielle).\\
    En écrivant $z = x + iy$, on a $u(x,y) = \ln \module{x + iy} = \ln \sqrt{x^2 + y^2} \enspace$ et $v(x,y)= \arg (x + iy)$.
    
    \begin{remark}\hfill
    \begin{enumerate}[label=\alph*)]
    \item 
    Les formules valables en analyse réelle ne sont pas nécessairement valables en analyse complexe.
    Par exemple, en général, on a $\log(z_1 z_2) \neq \log z_1 + \log z_2$.
    En effet, pour $z_1 = -1$ et $z_2 = -1$:
    
    \[\log\left[(-1)(-1)\right] = \log(1) := \ln \module{1} + i \arg(1) = 0 + i0 = 0\]
    
    mais
    
    \[\log(-1) + \log(-1) = 2 \log(-1) := 2 \left[\ln \module{-1} + i \arg (-1) \right] = 2i\pi \neq 0 \]
    
    \item 
    La fonction $\log z$ n'est pas continue sur le demi-axe réel négatif.
    En effet, par exemple pour $z = -1$, on considère $t > 0$:
    
    \[z^+_t = -1 + it \enspace \textrm{ et }\enspace
     z^-_t = -1 - it \]
     
    On a $\lim\limits_{t \rightarrow 0^+} z^+_t = -1$ et $\lim\limits_{t \rightarrow 0^+} z^-_t = -1$.

    \begin{note}
    Pour $z = x + iy$, on a la valeur principale:
    
    \[
    \arg z =
    \left\{
    \begin{array}{rcl}
    \pi + \Arctg\frac{y}{x} & \textrm{si} & x < 0, y > 0\\
    -\pi + \Arctg\frac{y}{x} & \textrm{si} & x < 0, y < 0\\
    \end{array}
    \right.
    \]
    \end{note}

    Avec la définition du logarithme, on a:
    
    \begin{align*}
    \lim\limits_{t \rightarrow 0^+} \log(z^+_t)
    &=
    \lim\limits_{t \rightarrow 0^+} \ln \module{-1 + it} + i \lim\limits_{t \rightarrow 0^+} \arg(-1 + it)
    \\&=
    \lim\limits_{t \rightarrow 0^+} \ln \sqrt{1 + t^2} + i \lim\limits_{t \rightarrow 0^+} \left[\pi + \Arctg(-t)\right]
    \\&=
    \ln (1) + i \pi = 0 + i\pi = i\pi
    \end{align*}
    
    \begin{align*}
    \lim\limits_{t \rightarrow 0^+} \log(z^-_t)
    &=
    \lim\limits_{t \rightarrow 0^+} \ln \module{-1 - it} + i \lim\limits_{t \rightarrow 0^+} \arg(-1 - it)
    \\&=
    \lim\limits_{t \rightarrow 0^+} \ln \sqrt{1 + t^2} + i \lim\limits_{t \rightarrow 0^+} \left[-\pi + \Arctg(t)\right]
    \\&=
    \ln (1) - i \pi = 0 - i\pi = -i\pi
    \end{align*}
    
    \textbf{Conclusion:} $\lim\limits_{t \rightarrow 0^+} \log(z^+_t) \neq \lim\limits_{t \rightarrow 0^+} \log(z^-_t) \implies$ la fonction $\log z$ n'est pas continue en $z = -1$.
    De façon analogue, on obtient le même résultat $\forall z \in \enspace ]-\infty;0[$.
    $\log z$ \textbf{n'est pas continue} pour $z \in \enspace ]-\infty;0]$
    
    \textbf{Résultat final:} En excluant le demi-axe réel négatif, on a que la fonction $\log z$ est continue sur l'ensemble:
    
    \[
    V = \Cx \enspace \setminus \enspace ]-\infty;0] = \Cx \setminus \left\{z \in \Cx : \Imp z = 0, \Rep z \leq 0\right\}
    \]
    
    C'est la \textbf{\og restriction continue du logarithme complexe \fg}.
    \end{enumerate}
    \end{remark}

    \item 
    Pour $z \in \Cx$, on définit les fonctions trigonométriques et hyperboliques:
    
    \begin{align*}
    \cos z = \frac{e^{iz} + e^{-iz}}{2}
    &\hspace{2cm} \sin z = \frac{e^{iz} - e^{-iz}}{2i}\\
    \cosh z = \frac{e^{z} + e^{-z}}{2}
    &\hspace{2cm} \sinh z = \frac{e^{z} - e^{-z}}{2}
    \end{align*}
    
    \textit{Cf. ex. 1, série 4}
\end{enumerate}
\end{example}


\section{Limites, continuité et dérivabilité}

\subsection{Définitions}

\begin{definition}[9.1, p.67]
    Les notions de topologie (ouverts, fermés, etc.), de limite, de continuité et de dérivabilité sont analogues à celles de l'analyse réelle.
    Soit $f: \Cx \rightarrow \Cx$, alors:
    
    \begin{enumerate}[label=\arabic{enumi})]
    \item 
    $f$ possède une limite $l \in \Cx$ en $z_0 \in \Cx$ (notation $\lim\limits_{z \rightarrow z_0} f(z) = l$) si:\\
    $\forall\, \epsilon > 0 \enspace \exists\, \delta > 0 :\enspace 0 < \module{z - z_0} < \delta \implies \module{f(z) - l} < \epsilon$
    
    \item 
    $f$ est continue en $z_0 \in \Cx$ si $\lim\limits_{z \rightarrow z_0} f(z) = f(z_0)$
    
    \item 
    $f$ est dérivable en $z_0 \in \Cx$ si $\lim\limits_{z \rightarrow z_0} \frac{f(z) - f(z_0)}{z - z_0}$ existe et est finie.
    La limite s'appelle la dérivée de $f$ en $z_0$ et est notée $f'(z_0)$.
    Les règles de dérivation établies dans $\R$ sont valables dans $\Cx$.
    
    \item 
    Étant donné un ouvert $V \subset \Cx$, on dit que la fonction $f: V \rightarrow \Cx$ est \textbf{holomorphe} (ou analytique complexe) dans $V$ si $f$ est \textit{définie et dérivable} $\forall z \in V$.
    \end{enumerate}
\end{definition}

\subsection{Équations de Cauchy-Riemann}

\begin{remark}[sur un abus de notation]
    Étant donné un ouvert $V \subset \Cx$, on l'identifie souvent au sous-ensemble correspondant de $\R^2$, i.e. on écrit indifféremment $z = x + iy \in V (\in \Cx)$ ou $(x,y) \in V (\in \R^2)$ de façon abusive.
\end{remark}

\newpage

\begin{theorem}[de Cauchy-Riemann; 9.2, p.67]
    Soit $V \in \Cx$ un ouvert et soit une fonction $f: V \rightarrow \Cx$, où $u: V \rightarrow \R$ et $v: V \rightarrow \R$ sont respectivement les parties réelles et imaginaires de $f$.
    Alors les deux affirmations suivantes sont équivalentes:
    \begin{enumerate}[label=\arabic{enumi})]
        \item 
        $f$ est holomorphe dans $V$
        \item 
        Les fonctions $u,v \in C^1(V)$ et satisfont les équations de Cauchy-Riemann:
        
        \[
        \frac{\partial u}{\partial x}(x,y) = \frac{\partial v}{\partial y}(x,y)
        \quad,\quad
        \frac{\partial u}{\partial y}(x,y) = - \frac{\partial v}{\partial x}(x,y)
        \]
        
        En particulier, si $f$ est holomorphe dans $V$, alors on a:
        
        \[
        f'(z) = \frac{\partial u}{\partial x}(x,y) + i \frac{\partial v}{\partial x}(x,y) = \frac{\partial v}{\partial y}(x,y) - i \frac{\partial u}{\partial y}(x,y) \quad \forall z = x + iy \in V
        \]
    \end{enumerate}
\end{theorem}

\begin{remark}\hfill
\begin{enumerate}[label=\arabic{enumi})]
    \item 
    Démonstration du Théorème: voir §2.3.4 (fin du chapitre)
    \item 
    Les équations de Cauchy-Riemann sont une condition nécessaire pour que $f$ soit holomorphe mais elles ne sont pas une condition suffisante.
    Si $u$ et $v$ sont continûment dérivables ($u,v \in C^1(V)$), alors elles deviennent une condition suffisante.
    \item 
    Utilité du Théorème: pour qu'une fonction $f$ soit holomorphe dans un ouvert $V$, il suffit de vérifier que les équations de Cauchy-Riemann pour $u = \Rep f \in C^1(V)$ et $v = \Imp f \in C^1(V)$ sont satisfaites dans $V$.
    Si les équations de Cauchy-Riemann ne sont pas vérifiées en $(x_0,y_0) \in V$, alors $f(z_0)$ n'est pas holomorphe en $z_0 = x_0 + iy_0$.
    \item 
    Pour alléger la notation, on écrit:
    
    \[
    u_x = \frac{\partial u}{\partial x}, \quad
    u_y = \frac{\partial u}{\partial y}, \quad
    v_x = \frac{\partial v}{\partial x}, \quad
    v_y = \frac{\partial v}{\partial y}
    \]
    
    Équations de Cauchy-Riemann:
    
    \[
    u_x = v_y \quad, \quad u_y = -v_x
    \]
    
    \textit{Exemples: ex. 1 à 4, série 4}
\end{enumerate}
\end{remark}

\subsection{Exemples}

\begin{example}[1]\hfill
    
    $f(z) = z^2$ définie pour $z = x + iy \in \Cx$.
    
    $f(z) = (x + iy) ^2 = x^2 - y^2 + i2xy$.
    
    \[\implies u(x,y) = x^2 - y^2, \quad v(x,y) = 2xy\]
    
    \[\left.
    \begin{array}{ll}
    u_x(x,y) = 2x &
    u_y(x,y) = -2y \\
    v_x(x,y) = 2y &
    v_y(x,y) = 2x
    \end{array}
    \right\} \implies
    \begin{array}{l}
    u_x = v_y \\
    u_y = -v_x
    \end{array}
    \forall (x,y) \in \Cx\]
    
    CR (Cauchy-Riemann) satisfaites $\forall z \in \Cx \implies f$ holomorphe dans $\Cx$.
    
    De plus: $f'(z) = u_x(x,y) + i v_x(x,y) = 2x + 2iy = 2(x + iy) = 2z$.
\end{example}

\begin{example}[2]\hfill
    
    $f(z) = \conj{z}$ définie pour $z = x + iy \in \Cx$.
    
    $f(z) = \conj{x + iy} = x - iy$.
    
    \[\implies u(x,y) = x, \quad v(x,y) = -y\]
    
    \[\left.
    \begin{array}{ll}
    u_x(x,y) = 1 &
    u_y(x,y) = 0 \\
    v_x(x,y) = 0 &
    v_y(x,y) = -1
    \end{array}
    \right\} \implies
    \begin{array}{l}
    u_x \neq v_y \\
    u_y = -v_x
    \end{array}\]
    
    CR non satisfaites $\implies f$ n'est pas holomorphe dans $\Cx$.
\end{example}

\begin{example}[3]\hfill
    
    $f(z) = e^z$ définie pour $z = x + iy \in \Cx$.
    
    $f(z) = e^x (\cos y + i \sin y) = e^x \cos y + i e^x \sin y$
    
    \[\implies u(x,y) = e^x \cos y, \quad v(x,y) = e^x \sin y\]
    
    \[\left.
    \begin{array}{ll}
    u_x(x,y) = e^x \cos y &
    u_y(x,y) = -e^x \sin y \\
    v_x(x,y) = e^x \sin y &
    v_y(x,y) = e^x \cos y
    \end{array}
    \right\} \implies
    \begin{array}{l}
    u_x = v_y \\
    u_y = -v_x
    \end{array}\]
    
    CR satisfaites $\implies f$ holomorphe dans $\Cx$.
    
    \[f'(z) = u_x(x,y) + i v_x(x,y) = e^x \cos y + i e^x \sin y := e^z\]
\end{example}

\begin{example}[4]\hfill
    
    $f(z) = \log z = \ln \module{z} + i \arg z$ avec la détermination principale définie pour $V = \Cx\, \setminus\, ]-\infty; 0] = \Cx \setminus \{z \in \Cx : \Imp z = 0, \Rep z \leq 0\}$.
    $\log z$ est holomorphe dans $V$ et on a:
    
    \[f'(z) = \frac{1}{z} \enspace \forall z \in V\]
    
    En effet: preuve pour le demi-plan $D = \{z \in \Cx : \Rep z > 0\}$.
    Pour $z = x +iy$ avec $x > 0$ et $y \in \R$.
    On a: $\module{z} = \sqrt{x^2 + y^2}$ et $\arg z = \Arctg \frac{y}{x}$.
    
    Donc $\log z := \ln \sqrt{x^2 + y^2} + i \Arctg \frac{y}{x}.$
    
    \[\implies u(x,y) = \ln \sqrt{x^2 + y^2}, \quad v(x,y) = \Arctg \frac{y}{x}\]
    
    \[\left.
    \begin{array}{ll}
    u_x(x,y) = \frac{\frac{1}{2}\frac{2x}{\sqrt{x^2 + y^2}}}{\sqrt{x^2 + y^2}}
    = \frac{x}{x^2 + y^2} &
    u_y(x,y) = \frac{\frac{1}{2}\frac{2y}{\sqrt{x^2 + y^2}}}{\sqrt{x^2 + y^2}}
    = \frac{y}{x^2 + y^2} \\
    v_x(x,y) = \frac{-\frac{y}{x^2}}{1 + \frac{y^2}{x^2}}
    = - \frac{y}{x^2 + y^2} &
    v_y(x,y) = \frac{\frac{1}{x}}{1 + \frac{y^2}{x^2}}
    =  \frac{x}{x^2 + y^2}
    \end{array}
    \right\} \implies
    \begin{array}{l}
    u_x = v_y \\
    u_y = -v_x
    \end{array}\]
    
    CR satisfaites $\implies f$ holomorphe dans $D$.
    
    De plus:
    
    \begin{align*}
    f'(z) &= u_x(x,y) + i v_x(x,y) = \frac{x}{x^2 + y^2} - i \frac{y}{x^2 + y^2}
    \\&=
    \frac{x - iy}{x^2 + y^2} = \frac{x - iy}{(x + iy)(x - iy)} = \frac{1}{x + iy} = \frac{1}{z}
    \end{align*}
\end{example}

\textit{Cf. ex. 1, série 5}

\subsection{Démonstration des équations de Cauchy-Riemann}

\begin{proof}{'$\implies$'}\hfill

    Soient $z_0 = x_0 + i y_0 \in V$ et $z = (x_0 + \alpha) + i (y_0 + \beta) \in V$ avec $\alpha, \beta \in \R$.
    Puisque $f$ est holomorphe dans $V$, alors $f'(z_0) := \lim\limits_{z \rightarrow z_0} \frac{f(z) - f(z_0)}{z - z_0}$ existe $\forall z_0 \in V$
    On a:
    
    \[\frac{f(z) - f(z_0)}{z - z_0} = \frac{[u(x_0 + \alpha, y_0 + \beta) + iv(x_0 + \alpha, y_0 + \beta)] - [u(x_0,y_0) + iv(x_0,y_0)]}{\alpha + i\beta}\]
    
    \begin{enumerate}[label=\alph*)]
    \item
    En posant $\beta = 0$, on obtient:
    
    \begin{align*}
    f'(z_0) &= \lim\limits_{\alpha \rightarrow 0} \frac{[u(x_0 + \alpha, y_0) + iv(x_0 + \alpha, y_0)] - [u(x_0,y_0) + iv(x_0,y_0)]}{\alpha}
    \\&=
    \lim\limits_{\alpha \rightarrow 0} \frac{u(x_0 + \alpha, y_0)  - u(x_0,y_0)}{\alpha} + i \lim\limits_{\alpha \rightarrow 0} \frac{v(x_0 + \alpha, y_0) - v(x_0,y_0)}{\alpha}
    \\&=
    u_x(x_0,y_0) + i v_x(x_0,y_0)
    \end{align*}
    
    \item
    En posant $\alpha = 0$, on obtient:
    
    \begin{align*}
    f'(z_0) &= \lim\limits_{\beta \rightarrow 0} \frac{u(x_0, y_0 + \beta) - u(x_0,y_0)}{i\beta} + i \lim\limits_{\beta \rightarrow 0} \frac{v(x_0, y_0 + \beta) - v(x_0,y_0)}{i\beta}
    \\&=
    \frac{1}{i}\, u_y(x_0,y_0) + v_y(x_0,y_0) = v_y(x_0,y_0) - i u_y(x_0,y_0)
    \end{align*}
    \end{enumerate}

    Les deux limites existent et sont identiques.

    \[\implies u_x(x_0,y_0) = v_y(x_0,y_0), \quad u_y(x_0,y_0) = -v_x(x_0,y_0) \quad \textrm{(équations de CR)}\]
    
    '$\Longleftarrow$' \hspace{1em} On utilise les développements de Taylor au 1\ier{} ordre de $u(x,y)$ et $v(x,y)$ pour montrer que $f'(z_0)$ existe.
    
    \textit{Cf. ex. 5, série 4 et ex. 2-4, série 5}
\end{proof}

\begin{remark}[finale]
    Affirmer que $f: \Cx \rightarrow \Cx$ est dérivable en $z = x + iy$ \textbf{n'est pas équivalent} au fait que le champ vectoriel $\tilde{f}: \R^2 \rightarrow \R^2$ est continûment dérivable dans le contexte usuel de $\R^2$ (i.e matrice jacobienne $\begin{pmatrix}u_x & u_y\\v_x & v_y\end{pmatrix}$ de $\tilde{f}$ existe avec $u_x, u_y, v_x, v_y$ continues).
    
    Par exemple, $f(z) = \conj{z}$ n'est pas holomorphe dans $\Cx$: si $z_0 = x_0 + i y_0$ et $z = (x_0 + \alpha) + i (y_0 + \beta)$, alors on a:
    
    \[\frac{f(z) - f(z_0)}{z - z_0} = \frac{(x_0 + \alpha) - i(y_0 + \beta) - x_0 + iy_0}{(x_0 + \alpha) + i(y_0 + \beta) - x_0 - iy_0} = \frac{\alpha - i\beta}{\alpha + i\beta} = \left\{\begin{array}{rl}-1 & \textrm{si } \alpha = 0\\1 & \textrm{si } \beta = 0\end{array}\right.\]
    
    \[\lim\limits_{\alpha \rightarrow 0} \frac{f(z) - f(z_0)}{z - z_0} = -1, \quad \lim\limits_{\beta \rightarrow 0} \frac{f(z) - f(z_0)}{z - z_0} = 1\]
    
    $\implies f'(z_0)$ n'existe pas.
    
    Mais $\tilde{f}:\begin{array}{rcl} \R^2 & \longrightarrow & \R^2\\ (x,y) & \longmapsto & (u(x,y),v(x,y)) = (x,-y)\end{array}$
    
    $\implies$ Matrice jacobienne $\begin{pmatrix}u_x & u_y\\v_x & v_y\end{pmatrix} = \begin{pmatrix}1 & 0\\0 & -1 \end{pmatrix}$ existe $\forall (x,y) \in \R^2$
\end{remark}

% !TeX spellcheck = fr_FR
\chapter{Théorème et formule de Cauchy}


\section{Intégration complexe}

\subsection{Définitions et notations}

\begin{definition}[10.1, p.73]\hfill
\begin{enumerate}[label=\arabic{enumi})]
    \item
    $\Gamma \subset \Cx$ est une \textbf{courbe simple régulière} s'il existe un intervalle $[a,b] \subset \R$ et une fonction $\gamma: \begin{array}{ccl} [a,b] & \longrightarrow & \Cx\\ t & \longmapsto & \gamma(t) = \gamma_1(t) + i\gamma_2(t)\end{array}$ telle que:
    
    \begin{itemize}
    \item $\Gamma = \gamma([a,b])$, la courbe $\Gamma$ est l'image de $\gamma$
    \item $\gamma(t_1) = \gamma(t_2) \implies t_1 = t_2 \enspace \forall t_1,t_2 \in [a,b[$
    \item $\gamma \in C^1\left([a,b]\right)$
    \item $\module{\gamma'(t)} = \left[\gamma_1'(t)^2 + \gamma_2'(t)^2\right]^\frac{1}{2} \neq 0 \enspace \forall t \in [a,b]$
    \end{itemize}

    $\gamma$ s'appelle une paramétrisation de $\Gamma$ décrite par $t \in [a,b]$.
    
    \item
    $\Gamma \subset \Cx$ est une courbe simple régulière \textbf{fermée} si de plus $\gamma(a) = \gamma(b)$.
    
    \item
    $\Gamma \subset \Cx$ est une courbe simple régulière \textbf{par morceaux} si $\exists \Gamma_1, \Gamma_2,\ldots,\Gamma_k$ des courbes simples régulières telles que $\Gamma = \bigcup\limits_{j = 1}^k \Gamma_j$.
    
    \begin{note}[Abus de langage et de notation]
        En analyse complexe, on identifie souvent la courbe $\Gamma$ à sa paramétrisation $\gamma$.
        On dit \og soit $\gamma$ une courbe... \fg{} au lieu de \og soit $\Gamma$ une courbe...\fg{}
    \end{note}

    \item
    Si $\Gamma \subset \Cx$ est une courbe simple fermée régulière (par morceaux) de paramétrisation $\gamma$, on note \textbf{l'intérieur} $\inte \Gamma$ (ou aussi $\inte \gamma$) l'ensemble ouvert et borné $V \in \Cx$ dont le bord est $\Gamma$ (i.e. tel que $\bord V = \Gamma$).
    
    Pour l'adhérence de $V$, on écrit $\adh \gamma = \inte \gamma \cup \bord V$.
    
    \begin{note}
        $\gamma$ est dite orientée \textbf{positivement} si le sens de parcours laisse l'intérieur $\inte \gamma$ à gauche.
    \end{note}
    
    \item
    Soit $\Gamma \subset \Cx$ une courbe simple régulière de paramétrisation $\gamma: [a,b] \rightarrow \Cx$ et soit $f: \Gamma \rightarrow \Cx$ une fonction continue.
    L'\textbf{intégrale} de $f$ le long de $\Gamma$ est définie par:
    
    \[\int_\Gamma f(z) \dz = \int_\gamma f(z) \dz = \int_a^b f(\gamma(t)) \gamma'(t) \dt\]
    
    \item
    Si la courbe $\Gamma = \bigcup\limits_{j = 1}^k \Gamma_j$ est simple régulière par morceaux, alors:
    
    \[\int_\Gamma f(z) \dz = \sum_{j=1}^k \int_{\Gamma_k} f(z) \dz\]
\end{enumerate}
\end{definition}

\subsection{Exemples}

\begin{example}
    Calculer $\int_\gamma f(z) \dz$ pour $f(z) = z^2$ et $\gamma$ le demi-cercle unité de rayon 1 centré à l'origine.
    
    \[
    \gamma:
    \begin{array}{ccl}
    [0;\pi] & \longrightarrow & \Cx \\
    \theta & \longmapsto & \gamma(\theta) = e^{i\theta} = \cos \theta + i \sin \theta
    \end{array}
    \]
    
    \[
    \gamma'(\theta) = -\sin \theta + i \cos \theta = i(\cos \theta + i \sin \theta) = i e^{i\theta}
    \]
    
    \begin{align*}
    \implies \int_\gamma f(z) \dz &= \int_0^\pi f(\gamma(\theta)) \gamma'(\theta) \dth = \int_0^\pi \left(e^{i\theta}\right)^2 i e^{i\theta} \dth
    \\&=
    i \int_0^\pi e^{3i\theta} \dth = \left.\frac{1}{3} e^{3i\theta} \right|_0^\pi = \frac{1}{3} \left[e^{3i\pi} - e^{i0}\right]
    \\&=
    \frac{1}{3} (-1-1) = -\frac{2}{3}
    \end{align*}
    
    \textit{Autres exemples: ex.5, série 5}
\end{example}


\section{Théorème de Cauchy}

\subsection{Théorème}

\begin{theorem}[10.2, p.73]
    Soient $D \subset \Cx$ un domaine simplement connexe, $f: D \rightarrow \Cx$ une fonction holomorphe dans $D$ et $\gamma$ une courbe simple régulière fermée contenue dans $D$.
    Alors:
    
    \[\int_\gamma f(z) \dz = 0\]
    
    \textit{Cf. ex. 6, série 5}
\end{theorem}

\begin{terminology}
    On appelle domaine simplement connexe un ensemble ouvert $D \subset \Cx$ qui \og n'a pas de trous \fg{}.
\end{terminology}

\subsection{Exemples}

\begin{example}[1]
    $D = \Cx, \quad f(z) = z^2$ holomorphe dans $D$ et $\gamma$ une courbe simple fermée régulière (par morceaux) \textit{quelconque} dans $D$, alors:
    
    \[\textrm{Thm. de Cauchy } \implies \int_\gamma z^2 \dz = 0\]
    
    Par exemple, si $\gamma(\theta) = e^{i\theta}$ avec $\theta \in [0, 2\pi[$ (cercle unité centré à l'origine), on a bien:
    
    \[
    \int_\gamma z^2 \dz =
    \int_0^{2\pi} \module{e^{i\theta}}^2 i e^{i\theta}\dth =
    i \int_0^{2\pi} e^{3 i\theta}\dth =
    \left.\frac{1}{3} e^{3 i\theta}\right|_0^{2\pi} =
    \frac{1}{3} \left[e^{6 i \pi} - 1\right] =
    \frac{1}{3} \left[1 - 1\right] =
    0
    \]
\end{example}

\begin{example}[2]
    $f(z) = \frac{1}{z}$
    
    \begin{enumerate}[label=\alph*)]
    \item 
    $D = \Cx$ Le Théorème de Cauchy ne s'applique pas car $f$ n'est pas holomorphe en $t = 0$.
    
    $D = \Cx^\ast = \Cx \setminus \{0\}$
    Le Théorème de Cauchy ne s'applique pas non plus car $D$ n'est pas simplement connexe.
    Par exemple, si $\gamma$ le cercle unité centré en $z = 0$, alors:
    
    \[
    \int_\gamma f(z)\dz = \int_\gamma \frac{1}{z} \dz = \ldots = 2i\pi \neq 0
    \]
    
    \item 
    $D = \{z \in \Cx : \Rep z > 0\}$
    Le Théorème de Cauchy s'applique car $D$ est simplement connexe et $f$ est holomorphe dans $D$.
    
    \[\textrm{Thm. de Cauchy } \implies \int_\gamma \frac{1}{z} \dz = 0\]
    
    pour $\gamma \subset D$ courbe simple fermée régulière quelconque.
    Par exemple, si $\gamma$ est le cercle unité centré en $z = 2$, alors:
    
    \[
    \int_\gamma \frac{1}{z} \dz = \ldots = 0
    \]
    
    \textit{Cf. ex.4, série 6, application avec variantes de $\gamma(\theta) = e^{i\theta}$}
    \end{enumerate}
\end{example}

\newpage

\subsection{Démonstration du théorème de Cauchy}

\begin{proof}
    Soient $\gamma \subset D$ une courbe simple régulière fermée
    
    \[\gamma: \begin{array}{ccl} [a,b] & \longrightarrow & \Cx\\ t & \longmapsto & \gamma(t) = \alpha(t) + i\beta(t)\end{array}\]
    
    et $f$ une fonction holomorphe dans $D$ définie par $f(x + iy) = u(x,y) + iv(x,y)$.
    On a:
    
    \begin{align*}
    \int_\gamma f(z) \dz
    =& \int_a^b f\big(\gamma(t)\big) \gamma'(t) \dt
    \\=& \int_a^b \Big[u\big(\alpha(t), \beta(t)\big) + i v\big(\alpha(t), \beta(t)\big)\Big] \cdot \Big[\alpha'(t) + i\beta'(t)\Big] \dt
    \\=& \int_a^b \Big[u\big(\alpha(t), \beta(t)\big) \alpha'(t) - v\big(\alpha(t), \beta(t)\big) \beta'(t)\Big] \dt
      \\&+ i \int_a^b \Big[u\big(\alpha(t), \beta(t)\big) \beta'(t) + v\big(\alpha(t), \beta(t)\big) \alpha'(t)\Big] \dt
    \\=& \underbrace{\int_a^b \left(
    \begin{array}{c}
    u\big(\alpha(t), \beta(t)\big)\\
    - v\big(\alpha(t), \beta(t)\big)
    \end{array} \right) \cdot \left(
    \begin{array}{c}
    \alpha'(t)\\
    \beta'(t)
    \end{array} \right) \dt}_{I_1}
      \\&+ i \underbrace{\int_a^b \left(
      \begin{array}{c}
      v\big(\alpha(t), \beta(t)\big)\\
      u\big(\alpha(t), \beta(t)\big)
      \end{array} \right) \cdot \left(
      \begin{array}{c}
      \alpha'(t)\\
      \beta'(t)
      \end{array} \right) \dt}_{I_2}
    \end{align*}
    
    On a que $I_1 = \int_\gamma F \cd \dl$ est l'intervalle curviligne le long de $\gamma$ du champ vectoriel $F: \R^2 \rightarrow \R^2$ défini par $F(x,y) = \big(u(x,y),-v(x,y)\big)$.
    En appliquant le Théorème de Green, on obtient:
    
    \[
    I_1 = \iint_{\inte\gamma} \rot F(x,y) \, \dx\dy
    = \iint_{\inte\gamma} \big[-v_x(x,y) - u_y(x,y)\big] \, \dx\dy
    \]
    
    $f$ holomorphe dans $D$ et $\gamma \subset D \xRightarrow[\textrm{CR}]{\textrm{Thm}} v_x(x,y) + u_y(x,y) = 0 \enspace \forall(x,y) \in \inte\gamma \implies I_1 = 0$
    
    On a que $I_2 = \int_\gamma G \cd \dl$ est l'intervalle curviligne le long de $\gamma$ du champ vectoriel $G: \R^2 \rightarrow \R^2$ défini par $G(x,y) = \big(v(x,y),u(x,y)\big)$.
    En appliquant le Théorème de Green, on obtient:
    
    \[
    I_2 = \iint_{\inte\gamma} \rot G(x,y) \, \dx\dy
    = \iint_{\inte\gamma} \big[u_x(x,y) - v_y(x,y)\big] \, \dx\dy
    \]
    
    $f$ holomorphe dans $D$ et $\gamma \subset D \xRightarrow[\textrm{CR}]{\textrm{Thm}} u_x(x,y) - v_y(x,y) = 0 \enspace \forall(x,y) \in \inte\gamma \implies I_2 = 0$
    
    \textbf{Conclusion:}
    
    \[\int_\gamma f(z) \dz = I_1 + iI_2 = 0 + i0 = 0\]
\end{proof}

\subsection{Corollaire du Théorème de Cauchy}

\begin{corollary}
    Soient $D_0,D_1,D_2,\ldots,D_m \subset \Cx$ des domaines simplement connexes tels que:
    
    \begin{enumerate}[label=\arabic{enumi})]
    \item 
    $\adh{D_j} \subset D_0 \quad \forall j = 1,\ldots,m$
    \item 
    $\adh{D_j} \cap \adh{D_k} = \emptyset \quad \forall j,k = 1,\ldots,m; \enspace j \neq k$ (domaines disjoints)
    \item 
    $\bord D_j = \gamma_j$ pour $j=0,1,\ldots,m$ sont des courbes simples fermées régulières (par morceaux)
    \end{enumerate}

    Soit $f: D = \adh{D_0} \setminus \bigcup\limits_{j=1}^m D_j \rightarrow \Cx$ une fonction holomorphe dans $D$.
    Alors:
    
    \[
    \int_{\gamma_0} f(z) \dz = \sum_{j=1}^m \int_{\gamma_j} f(z) \dz
    \]
    
    où toutes les courbes $\gamma_j$ sont orientées positivement. % (par rapport à \emph{leurs} intérieurs).
\end{corollary}

\subsubsection*{Justification heuristique du corollaire}

\begin{figure}[h]
    \centering
    \includegraphics[width=.65\linewidth]{img/3_2_4-heur.png}
    \caption{Illustration de la justification heuristique}
    \label{3_2_4-heur}
\end{figure}

\[
D = \adh{D_0} \setminus \big(D_1 \cup D_2\big) = \adh{C_1} \cup \adh{C_2}
\]

Les bords de $\gamma_0, \gamma_1$ et $\gamma_2$ de $D_0,D_1$ et $D_2$ appartiennent à $D$.

\begin{itemize}
    \item 
    D'une part, $f$ holomorphe dans $\adh{C_1}$ et $\adh{C_2}$, $C_1$ et $C_2$ sont fermés $\xRightarrow[\textrm{Cauchy}]{\textrm{Thm.}}$
    
    \[\int_{C_1} f(z) \dz = 0 \textrm{ et } \int_{C_2} f(z) \dz = 0\]
    \item 
    D'autre part, avec $C_1$ et $C_2$ orientés positivement, on a:
    
    \[
    \int_{C_1} f(z) \dz + \int_{C_2} f(z) \dz = \int_{\gamma_0} f(z) \dz + \int_{\gamma_1} f(z) \dz + \int_{\gamma_2} f(z) \dz
    \]
    
    où $\gamma_0$ est orientée \textit{positivement} ($D_0$ à gauche), mais $\gamma_1$ et $\gamma_2$ orientées \textit{négativement} ($D_1$ et $D_2$ à droite).
\end{itemize}

Donc:

\[\int_{\gamma_0} f(z) \dz + \int_{\gamma_1} f(z) \dz + \int_{\gamma_2} f(z) \dz = 0\]

\begin{align*}
\implies \int_{\gamma_0} f(z) \dz &= - \int_{\gamma_1} f(z) \dz - \int_{\gamma_2} f(z) \dz \quad \textrm{orientation négative (dessin)}\\
&= \int_{\gamma_1} f(z) \dz + \int_{\gamma_2} f(z) \dz \quad \textrm{orientation positive (énoncé)}
\end{align*}


\section{Formule intégrale de Cauchy}

\subsection{Énoncé}

\begin{theorem}[10.2, p.73]
    Soient $D \subset \Cx$ un domaine simplement connexe, $f: D \rightarrow \Cx$ une fonction holomorphe dans $D$ et $\gamma$ une courbe simple fermée régulière (par morceaux) orientée positivement contenue dans $D$.
    Alors:
    
    \[
    f(z) = \oipi \int_\gamma \frac{f(\xi)}{\xi - z} \dxi \quad \forall z \in \inte \gamma
    \]
\end{theorem}

\begin{illustration}
    $D = \Cx$
    
    Si $f$ est une fonction holomorphe dans $\Cx$, la valeur de la fonction $f$ en un point $z \in \Cx$ s'obtient en intégrant $\frac{f(\xi)}{\xi - z}$ le long de n'importe quelle courbe $\gamma$ (orientée positivement) telle que $z \in \inte \gamma$.
\end{illustration}

\subsection{Exemples d'utilisation}

\begin{example}[1]
    Soit $\gamma$ une courbe simple fermée régulière.
    Discuter en fonction de $\gamma$ la valeur de l'intégrale:
    
    \[\int_\gamma \frac{\cos 2z}{z} \dz\]
    
    Constatation: la fonction $g(z) = \frac{\cos 2z}{z}$ n'est pas définie en $z = 0$.
    
    Distinction de différents cas:
    
    \begin{description}
    \item[1\ier{} cas $0 \in \gamma$]
    L'intégrale n'est pas définie puisque $g(z) = \frac{\cos 2z}{z}$ n'est pas continue sur $\gamma$.
    
    \item[2\ieme{} cas $0 \notin \adh{\gamma}$]
    La fonction $g(z) = \frac{\cos 2z}{z}$ est holomorphe dans un domaine $D$ simplement connexe tel que $\adh{\gamma} \subset D$.
    Comme $\gamma \subset \adh{\gamma} \subset D$, alors le Théorème de Cauchy s'applique à la fonction $g$ et on trouve:
    
    \[\int_\gamma \frac{\cos 2z}{z} \dz = 0\]
    
    $\forall \gamma$ de ce type (i.e. $0 \notin \adh{\gamma}$).
    
    \item[3\ieme{} cas $0 \in \inte{\gamma}$]
    La fonction $f(\xi) = \cos 2\xi$ est holomorphe dans $\Cx$.
    Comme $\gamma \subset \Cx$, en lui appliquant la formule intégrale de Cauchy pour $z = 0$ (avec $D = \Cx$), on trouve:
    
    \[
    f(0) = \oipi \int_\gamma \frac{\cos 2\xi}{\xi - 0} \dxi
    \implies \int_\gamma \frac{\cos 2\xi}{\xi} \dxi = 2\pi i f(0) = 2\pi i \cos 0 = 2\pi i
    \]
    
    \textbf{Conclusion:} $\int_\gamma \frac{\cos 2z}{z} \dz = 2\pi i \quad \forall \gamma$ de ce type.
    \end{description}
\end{example}

\begin{remark}
    Pour le cercle unité de rayon $1$, on a $\gamma(\theta) = e^{i\theta}$ avec $\theta \in [0,2\pi[$, il faudrait calculer:
    
    \[
    \int_\gamma \frac{\cos 2z}{z} \dz
    = \int_0^{2\pi} \frac{\cos \big(2e^{i\theta}\big)}{e^{i\theta}} i e^{i\theta} \dth
    = i \int_0^{2\pi} \cos \big(2e^{i\theta}\big) \dth
    \]
\end{remark}

\begin{example}[2]
    Calculer
    
    \[\int_\gamma \frac{e^{z^2}}{z + i\pi} \dz\]
    
    où $\gamma$ est le cercle de rayon $4$ centré en $z = 1$.
    
    Utilisation de la formule de Cauchy, constatations:
    
    \begin{enumerate}[label=\arabic{enumi})]
    \item 
    La fonction $g(z) = \frac{e^{z^2}}{z + i\pi}$ n'est pas définie en $z = -i\pi$
    \item 
    $-i\pi \in \inte{\gamma}$
    \end{enumerate}

    On considère $\gamma$ orientée positivement et $f(\xi) = e^{\xi^2}$ qui est holomorphe dans $\Cx$.
    Formule intégrale de Cauchy pour $z = -i\pi$ (avec $D = \Cx$) donne:
    
    \[
    f(-i\pi) = \oipi \int_\gamma \frac{e^{\xi^2}}{\xi + i\pi} \dxi \implies \int_\gamma \frac{e^{\xi^2}}{\xi + i\pi} \dxi = 2\pi i f(-i\pi)
    \]
    
    Mais $f(-i\pi) = e^{(-i\pi)^2} = e^{-\pi^2}$, donc:
    
    \[\int_\gamma \frac{e^{z^2}}{z + i\pi} \dz = 2\pi i e^{-\pi^2}\]
    
    \textit{Autres exemples: ex. 1-4, série 6}
\end{example}

% 3.3.3
\subsection{Démonstration de la formule intégrale de Cauchy}

\begin{proof}
Soit $f$ holomorphe dans $D$ et $\gamma$ une courbe simple fermée régulière orientée positivement et contenue dans $D$.
Soient $z \in \inte \gamma$ et $C$ un cercle de rayon $r$ centré en $z$ orienté positivement tel que $C \subset \inte \gamma$.
On note $V = \adh{\gamma} \setminus \inte C$.

Corollaire du Théorème de Cauchy (§3.2.4) appliqué à la fonction $g(\xi) = \frac{f(\xi)}{\xi - z}$ holomorphe pour $\xi \in V$ 

\begin{align*}
\implies \int_\gamma \frac{f(\xi)}{\xi - z} \dxi
&\overset{\textrm{Cor.}}{\underset{\textrm{Thm. Cauchy}}{=}}
\int_C \frac{f(\xi)}{\xi - z} \dxi
\overset{\ast}{=}
\int_0^{2\pi} \frac{f(z + re^{i\theta})}{re^{i\theta}}ire^{i\theta} \dth
\\&=
i \int_0^{2\pi} f(z + re^{i\theta}) \dth
\end{align*}

\[^\ast \xi(\theta) = z + re^{i\theta},
\quad \theta \in [0, 2\pi[,
\quad \xi'(\theta) = ire^{i\theta}\]

D'une part, on a:

\[
\lim_{r \rightarrow 0} \int_\gamma \frac{f(\xi)}{\xi - z} \dxi = \int_\gamma \frac{f(\xi)}{\xi - z} \dxi \ \textrm{(grandeur indépendante de $r$)}
\]

D'autre part, on a:

\begin{align*}
\lim_{r \rightarrow 0} \int_0^{2\pi} f(z + re^{i\theta}) \dth &=
\int_0^{2\pi} \lim_{r \rightarrow 0} \big[f(z + re^{i\theta})\big] \dth
\overset{f \ \textrm{continue}}{=}
\int_0^{2\pi} f(z) \dth
\\&=
f(z) \int_0^{2\pi} \dth
= 2\pi f(z)
\end{align*}

Égalité des limites:

\[\implies \int_\gamma \frac{f(\xi)}{\xi - z} \dxi = 2\pi i f(z)\]
\end{proof}

% 3.4
\section{Corollaire de la formule intégrale de Cauchy}

\subsection{Énoncé}

\begin{corollary}[10.2, p.73]
    Avec les mêmes hypothèses du §3.3 ($D \subset \Cx$ domaine simplement connexe, $f: D \rightarrow \Cx$ holomorphe dans $D$, $\gamma \subset D$ courbe fermée régulière orientée positivement), on a:
    
    \begin{enumerate}[label=\arabic{enumi})]
        \item 
        $f$ est infiniment dérivable dans $D$
        \item 
        \[
        f^{(n)}(z) = \frac{n!}{2\pi i} \int_\gamma \frac{f(\xi)}{(\xi - z)^{n+1}} \dxi \quad \textrm{pour } n \in \N, \ \forall z \in \inte \gamma
        \]
    \end{enumerate}
\end{corollary}

\begin{comment}\hfill

    \begin{enumerate}[label=\arabic{enumi})]
        \item 
        Pour $n=0$, le corollaire redonne la formule intégrale de Cauchy:
        
        \[
        f(z) = f^{(0)}(z) = \frac{0!}{2\pi i} \int_\gamma \frac{f(\xi)}{\xi - z} \dxi
        \]
        
        \item 
        Résultat remarquable: le corollaire affirme qu'une fonction holomorphe dans $D$ (i.e. dérivable $\forall z \in D$) est en fait infiniment dérivable et que sa $n$-ième dérivée se calcule en dérivant $n$ fois par rapport à $z$ sous l'intégrale de la formule de Cauchy.
        En effet: $\quad ' = \frac{\dd}{\dz}$
        
        \[
        f^{(1)}(z) = \frac{1}{2\pi i} \int_\gamma f(\xi) \left[\frac{1}{\xi - z}\right]' \dxi
        = \frac{1}{2\pi i} \int_\gamma \frac{f(\xi)}{(\xi - z)^2} \dxi = \frac{1!}{2\pi i} \int_\gamma \frac{f(\xi)}{(\xi - z)^2} \dxi
        \]
        
        \[
        f^{(2)}(z) = \frac{1}{2\pi i} \int_\gamma f(\xi) \left[\frac{1}{(\xi - z)^2}\right]' \dxi
        = \frac{2}{2\pi i} \int_\gamma \frac{f(\xi)}{(\xi - z)^3} \dxi = \frac{2!}{2\pi i} \int_\gamma \frac{f(\xi)}{(\xi - z)^3} \dxi 
        \]
        
        \[
        f^{(3)}(z) = \frac{2}{2\pi i} \int_\gamma f(\xi) \left[\frac{1}{(\xi - z)^3}\right]' \dxi
        = \frac{2 \cd 3}{2\pi i} \int_\gamma \frac{f(\xi)}{(\xi - z)^4} \dxi = \frac{3!}{2\pi i} \int_\gamma \frac{f(\xi)}{(\xi - z)^4} \dxi 
        \]
        
        Récurrence sur $n$:
        
        \[
        f^{(n)}(z) = \frac{n!}{2\pi i} \int_\gamma \frac{f(\xi)}{(\xi - z)^{n+1}} \dxi
        \]
    \end{enumerate}
\end{comment}

\subsection{Exemples d'utilisation}

\begin{example}[1]
    Calculer:
    
    \[
    \int_\gamma \frac{z e^{3z + 5}}{(z + 1)^3}\dz \ \textrm{ où }\ \gamma = \big\{z \in \Cx : \module{z - i} = 2\big\}
    \]
    
    Constatations:
    
    \begin{enumerate}[label=\arabic{enumi})]
    \item 
    la fonction $g(z) = \frac{z e^{3z + 5}}{(z + 1)^3}$ n'est pas définie en $z = -1$
    
    \item 
    $\gamma$ est le cercle de rayon $2$ centré en $z_0 = i$ et $-1 \in \inte \gamma$.
    
    On considère $\gamma$ orientée positivement et la fonction $f(\xi) = \xi e^{3\xi + 5}$ qui est holomorphe dans $\Cx$.
    \end{enumerate}

    En appliquant à $f$ le corollaire de la formule de Cauchy pour $z = -1$ et $n = 2$ (avec $D = \Cx$), on obtient:
    
    \[f''(-1) = \frac{2!}{2\pi i} \int_\gamma \frac{\xi e^{3\xi + 5}}{(\xi + 1)^3} \dxi\]
    
    mais $f'(\xi) = \big(\xi e^{3\xi + 5}\big)' = e^{3\xi + 5} + 3 \xi e^{3\xi + 5}$ et $f''(\xi) = 3 e^{3\xi + 5} + 3 e^{3\xi + 5} + 9 \xi e^{3\xi + 5}$
    
    \[ \implies f''(-1) = -3 e^2 \]
    
    \textbf{Donc:}
    
    \[
    \int_\gamma \frac{\xi e^{3\xi + 5}}{(\xi + 1)^3} \dxi
    = -3\pi i e^2
    \]
\end{example}

\begin{example}[2]
    Soit $\gamma$ une courbe simple fermée régulière.
    Discuter en fonction de $\gamma$ la valeur de l'intégrale:
    
    \[\int_\gamma \frac{z^2 \sin z}{2 (z - \frac{\pi}{2})^2} \dz\]
    
    La fonction $g(z) = \frac{z^2 \sin z}{2 (z - \frac{\pi}{2})^2}$ n'est pas définie pour $z = \frac{\pi}{2} \implies$ distinction de plusieurs cas.
    
    \begin{itemize}
        \item \textbf{1\ier{} cas} $\frac{\pi}{2} \in \gamma$
        
        L'intégrale n'est pas définie puisque $g(z) = \frac{z^2 \sin z}{2 (z - \frac{\pi}{2})^2}$ n'est pas continue en $z = \frac{\pi}{2}$
        
        \item \textbf{2\ieme{} cas} $\frac{\pi}{2} \notin \adh{\gamma}$
        
        La fonction $g(z) = \frac{z^2 \sin z}{2 (z - \frac{\pi}{2})^2}$ est holomorphe dans un domaine simplement connexe $D$ tel que $\adh{\gamma} \subset D$. Comme $\gamma \subset \adh{\gamma} \subset D$, alors le Théorème de Cauchy s'applique à $g$ et on trouve:
        
        \[ \int_\gamma \frac{z^2 \sin z}{2 (z - \frac{\pi}{2})^2} \dz = 0 \quad \forall \gamma \textrm{ de ce type} \]
        
        \item \textbf{3\ieme{} cas} $\frac{\pi}{2} \in \inte\gamma$
        
        La fonction $f(\xi) = \frac{\xi^2}{2} \sin \xi$ est holomorphe dans $\Cx$. Comme $\gamma \subset \Cx$, en lui appliquant le corollaire de la formule de Cauchy pour $z = \frac{\pi}{2}$ et $n = 1$ (avec $D = \Cx$), on obtient:
        
        \[
        f'\left(\frac{\pi}{2}\right) = \frac{1!}{2\pi i} \int_\gamma \frac{\frac{\xi^2}{2} \sin \xi}{(\xi - \frac{\pi}{2})^2} \dxi
        \]
        
        mais $f'(\xi) = \big(\frac{\xi^2}{2} \sin \xi\big)' = \frac{2\xi}{2} \sin \xi + \frac{\xi^2}{2} \cos \xi$
        
        \[ \implies f'\left(\frac{\pi}{2}\right) = \frac{\pi}{2} + 0 = \frac{\pi}{2} \]
        
        \textbf{Conclusion:}
        
        \[
        \int_\gamma \frac{z^2 \sin z}{2(z - \frac{\pi}{2})^2} \dz = 2\pi i \frac{\pi}{2} = \pi^2 i
        \]
    \end{itemize}
\end{example}

\textit{Autres exemples: ex. 1-4, série 7}

% !TeX spellcheck = fr_FR
\chapter{Séries de Laurent, pôles et résidus}


\section{Polynôme et série de Taylor d'une fonction holomorphe}

\subsection{Définitions et résultats}

\begin{hypothesis}
    Soit un ouvert $D \subset \Cx$ et $f: \begin{array}{ccc}
    D & \longrightarrow & \Cx\\
    z & \longmapsto & f(z)
    \end{array}$ une fonction holomorphe dans $D$ et $z_0 \in D$.
\end{hypothesis}

\begin{definition}
    Pour $N \in \N$, le \textbf{polynôme de Taylor} de $f$ de degré $N$ en $z_0$ est:
    
    \[
    T_N f(z) = \sum_{n = 0}^N \frac{f^{(n)}(z_0)}{n!}(z - z_0)^n
    \]
\end{definition}

\begin{result}[séries de Taylor]
    Soit $R > 0$ et $D_R(z_0) = \{z \in \Cx : \module{z - z_0} < R \}$ le plus grand disque de rayon $R$ centré en $z_0$ contenu dans $D$.
    
    Convention: si $D = \Cx \implies R = +\infty$ et $D_R(z_0) = \Cx$
    
    Alors:
    
    \begin{enumerate}[label=\arabic{enumi})]
        \item 
        \[
        T f(z) = \lim_{N \rightarrow +\infty} T_N f(z) = \sum_{n = 0}^{+\infty} \frac{f^{(n)}(z_0)}{n!}(z - z_0)^n
        \]
        
        existe et est finie $\forall z \in D_R(z_0)$. L'expression $T f(z)$ s'appelle \textbf{la série de Taylor} de $f$ en $z_0$.
        
        \item 
        De plus, on a $f(z) = T f(z) \quad \forall z \in D_R(z_0)$
        
        $R$ est appelé \textbf{le rayon de convergence} de la série de Taylor.
        
        \item 
        Les coefficients de la série de Taylor sont reliés à la formule de Cauchy par le corollaire du §3.4.
        On a:
        
        \[ \frac{f^{(n)}(z_0)}{n!} = \frac{1}{2\pi i} \int_\gamma \frac{f(\xi)}{(\xi - z_0)^{n+1}} \dxi \]
        
        où $\gamma \subset D_R(z_0)$ est une courbe simple fermée régulière orientée positivement telle que $z_0 \in \inte \gamma$.
    \end{enumerate}
\end{result}

\subsection{Exemples}

\begin{example}[1]
    \[f(z) = e^z\]
    
    est holomorphe dans $\Cx$.
    On a $f^{(n)}(z) = e^z$ et $f^{(n)}(0) = 1 \quad \forall n \in N$.
    
    Donc:
    
    \[ e^z = \sum_{n=0}^{+\infty} \frac{z^n}{n!} \quad \forall z \in \Cx \]
\end{example}

\begin{example}[2]
    \[f(z) = \frac{1}{1-z}\]
    
     est holomorphe dans $D = \Cx \setminus \{1\}$.
    
    Le plus grand disque centré en $z_0 = 0$ contenu dans $D$ est $D_1(0) = \{ z \in \Cx : \module{z} < 1 \}$.
    
    On a $f^{(n)}(z) = \frac{n!}{(1-z)^{n+1}}$ et $f^{(n)}(0) = n! \ \forall n \in \N$.
    
    Donc:
    
    \[
    \frac{1}{1-z} = \sum_{n=0}^{+\infty} z^n \quad \forall z \in \Cx, \module{z} < 1
    \]
    
    \textbf{\og Série géométrique \fg{}} avec rayon de convergence $R = 1$.
\end{example}

\begin{example}[3]
    \[f(z) = \frac{1}{1 + z^2}\]
    
    est holomorphe dans $D = \Cx \setminus \{-i; i\}$.
    
    Le plus grand disque centré en $z_0 = 0$ et contenu dans $D$ est $D_1(0) = \{z \in \Cx : \module{z} < 1\}$.
    
    On a:
    
    \[
    \frac{1}{1 + z^2}
    = \frac{1}{1 - \left(-z^2\right)}
    \overset{\textrm{Ex. 2}}{=}
    \sum_{n=0}^{+\infty} (-z^2)^n
    = \sum_{n=0}^{+\infty} (-1)^n z^{2n} \quad \forall z \in \Cx, \module{z} < 1
    \]
    
    Le rayon de convergence $R = 1$
\end{example}

\textit{Autre exemple: ex. 5, série 7}

\subsection{Applications}

\begin{enumerate}[label=\arabic{enumi})]
    \item Règle de l'Hôpital
    
    Soit $z_0 \in \Cx$ et $f, g$ deux fonctions holomorphes au voisinage de $z_0$ telles que $f(z_0) = 0$, $g(z_0) = 0$ et $g'(z_0) \neq 0$.
    Alors:
    
    \[ \lim_{z \rightarrow z_0} \frac{f(z)}{g(z)} = \lim_{z \rightarrow z_0} \frac{f'(z)}{g'(z)} = \frac{f'(z_0)}{g'(z_0)} \]
    
    \textit{Preuve: ex.4, série 8}
    
    \item Théorème de Liouville
    
    Soit $f: \Cx \rightarrow \Cx$ une fonction \textit{bornée} et holomorphe \textit{dans} $\Cx$, alors $f$ est constante.
    
    \textit{Preuve: corrigé de l'ex. 18, p.248 (§11.3)}
\end{enumerate}


\section{Développement et série de Laurent d'une fonction holomorphe}

\subsection{Problématique, définitions et résultats}


\begin{motivation}\hfill
    
    Le développement de Taylor d'une fonction $f$ donne une série en puissances \textbf{positives} de $z - z_0$ au voisinage d'un point $z_0$ où $f$ \textbf{est} holomorphe.
    
    \textbf{But:} généralisation avec un développement en puissances \textbf{positives} et \textbf{négatives} de $z - z_0$ où $z_0$ peut être une \textbf{singularité} de $f$.
\end{motivation}


\begin{hypothesis}\hfill
    
    Soit $D \subset \Cx$ un domaine simplement connexe, $z_0 \in D$ et $f: D \setminus \{z_0\} \rightarrow \Cx$ une fonction holomorphe dans $D \setminus \{z_0\}$.
\end{hypothesis}

\begin{definition}[11.1, p.78]
    Pour $N \in \N$, le développement de Laurent de $f$ de degré $N$ au voisinage de $z_0$ est:
    
    \[ L_N f(z) = \sum_{n = -N}^{N} c_n (z - z_0)^n \]
    
    \[ = \frac{c_ {-N}}{(z - z_0)^N} + \ldots + \frac{c_ {-1}}{z - z_0} + c_0 + c_1 (z - z_0) + \ldots + c_n (z - z_0)^N \]
    
    avec
    
    \[ c_n = \oipi \int_\gamma \frac{f(\xi)}{(\xi - z_0)^{n + 1}} \dxi \]
    
    où $\gamma \subset D$ est une courbe simple fermée régulière (par morceaux) orientée positivement telle que $z_0 \in \inte \gamma$.
\end{definition}

\begin{result}[11.2, p.80]
    Soit $R > 0$ et $D_R(z_0) = \{ z \in \Cx : \module{z - z_0} < R \}$ le plus grand disque de rayon $R$ centré en $z_0$ et contenu dans $D$.
    Alors:
    
    \begin{enumerate}[label=\arabic{enumi})]
        \item 
        \[ Lf(z) = \lim_{N \rightarrow \infty} L_N f(z) = \sum_{n = -\infty}^\infty c_n (z - z_0)^n \]
        
        existe et est finie pour tout $z \in D_R(z_0) \setminus \{z_0\}$.
        L'expression $Lf(z)$ s'appelle la \textbf{série de Laurent} de $f$ au voisinage de $z_0$.
        
        \item 
        De plus, on a $f(z) = Lf(z) \quad \forall z \in D_R(z_0) \setminus \{z_0\}$ et $R$ est appelé le \textbf{rayon de convergence} de la série de Laurent.
    \end{enumerate}
\end{result}

\begin{remark}\hfill
    
    \begin{enumerate}[label=\alph*)]
        \item 
        La série de Laurent de $f$ peut s'écrire sous la forme suivante:
        
        \[ Lf(z) = \sum_{n = -\infty}^{-1} c_n(z - z_0)^n + \sum_{n = 0}^{\infty} c_n(z - z_0)^n \]
        
        \begin{itemize}
        \item 
        La première série
        
        \[
        \sum_{n = -\infty}^{-1} c_n(z - z_0)^n =
        \sum_{n = 1}^{\infty} c_{-n}(z - z_0)^{-n} =
        \frac{c_{-1}}{z - z_0} + \frac{c_{-2}}{(z - z_0)^2} + \ldots
        \]
        
        s'appelle la \textbf{partie singulière} de la série de Laurent.
        
        \item 
        La deuxième série
        
        \[
        \sum_{n = 0}^{\infty} c_n(z - z_0)^n =
        c_0 + c_1(z - z_0) + c_2(z - z_0)^2 + \ldots
        \]
        
        s'appelle la \textbf{partie régulière} de la série de Laurent.
        \end{itemize}

        \item 
        \textit{Si} $f: D \rightarrow \Cx$ est holomorphe en $z_0$, alors la série de Laurent coïncide avec la série de Taylor.
        
        En effet, la partie singulière de la série de Laurent est nulle puisque:
        
        Pour $n = 1, 2, 3, \ldots$, on a:
        
        \[ c_{-n} =  \oipi \int_\gamma \frac{f(\xi)}{(\xi - z_0)^{-n + 1}} \dxi
        = \oipi \int_\gamma f(\xi)(\xi - z_0)^{n - 1} \dxi
        = 0 \]
        
        par le Théorème de Cauchy (§3.2), car $f(\xi)(\xi - z_0)^{n - 1}$ est holomorphe dans $D$.
        
        Les coefficients de la partie régulière donnent la série de Taylor car:
        
        Pour $n = 1, 2, 3, \ldots$, on a:
        
        \[ c_{n} =  \oipi \int_\gamma \frac{f(\xi)}{(\xi - z_0)^{n + 1}} \dxi
        = \frac{f^{(n)}(z_0)}{n!}\]
        
        par le corollaire de la formule intégrale de Cauchy (§3.4), car $f(\xi)$ est holomorphe dans $D$.
    \end{enumerate}
\end{remark}

\subsection{Définitions issues de la série de Laurent}

\begin{note}
    Ces définitions sont issues de la \textbf{Définition 11.2} page 80 du livre de référence.
\end{note}

\begin{definition}[1]
    $z_0 \in \Cx$ est un \textbf{point régulier} de $f \iff$ la partie singulière de la série de Laurent au voisinage de $z_0$ est nulle.
    
    C'est-à-dire:
    
    \[ Lf(z) = Tf(z) = \sum_{n = 0}^{\infty} \frac{f^{(n)}(z_0)}{n!} (z - z_0)^n \]
\end{definition}

\begin{definition}[2]
    Soit $m \in \N^\ast$, $z_0 \in \Cx$ est \textbf{un pôle d'ordre $m$} de $f \iff c_{-m} \neq 0$ et $c_{-k} = 0 \quad \forall k \geq m + 1$.
    
    C'est-à-dire:
    
    \[ Lf(z) = \frac{c_{-m}}{(z-z_0)^m} + \ldots + \frac{c_{-1}}{z-z_0} + \sum_{n = 0}^{\infty} c_n(z - z_0)^n \]
\end{definition}

\begin{definition}[3]
    $z_0 \in \Cx$ est une \textbf{singularité essentielle} (isolée) de $f \iff c_{-n} \neq 0$ pour une infinité d'indices $n$.
    
    C'est-à-dire:
    
    \[ Lf(z) = \sum_{n = 1}^{\infty} \frac{c_{-n}}{(z - z_0)^{n}} + \sum_{n = 0}^{\infty} c_n(z - z_0)^n \]
\end{definition}

\begin{definition}[4]
    Le \textbf{résidu} de $f$ en $z_0$, noté $\Res_{z_0}(f)$, est la valeur du coefficient $c_{-1}$ de la série de Laurent de $f$ au voisinage de $z_0$.
    
    C'est-à-dire:
    
    \[ \Res_{z_0}(f) := c_{-1} = \oipi \int_\gamma f(\xi) \dxi \]
    
    où $\gamma \subset D$ avec $z_0 \in \inte \gamma$
\end{definition}

\subsection{Exemples}

\begin{example}[1]
    Soit
    
    \[f(z) = \frac{1}{z}\]
    
    ($f$ holomorphe dans $\Cx \setminus \{0\}$)
    
    \begin{enumerate}[label=\alph*)]
        \item 
        Au voisinage de $z_0 = 0$, on a $Lf(z) = \frac{1}{z} + 0$
        
        $c_{-1} = 1$ et $c_{-n} = 0$ pour $n \geq 2 \implies z_0$ est un pôle d'ordre 1 de $f$ et $\Res_0(f) := c_{-1} = 1$
        
        \item 
        Au voisinage de $z_0 = 1$, on a:
        
        \[ \frac{1}{z} = \frac{1}{1 - (1 - z)} = \sum_{n=0}^\infty (1-z)^n = \sum_{n=0}^\infty (-1)^n (z-1)^n = Tf(z) = Lf(z) \]
        
        \textit{(cf. série géométrique, donc le rayon de convergence est de 1)}
        
        Partie singulière nulle $\implies z_0 = 1$ est un point régulier de $f$ et $\Res_1(f) = 0$
    \end{enumerate}
\end{example}

\begin{example}[2]
    \[ f(z) = \frac{1}{z^3} + \frac{2}{z}\]
    
    ($f$ holomorphe dans $\Cx \setminus \{0\}$).
    Au voisinage de $z_0 = 0$, on a:
    
    \[ Lf(z) = \frac{1}{z^3} + \frac{2}{z} + 0 \]
    
    On a $c_{-1} = 2$, $c_{-2} = 0$, $c_{-3} = 1$ et $c_{-n} = 0$ pour $n \geq 4$.
    
    $\implies z_0 = 0$ est un pôle d'ordre 3 de $f$ et $\Res_0(f) := c_{-1} = 2$
\end{example}

\begin{example}[3]
    Soit
    
    \[ f(z) = \frac{1}{z^2 + z} \]
    
    ($f$ holomorphe dans $\Cx \setminus \{-1,0\}$).
    Au voisinage de $z_0 = 0$, on a:
    
    \begin{align*}
        \frac{1}{z^2 + z} &= \frac{1}{z(z+1)} = \frac{1}{z} - \frac{1}{z + 1} = \frac{1}{z} - \frac{1}{1 - (-z)}
        \\&= \frac{1}{z} - \sum_{n = 0}^\infty (-z)^n
        = \frac{1}{z} - \sum_{n = 0}^\infty (-1)^n z^n
        = \frac{1}{z} + \sum_{n = 0}^\infty (-1)^{n+1} z^n
    \end{align*}
    
    $\implies c_{-1} = 1$ et $c_{-n} = 0$ pour $n \geq 2 \implies z_0 = 0$ est un pôle d'ordre 1 de $f$ et $\Res_0(f) = 1$
\end{example}

\begin{example}[4]
    \[ f(z) = \frac{\sin z}{z},\ g(z) = \frac{\cos z}{z} \]
    
    ($f$ et $g$ holomorphes dans $\Cx \setminus \{0\}$).
    Au voisinage de $z_0 = 0$, on a:
    
    \begin{enumerate}[label=\alph*)]
        \item 
        \begin{align*}
        \frac{\sin z}{z} &= \frac{1}{z} \sin z = \frac{1}{z} \sum_{n = 0}^\infty (-1)^n \frac{z^{2n + 1}}{(2n + 1)!} = \sum_{n = 0}^\infty (-1)^n \frac{z^{2n}}{(2n + 1)!}
        \\&= 0 + 1 - \frac{z^2}{3!} + \frac{z^4}{5!} + \ldots = Lf(z)
        \end{align*}
        
        $\implies z_0 = 0$ est un \textbf{point régulier} de $f$ et $\Res_0(f) = c_{-1} = 0$ ($z_0 = 0$ est une singularité éliminable).
        
        \item 
        \begin{align*}
        \frac{\cos z}{z} &= \frac{1}{z} \cos z = \frac{1}{z} \sum_{n = 0}^\infty (-1)^n \frac{z^{2n}}{(2n)!} = \sum_{n = 0}^\infty (-1)^n \frac{z^{2n - 1}}{(2n)!}
        \\&= \frac{1}{z} + \sum_{n = 1}^\infty (-1)^n \frac{z^{2n - 1}}{(2n)!}
        = \frac{1}{z} - \frac{z}{2!} + \frac{z^3}{4!} - \ldots = Lf(z)
        \end{align*}
        
        $\implies z_0 = 0$ est un pôle d'ordre 1 de $g$ et $\Res_0(g) = 1$
    \end{enumerate}
\end{example}

\begin{example}[5]
    \[ f(z) = e^{\frac{1}{z}} \]
    
    ($f$ holomorphe dans $\Cx \setminus \{0\}$).
    Au voisinage de $z_0 = 0$, on a:
    
    \[
    e^{\frac{1}{z}} = \sum_{n=0}^\infty \frac{1}{n!} \left(\frac{1}{z}\right)^n = 1 + \sum_{n=1}^\infty \frac{1}{n!}z^{-n} = Lf(z)
    \]
    
    $c_{-n} = \frac{1}{n!} \neq 0 \ \forall n \geq 1 \implies z_0$ est une singularité essentielle de $f$ et $\Res_0(f) = c_{-1} = \frac{1}{1!} = 1$.
\end{example}


\section{Étude des pôles d'une fonction et calcul des résidus}

\subsection{Méthodes de détection des pôles}

\begin{note}
    Ces résultats sont issus de la \textbf{Proposition 11.3} page 81 du livre de référence.
\end{note}

\begin{definition}
    Soient $n \in N^\ast$ et $z_0 \in \Cx$. $z_0$ est un \textbf{zéro d'ordre} $n$ de $f$ lorsque:
    
    \[ f(z_0) = f^{(1)}(z_0) = f^{(2)}(z_0) = \ldots = f^{(n-1)}(z_0) = 0 , \quad \textrm{mais } f^{(n)}(z_0) \neq 0 \]
\end{definition}

\begin{convention}
    Si $z_0$ n'est pas un zéro de $f$, alors $f(z_0) \neq 0$ et puisque $f^{(0)}(z_0) = f(z_0) \neq 0$, en posant $n=0$, on dit que \og $z_0$ est un zéro d'ordre 0 \fg{}.
\end{convention}

\begin{method}\hfill
    
    \begin{enumerate}[label=\alph*)]
    \item 
    Soit $f(z) = \frac{p(z)}{q(z)}$ où $p$ et $q$ sont des fonctions holomorphes au voisinage de $z_0 \in \Cx$ qui est un zéro d'ordre $k$ de $p$ et un zéro d'ordre $\ell$ de $q$.
    Deux cas sont possibles:
    \begin{definition}
        \item[Cas 1:]
        si $\ell > k$, alors $z_0$ est un pôle d'ordre $\ell - k$ de $f$.
        \item[Cas 2:]
        si $\ell \leq k$, alors $z_0$ est un point régulier de $f$.
        On dit que $z_0$ est une \textbf{singularité éliminable} en posant $f(z_0) = \lim\limits_{z \rightarrow z_0} \frac{p(z)}{q(z)}$.
    \end{definition}

    \item 
    Soit $f$ une fonction holomorphe dans $D \setminus \{z_0\}$, soit $m \in \N^\ast$ et
    
    \[ L = \lim\limits_{z \rightarrow z_0} \left[ (z - z_0)^m f(z) \right] \]
    
    Si $L$ est finie et $L \neq 0$, alors $z_0$ est un pôle d'ordre $m$ de $f$.
    \end{enumerate}
\end{method}

\begin{example}\hfill

    \begin{enumerate}[label=\arabic{enumi})]
        \item 
        \[ f(z) =  \frac{\sin z}{z}, \quad z_0 = 0 \]
        
        Avec $p(z) = \sin z$ et $q(z) = z$, on a $p(0) = \sin(0) = 0$, $p'(0) = \cos(0) = 1$, $q(0) = 0$, $q'(0) = 1$.
        Alors $k = \ell = 1$.
        Donc $z_0$ est un point régulier.
        C'est une singularité éliminable en posant:
        
        \[
        f(z) =
        \left\{
        \begin{array}{cc}
        \frac{\sin z}{z} & \textrm{pour } z \neq 0\\
        \lim\limits_{z \rightarrow 0} \frac{\sin z}{z} = 1 & \textrm{pour } z = 0\\
        \end{array}
        \right.
        \]
        
        \item 
        \[ f(z) =  \frac{z}{\sin^2 z}, \quad z_0 = 0 \]
        
        Avec $p(z) = z$ et $q(z) = \sin^2 z$, on a $p(0) = 0$, $p'(0) = 1 \neq 0$, $q(0) = 0$, $q'(z) \Big|_{z=0} = 2 \sin z \cos z \Big|_{z=0} = 0$, $q''(z) \Big|_{z=0} = 2\cos^2 z - 2\sin^2 z \Big|_{z=0} = 2 \neq 0$.
        Alors $k = 1$ et $l = 2$.
        Donc $z_0 = 0$ est un pôle d'ordre $\ell - k = 1$ de $f$.
        
        \item 
        \[ f(z) = \frac{\sin (z - \pi)}{(z - \pi)^3} \]
        
        $z_0 = \pi$ est un pôle d'ordre 2 de $f$ car
        
        \[ \lim\limits_{z \rightarrow z_0} \left[ (z - \pi)^2 f(z) \right]
        = \lim\limits_{z \rightarrow z_0} \left[ (z - \pi)^2 \frac{\sin (z - \pi)}{(z - \pi)^3} \right]
        = \lim\limits_{z \rightarrow z_0} \frac{\sin (z - \pi)}{z - \pi} = 1\]
        
        Par ailleurs, on constate
        
        \[ \lim\limits_{z \rightarrow z_0} \left[ (z - \pi)^3 f(z) \right]
        = \lim\limits_{z \rightarrow z_0} \left[ (z - \pi)^3 \frac{\sin (z - \pi)}{(z - \pi)^3} \right]
        = \lim\limits_{z \rightarrow z_0} \sin (z - \pi) = 0 \]
        
        \[ \lim\limits_{z \rightarrow z_0} \left[ (z - \pi) f(z) \right]
        = \lim\limits_{z \rightarrow z_0} \left[ (z - \pi) \frac{\sin (z - \pi)}{(z - \pi)^3} \right]
        = \lim\limits_{z \rightarrow z_0} \frac{1}{z - \pi} \lim\limits_{z \rightarrow z_0} \frac{\sin (z - \pi)}{z - \pi} = \infty \]
    \end{enumerate}
\end{example}

\begin{remark}
    Les preuves des critères a) et b) découlent du développement en série de Laurent et de la définition de pôle (§4.2.1 et §4.2.2).
\end{remark}

\subsection{Formules de calcul du résidu d'une fonction}

\begin{note}
    Ces résultats sont issus des \textbf{Propositions 11.4, 11.5} page 81 du livre de référence.
\end{note}

\begin{method}\hfill
    
    \begin{enumerate}[label=\alph*)]
    \item 
    Soit $f$ une fonction holomorphe dans $D\setminus \{ z_0 \}$, soit $m \in \N^\ast$.
    Si $z_0$ est un pôle d'ordre m de $f$, alors:
    
    \[ \Res_{z_0}(f) = \frac{1}{(m - 1)!} \lim\limits_{z \rightarrow z_0} \frac{\dd^{m-1}}{\dz^{m - 1}} \left[ (z - z_0)^m f(z) \right] \]
    
    \item 
    Soit $f(z) = \frac{p(z)}{q(z)}$ où $p$ et $q$ sont des fonctions holomorphes au voisinage de $z_0 \in \Cx$ telles que $z_0$ est un zéro d'ordre 1 de $q(z_0)$ et $p(z_0) \neq 0$.
    Alors:
    
    \[ \Res_{z_0}(f) = \frac{p(z_0)}{q'(z_0)} \]
    \end{enumerate}
\end{method}

\begin{example}\hfill
    
    \begin{enumerate}[label=\arabic{enumi})]
    \item 
    \[ f(z) = \frac{3z^2}{z + 2} \]
    
    alors $z_0 = -2$ est un pôle d'ordre 1 de $f$.
    
    \[ \implies \Res_{-2}(f) = \lim\limits_{z \rightarrow -2} (z + 2) \frac{3z^2}{z + 2} = 12 \]
    
    \item 
    \[ f(z) = \frac{e^z}{(z - 5)^3} \]
    
    alors $z_0 = 5$ est un pôle d'ordre 3 de $f$.
    
    \[ \implies \Res_{5}(f) = \frac{1}{2!} \lim\limits_{z \rightarrow 5} \frac{\dd^2}{\dz^2} \left[ (z - 5)^3 \frac{e^z}{(z - 5)^3}  \right] = \frac{1}{2} \lim\limits_{z \rightarrow 5} e^z = \frac{e^5}{2} \]
    
    \item 
    \[ f(z) = \frac{\sin z}{z^2 + 1} \]
    
    comme $z^2 + 1 = (z - i)(z + i)$, alors $z_0 = i$ et $z_0 = -i$ sont deux pôles d'ordre 1 de $f$.
    
    \[ \implies \Res_{i}(f) = \lim\limits_{z \rightarrow i} (z - i) \frac{\sin z}{(z - i)(z + i)} = \frac{\sin i}{2i} \]
    
    \[ \implies \Res_{-i}(f) = \lim\limits_{z \rightarrow -i} (z + i) \frac{\sin z}{(z - i)(z + i)} = \frac{\sin (-i)}{-2i} = \frac{\sin i}{2i} \]
    
    \item 
    \[ f(z) = \frac{3z^2}{z + 2} \]
    
    et $z_0 = -2$.
    Avec $p(z) = 3z^2$ et $q(z) = z + 2$, on a que $z_0 = -2$ est un zéro d'ordre 1 de $q$ avec $p(-2) = 12$ et $q'(-2) = 1$.
    
    \[ \implies \Res_{-2}(f) = \frac{12}{1} = 12 \]
    \end{enumerate}
    
\end{example}

\subsection{Démonstration des formules}

\begin{proof}
    
    \begin{enumerate}[label=\alph*)]
    \item 
    Si $m = 1$, la série de Laurent de $f$ au voisinage de $z_0$ donne:
    
    \[ f(z) = \frac{c_{-1}}{z - z_0} + \sum_{n = 0}^\infty c_n (z - z_0)^n \]
    
    Alors $(z - z_0) f(z) = c_{-1} + F(z)$ avec $F(z) = \sum\limits_{n = 0}^\infty c_n (z - z_0)^{n + 1}$
    
    \[ \implies \lim_{z \rightarrow z_0} \left[ (z - z_0) f(z) \right] = c_{-1} + \lim_{z \rightarrow z_0} F(z) = c_{-1} =: \Res_{z_0}(f) \]
    
    Si $m = 2$, la série de Laurent de f au voisinage de $z_0$ donne:
    
    \[ f(z) = \frac{c_{-2}}{(z - z_0)^2} + \frac{c_{-1}}{z - z_0} + \sum_{n = 0}^\infty c_n (z - z_0)^n \]
    
    \[ (z - z_0)^2f(z) = c_{-2} + c_{-1}(z - z_0) + \sum_{n = 0}^\infty c_n (z - z_0)^{n + 2} \]
    
    et
    
    \begin{align*}
    \frac{\dd}{\dz}\left[ (z - z_0)^2 f(z) \right]
    &= \frac{\dd}{\dz}\left[ c_{-2} + c_{-1}(z - z_0) + \sum_{n = 0}^\infty c_n (z - z_0)^{n + 2} \right]
    \\&= 0 + c_{-1} + \sum_{n = 0}^\infty c_n (n + 2) (z - z_0)^{n + 1}
    \\&= c_{-1} + G(z)
    \end{align*}
    
    où
    
    \[ G(z) = \sum_{n = 0}^\infty c_n (n + 2) (z - z_0)^{n + 1} \]
    
    \newpage
    
    Donc
    
    \[
    \lim_{z \rightarrow z_0} \frac{\dd}{\dz}\left[ (z - z_0)^2 f(z) \right] = c_{-1} + \lim_{z \rightarrow z_0} G(z) = c_{-1} =: \Res_{z_0}(f)
    \]
    
    Si $m \geq 3$, raisonnement analogue et preuve par récurrence qui fait apparaître le terme $\frac{1}{(m - 1)!}$ dans la formule
    
    \item 
    On applique la formule a) à $f(z) = \frac{p(z)}{q(z)}$ où $z_0$ est un pôle d'ordre 1 de $f$.
    
    \textit{Cf. ex. 5, série 9}
    \end{enumerate}
\end{proof}

% !TeX spellcheck = fr_FR
\chapter{Théorème des résidus et applications au calul d'intégrales réelles}


\section{Théorème des résidus}

\subsection{Énoncé}

\begin{theorem}
    Soient $D \subset \Cx$ un ouvert simplement connexe, $\gamma$ une courbe simple fermée régulière (par morceaux) contenue dans $D$ orientée positivement et $z_1, z_2, \ldots, z_m \in \inte \gamma$ tels que $z_i \neq z_j$ pour $i \neq j$.
    Si une fonction $f: D\setminus \{z_1, z_2, \ldots, z_m\} \rightarrow \Cx$ est holomorphe, alors:
    
    \[ \int_\gamma f(z) \dz = 2\pi i \sum_{k = 1}^m \Res_{z_k}(f) \]
\end{theorem}

\begin{note}
    Si une fonction est holomorphe sauf peut-être en un nombre fini de points $z_1, z_2,$ $ \ldots, z_m$, alors l'intégrale de $f$ le long de n'importe quelle courbe simple fermée régulière $\gamma$ contenue dans $D$ et orientée positivement est donnée par la somme (multipliée par $2\pi i$) des résidus de la fonction aux points $z_k$ (où $f$ n'est pas holomorphe) enfermés à l'intérieur de $\gamma$.
\end{note}

% 5.2.1??
\subsection{Exemples}

\begin{example}[1]
    Soit
    
    \[ f(z) = \frac{2}{z} + \frac{3}{z - 1} + \frac{1}{z^2} \]
    
    et $\gamma \subset \Cx$ une courbe régulière simple fermée orientée positivement.
    
    Discuter $\int_\gamma f(z) \dz$ en fonction de $\gamma$.
    
    \[ f(z) = \frac{p(z)}{q(z)} = \frac{2z(z-1) + 3z^2 + z -1 }{z^2 (z - 1)} \]
    
    $z_1 = 0$ est un pôle d'ordre 2 de $f$ ($p(0) \neq 0$, $q(0) = 0$, $q'(0) = 0$ et $q''(0) \neq 0$).
    
    $z_2 = 1$ est un pôle d'ordre 1 de $f$ ($p(1) \neq 0$, $q(1) = 0$ et $q'(1) \neq 0$).
    
    On a 
    
    \begin{align*}
    \Res_{z_1}(f) &= \Res_{0}(f) = \lim_{z \rightarrow 0} \frac{\dd}{\dz} \left[ z^2 f(z) \right] = \lim_{z \rightarrow 0} \frac{\dd}{\dz} \left[ 2z + \frac{3z^2}{z - 1} + 1 \right] \\&= \lim_{z \rightarrow 0} \left[ 2 + \frac{6z(z-1) - 3z^2}{(z - 1)^2} \right] = 2 + 0 = 2
    \end{align*}
    
    \begin{align*}
    \Res_{z_2}(f) &= \Res_{1}(f) = \lim_{z \rightarrow 1} \left[ (z - 1) f(z) \right] = \lim_{z \rightarrow 1} \left[ \frac{2(z-1)}{z} + \frac{3(z-1)}{z - 1} + \frac{z -1}{z^2} \right] \\&= 0 + 3 + 0 = 3
    \end{align*}
    
    \textbf{Distinction de cinq cas}
    
    \begin{description}
    \item[1\ier{} cas:] 0 et $1 \in \inte \gamma$
    
    \[ \int_\gamma f(z) \dz = 2\pi i \left[ \Res_{0}(f) + \Res_{1}(f) \right] = 2\pi i \left[ 2 + 3 \right] = 10 \pi i \]
    
    \item[2\ieme{} cas:] $0 \in \inte \gamma$ mais $1 \notin \adh{\gamma}$
    
    \[ \int_\gamma f(z) \dz = 2\pi i \ \Res_{0}(f) = 2\pi i \cdot 2 = 4 \pi i \]
    
    \item[3\ieme{} cas:] $0 \notin \adh{\gamma}$ mais $1 \in \inte \gamma$
    
    \[ \int_\gamma f(z) \dz = 2\pi i \ \Res_{1}(f) = 2\pi i \cdot 3 = 6 \pi i \]
    
    \item[4\ieme{} cas:] 0 et $1 \notin \adh{\gamma}$
    
    \[ \int_\gamma f(z) \dz = 0 \quad \textrm{(Théorème de Cauchy)} \]
    
    \item[5\ieme{} cas:] 0 ou $1 \in \gamma$
    
    L'intégrale $\int_\gamma f(z) \dz$ n'est pas définie.
    \end{description}

    \textit{Exemples: ex 2 et 3, série 9}
\end{example}

\begin{example}[2]
    Soit
    
    \[ f(z) = e^{\frac{1}{z^2}} \]
    
    et $\gamma$ une courbe simple fermée régulière orientée positivement.
    
    Discuter la valeur de $\int_\gamma f(z) \dz$ en fonction de $\gamma$.
    
    $f$ n'est pas holomorphe en $z_1 = 0$.
    Au voisinage de $z_1 = 0$, on a:
    
    \[ Lf(z) = \sum_{n = 0}^\infty \frac{1}{n!} \left(\frac{1}{z^2}\right)^n =
    1 + \sum_{n = 1}^\infty \frac{1}{n!z^{2n}} \]
    
    $\implies z_1 = 0$ est une singularité essentielle \textit{(cf. §4.2.2)}.
    
    On a que $\Res_{0}(f) := c_{-1} = 0$ (pas de terme $z^{-1}$ dans $Lf(z)$).
    
    \textbf{Distinction de trois cas}
    
    \begin{description}
        \item[1\ier{} cas:] $0 \in \inte \gamma$
        
        \[ \int_\gamma f(z) \dz = 2\pi i \ \Res_{0}(f) = 2\pi i \ 0 = 0 \]
        
        (mais $f$ n'est pas holomorphe en $z = 0$)
        
        \item[2\ieme{} cas:] $0 \notin \adh{\gamma}$
        
        \[ \int_\gamma f(z) \dz = 0  \quad \textrm{(Théorème de Cauchy)} \]
        
        $f$ holomorphe dans un domaine $D$ simplement connexe tel que $\adh{\gamma} \subset D \implies \gamma \subset \adh{\gamma} \subset D$
        
        \item[3\ieme{} cas:] $0 \in \gamma$
        
        L'intégrale $\int_\gamma f(z) \dz$ n'est pas définie.
    \end{description}
\end{example}

\subsection{Démonstration du Théorème des résidus}

\begin{proof}
    Soient $\gamma_k$ avec $k = 1, 2, \ldots, m$ $m$ courbes simples fermées régulières orientées positivement contenues dans $\inte \gamma$ et contenant $z_k$ dans leur intérieur.
    
    Comme $f: \adh{\gamma} \setminus \bigcup\limits_{k = 1}^m \inte \gamma_k \rightarrow \Cx$ est holomorphe, en appliquant le corollaire du Théorème de Cauchy \textit{(cf. §3.2.4)}, on a:
    
    \[ \int_\gamma f(z) \dz = \sum_{k = 1}^m \int_{\gamma_k} f(z) \dz := 2\pi i \sum_{k = 1}^m \Res_{z_k}(f) \]
    
    L'intégrale dans la somme est le coefficient $c_{-1}$ de la série de Laurent de $f$ au voisinage de $z_k$ multipliée par $2\pi i$.
    La deuxième égalité est donnée par la définition du résidu de $f$ en $z_k$ \textit{(§4.2.2)}.
\end{proof}

\begin{remark}
    Si $f$ est holomorphe dans $D$, alors pour toute courbe simple $\gamma$ fermée régulière dans $D$, il n'y a aucune singularité $z_k \in \inte \gamma$.
    Dans ce cas, $\sum\limits_{k = 1}^m \Res_{z_k}(f) = 0$ et le Théorème des résidus donne le résultat $\int_\gamma f(z) \dz = 0$ du Théorème de Cauchy \textit{(§3.2.1)}.
\end{remark}


\section{Application du Théorème des résidus au calcul d'intégrales réelles}

\subsection{Calcul d'intégrale de fonctions périodiques}

\begin{enumerate}[label=\alph*)]
    \item But: calculer des intégrales de la forme
    
    \[ \int_0^{2\pi} f(\cos \theta, \sin \theta) \dth \]
    
    avec $f: \R^2 \rightarrow \R$ définie par
    
    \[ (x,y) \longmapsto f(x,y) = \frac{p(x,y)}{q(x,y)} \]
    
    où $p$ et $q$ sont des fonctions polynômiales avec $q(\cos \theta, \sin \theta) \neq 0 \ \forall \theta \in [0, 2\pi]$
    
    \item Méthode:
    
    \begin{itemize}
    \item 
    On pose $z = e^{i\theta}$ et on a donc
    
    \[ \cos \theta = \frac{e^{i\theta} + e^{-i\theta}}{2}
    = \frac{1}{2} \left(z + \frac{1}{z}\right) \]
    
    \[ \sin \theta = \frac{e^{i\theta} - e^{-i\theta}}{2i}
    = \frac{1}{2i} \left(z - \frac{1}{z}\right) \]
    
    \item 
    On définit:
    
    \[ \tilde{f}: \Cx \longrightarrow \Cx \ ; \ z \longmapsto \tilde{f}(z) = \frac{1}{iz} f\left(\frac{1}{2} \left(z + \frac{1}{z}\right), \frac{1}{2i} \left(z - \frac{1}{z}\right)\right) \]
    
    On considère $\gamma$ le cercle \textit{unité} centré en $z = 0$ orienté positivement et $z_k$ pour $k = 1, 2, \ldots, m$ les singularités de $\tilde{f}$ à \textit{l'intérieur} de $\gamma$.
    
    $z_k \notin \gamma$ car $q(\cos \theta, \sin \theta) \neq 0$ pour $\theta \in [0, 2\pi] \implies$ pas de singularité de $\tilde{f}$ \textit{sur} $\gamma$
    
    \item 
    On applique le Théorème des résidus à la fonction $ \tilde{f}$ intégrée le long de $\gamma$:
    
    \[ \int_\gamma \tilde{f}(z) \dz = 2\pi i \sum_{k = 1}^m \Res_{z_k} (\tilde{f}) \]
    
    Mais on remarque que:
    
    \begin{align*}
    \int_\gamma \tilde{f}(z) \dz &= \int_\gamma \frac{1}{iz} f\left(\frac{1}{2} \left(z + \frac{1}{z}\right), \frac{1}{2i} \left(z - \frac{1}{z}\right)\right) \dz
    \\&= \int_0^{2\pi} \frac{1}{ie^{i\theta}} f(\cos \theta, \sin \theta) i e^{i\theta} \dth
    \\&= \int_0^{2\pi} f(\cos \theta, \sin \theta) \dth
    \end{align*}
    
    est exactement l'intégrale \textit{réelle} que l'on veut calculer.
    \end{itemize}

    Le résultat est:
    
    \[ \int_0^{2\pi} f(\cos \theta, \sin \theta) \dth = 2\pi i \sum_{k = 1}^m \Res_{z_k}(\tilde{f}) \]
    
    où $z_k$ pour $k = 1, 2, \ldots, m$ sont les singularités de $\tilde{f}$ à l'intérieur du cercle unité $\gamma$ centré en $z = 0$.
    
    \item Exemples
    
    \begin{example}[1]
        Calculer
        
        \[ \int_0^{2\pi} \frac{\dth}{\sqrt{5} - \sin \theta} \]
        
        On a $f(\cos \theta, \sin \theta) = \frac{1}{\sqrt{5} - \sin \theta}$, et $\sqrt{5} - \sin \theta \neq 0$ pour $\theta \in [0, 2\pi]$ et
        
        \[ \tilde{f} := \frac{1}{iz} f\left(\frac{1}{2} \left(z + \frac{1}{z}\right), \frac{1}{2i} \left(z - \frac{1}{z}\right)\right) = \frac{1}{iz} \frac{1}{\sqrt{5} - \frac{1}{2i} \left(z - \frac{1}{z}\right)} = \frac{2}{-z^2 + 2i\sqrt{5}z + 1} \]
        
        Les singularités de $\tilde{f}$ sont les zéros de $-z^2 + 2i\sqrt{5}z + 1$. $\Delta = (2i\sqrt{5})^2 + 4 = -16$
        
        \[ z_1 = \frac{-2i\sqrt{5} + 4i}{-2} = i(\sqrt{5} - 2) \]
        
        \[ z_2 = \frac{-2i\sqrt{5} - 4i}{-2} = i(\sqrt{5} + 2) \]
        
        On a que $-z^2 + 2i\sqrt{5}z + 1 = -(z - z_1)(z - z_2) = - \left[ z - i(\sqrt{5} - 2) \right] \left[ z - i(\sqrt{5} + 2) \right]$ et
        
        \[ \tilde{f}(z) = \frac{-2}{\left[ z - i(\sqrt{5} - 2) \right] \left[ z - i(\sqrt{5} + 2) \right]} \]
        
        Soit $\gamma$ le cercle unité centré en $z = 0$ et orienté positivement.
        
        \[ 0 < \Imp z_1 = \sqrt{5} - 2 < 1 \implies z_1 \in \inte \gamma  \]
        
        \[ \Imp z_2 = \sqrt{5} + 2 > 1 \implies z_2 \notin \inte \gamma \]
        
        On a:
        
        \[ \int_0^{2\pi} \frac{\dth}{\sqrt{5} - \sin \theta} = 2\pi i \ \Res_{z_1} (\tilde{f}) \]
        
        $z_1 = i(\sqrt{5} - 2)$ est un pôle d'ordre 1 de $\tilde{f}$
        
        \begin{align*}
        \Res_{z_1} (\tilde{f})& = \lim_{z \rightarrow z_1} (z - z_1) \tilde{f}(z) \\&= \lim_{z \rightarrow i(\sqrt{5} - 2)} (z - i(\sqrt{5} - 2)) \frac{-2}{\left[ z - i(\sqrt{5} - 2) \right] \left[ z - i(\sqrt{5} + 2) \right]}
        \\&= \frac{-2}{-4i} = \frac{1}{2i}
        \end{align*}
        
        Le résultat est
        
        \[ \int_0^{2\pi} \frac{\dth}{\sqrt{5} - \sin \theta} = 2 \pi i \frac{1}{2i} = \pi \]
    \end{example}

    \begin{example}[2]
        Calculer:
        
        \[ \int_0^{2\pi} \frac{\dth}{2 + \cos \theta} \]
        
        On a $f(\sin \theta, \cos \theta) = \frac{1}{2 + \cos \theta} \quad 2 + \cos \theta \neq 0 \ \forall \theta \in [0, 2\pi]$
        
        \[ \tilde{f} := \frac{1}{iz} f\left(\frac{1}{2} \left(z + \frac{1}{z}\right), \frac{1}{2i} \left(z - \frac{1}{z}\right)\right) = \frac{1}{iz} \frac{1}{2 + \frac{1}{2} \left(z + \frac{1}{z}\right)} = \frac{2}{i \left(z^2 + 4z + 1\right)} \]
        
        Les singularités de $\tilde{f}$ sont les zéros de $z^2 + 4z + 1$. $\Delta = 12$
        
        \[ z_1 = \frac{-4 + 2\sqrt{3}}{2} = \sqrt{3} - 2 \]
        \[ z_2 = \frac{-4 - 2\sqrt{3}}{2} = -(\sqrt{3} + 2) \]
        
        On a $z^2 +  4z + 1 = (z - z_1)(z - z_2) = (z + 2 - \sqrt{3})(z + 2 + \sqrt{3})$ et
        
        \[ \tilde{f} = \frac{2}{i(z + 2 - \sqrt{3})(z + 2 + \sqrt{3})} \]
        
        Soit $\gamma$ le cercle unité centré en $z = 0$ orienté positivement.
        
        \[ -1 < z_1 = \sqrt{3} - 2 < 0 \implies  z_1 \in \inte \gamma \]
        
        \[ z_2 = -(\sqrt{3} + 2) < -1 \implies  z_2 \notin \inte \gamma \]
        
        Donc
        
        \[ \int_0^{2\pi} \frac{\dth}{2 + \cos \theta} = 2\pi i \Res_{z_1} (\tilde{f}) \]
        
        $z_1 = \sqrt{3} - 2$ est un pôle d'ordre 1 de $\tilde{f}$, donc:
        
        \begin{align*}
        \Res_{z_1} (\tilde{f}) &= \lim_{z \rightarrow z_1} (z - z_1) \tilde{f}(z)
        \\&= \lim_{z \rightarrow \sqrt{3} - 2} (z + 2 - \sqrt{3}) \frac{2}{i(z + 2 - \sqrt{3})(z + 2 + \sqrt{3})}
        \\&= \frac{1}{i\sqrt{3}}
        \end{align*}
        
        Le résultat est:
        
        \[ \int_0^{2\pi} \frac{\dth}{2 + \cos \theta} = 2\pi i \frac{1}{i\sqrt{3}} = \frac{2\pi}{\sqrt{3}} \]
        
        \textit{Autres exemples: ex 1 à 4, série 10 et ex 1, série 11}
    \end{example}
\end{enumerate}

\subsection{Calcul d'intégrales généralisées}

\begin{enumerate}
\item But: calculer des intégrales de la forme

\[ \int_{-\infty}^\infty f(x) e^{i\alpha x} \dx \quad \alpha \in \R_+ \ (\alpha \geq 0),\ f: \R \rightarrow \R \]

Où $f(x) = \frac{p(x)}{q(x)}$, et $p, q$ sont des fonctions polynômiales telles que $q(x) \neq 0 \ \forall x \in \R$ et degré$(q) - $ degré$(p) \geq 2$.

\begin{remark}\hfill{}

    Les conditions sur $p$ et $q$ impliquent que l'intégrale généralisée $\int_{-\infty}^\infty |f(x)| \dx$ existe.
\end{remark}

\item Méthode: on choisit un nombre réel $r > 0$ et on considère la courbe $\gamma_r := L_r \cup C_r$ orientée positivement, où:

\begin{itemize}
    \item $L_r$ est le segment de droite $[-r,r]$ situé sur \textit{l'axe réel},
    \item $C_r$ est le demi-cercle de rayon $r$ centré en $z = 0$ et situé dans le \textit{demi-plan supérieur}.
\end{itemize}

$\gamma_r = L_r \cup C_r$ est une courbe simple fermée par morceaux.

On définit la fonction $g: \Cx \rightarrow \Cx$ par $z \mapsto g(z) = f(z) e^{i\alpha z} = \frac{p(z)}{q(z)} e^{i\alpha z}$.

\begin{constatation}
    Les seules singularités de $g$ sont les zéros de $q$.
    Par hypothèse, $q$ est une fonction polynômiale et $q(x) \neq 0 \ \forall x \in R$, alors $q$ possède un nombre \textit{fini} de zéros et aucun ne se trouve situé sur l'axe réel.
\end{constatation}

\begin{idea}
    On choisit $r > 0$ suffisamment grand pour que \textbf{tous} les zéros de $q$ situés dans le \textit{demi-plan supérieur} soient à l'intérieur de $\gamma_r$.
\end{idea}

En appliquant le Théorème des résidus à $g(z) = f(z) e^{i\alpha z}$ intégrée le long de $\gamma_r$, on a:

\[ \int_{\gamma_r} f(z) e^{i\alpha z} \dz = 2\pi i \sum_{k = 1}^m \Res_{z_k}(g) \]

où $z_k$ pour $k = 1, 2, \ldots, m$ sont les singularités de $f$ (i.e. les zéros de $q$) situées dans le demi-plan supérieur.

D'autre part, puisque $\gamma_r = L_r \cup C_r$, on a:

\[ \int_{\gamma_r} f(z) e^{i\alpha z} \dz = \int_{L_r} f(z) e^{i\alpha z} \dz + \int_{C_r} f(z) e^{i\alpha z} \dz \]

\[ \lim_{r \rightarrow \infty} \int_{\gamma_r} f(z) e^{i\alpha z} \dz
= \lim_{r \rightarrow \infty} \int_{L_r} f(z) e^{i\alpha z} \dz
+ \lim_{r \rightarrow \infty} \int_{C_r} f(z) e^{i\alpha z} \dz \]

\textbf{Étude de chaque limite}

\begin{enumerate}
    \item
    
    \[
        \lim_{r \rightarrow \infty} \int_{\gamma_r} f(z) e^{i\alpha z} \dz
        = \lim_{r \rightarrow \infty} \left[ 2\pi i \sum_{k = 1}^m \Res_{z_k}(g) \right]
        = 2\pi i \sum_{k = 1}^m \Res_{z_k}(g)
    \]
    
    \item 
    
    \[
        \lim_{r \rightarrow \infty} \int_{L_r} f(z) e^{i\alpha z} \dz
        = \lim_{r \rightarrow \infty} \int_{-r}^r f(x) e^{i\alpha x} \dx
        = \int_{-\infty}^\infty f(x) e^{i\alpha x} \dx
    \]
    
    où $z = x \in [-r, r] \subset \R$.
    Ceci est l'intégrale généralisée que l'on veut calculer!
    
    \item On montre que si $\deg(q) - \deg(p) \geq 2$, alors:
    
    \[
        \lim_{r \rightarrow \infty} \int_{C_r} f(z) e^{i\alpha z} \dz = 0
    \]
\end{enumerate}

\begin{result}[final]
    On obtient la formule suivante:
    
    \[ \int_{-\infty}^\infty f(x) e^{i\alpha x} \dx
    = 2\pi i \sum_{k = 1}^m \Res_{z_k}(g) \]
    
    où $g(z) = f(z)e^{i\alpha z}$ et $z_k$ pour $k = 1, 2, \ldots, m$ sont les singularités de $f$ situées dans le demi-plan supérieur (i.e. les zéros de $q$ tels que $\Imp z_k > 0$).
\end{result}

\item

\begin{example}[1]
Calculer

\[ \int_{-\infty}^\infty \frac{x^2}{x^4 + 16} \dx \]

où $\alpha = 0$ et $f(x) = \frac{x^2}{x^4 + 16}$.
On a $p(x) = x^2$ et $q(x) = x^4 + 16$.
Conditions vérifiées: $q(x) \neq 0 \ \forall x \in \R$ et $\deg(q) - \deg(p) = 2$.

Calcul avec la méthode des résidus avec $g(z) = f(z) = \frac{z^2}{z^4 + 16}$.

Singularités de $f(z) \iff$ recherche des zéros de $q(z)$

$q(z) = 0 \iff z^4 + 16 = 0 \iff z^4 = -16 \iff 16 e^{i\pi} \iff z = 2 e^{i\left(\frac{\pi}{4} + \frac{2n\pi}{4}\right)}$ avec $n = 0,1,2,3$.

Les singularités sont:

\begin{description}
    \item[pour n = 0]
        \[ z_1 = 2 e^{i\frac{\pi}{4}}
        = 2 \left( \cos \frac{\pi}{4} + i \sin \frac{\pi}{4}\right)
        = 2 \left( \frac{\sqrt{2}}{2} + i \frac{\sqrt{2}}{2} \right) = \sqrt{2} (1 + i)\]

    \item[pour n = 1]
        \[ z_2 = 2 e^{i\frac{3\pi}{4}} = \ldots = \sqrt{2} (-1 + i) \]

    \item[pour n = 2]
        \[ z_3 = 2 e^{i\frac{5\pi}{4}} = \ldots = - \sqrt{2} (1 + i) \]

    \item[pour n = 3]
        \[ z_4 = 2 e^{i\frac{7\pi}{4}} = \ldots = - \sqrt{2} (-1 + i) \]
\end{description}

Ce sont des pôles d'ordre 1, car $z^4 + 16 = (z - z_1) (z - z_2) (z - z_3) (z - z_4)$.

Les seuls pôles qui contribuent sont $z_1$ et $z_2$ situés dans le demi-plan \textit{supérieur}.

Calcul des résidus de $f$ en $z_1$ et $z_2$:

\[
    \Res_{z_1}(f) = \frac{p(z_1)}{q'(z_1)}
    = \frac{z_1^2}{4 z_1^3} = \frac{1}{4 z_1} = \frac{1}{4\sqrt{2}(1+i)} = \frac{1 - i}{8\sqrt{2}}
\]

\[
    \Res_{z_2}(f) = \frac{p(z_2)}{q'(z_2)}
    = \frac{z_2^2}{4 z_2^3} = \frac{1}{4 z_2} = \frac{1}{4\sqrt{2}(-1+i)} = - \frac{1 + i}{8\sqrt{2}}
\]

L'intégrale vaut:

\[
    \int_{-\infty}^\infty \frac{x^2}{x^4 + 16} \dx
    = 2 \pi i \left[ \Res_{z_1}(f) + \Res_{z_2}(f) \right]
    = 2 \pi i \left[ \frac{1 - i}{8\sqrt{2}} - \frac{1 + i}{8\sqrt{2}} \right] = \frac{\sqrt{2}\pi}{4}
\]

\end{example}


\begin{example}[2]
    Calculer
    
    \[ \int_{-\infty}^\infty \frac{\cos 5x}{x^2 + 1} \dx \]
    
    On utilise la formule d'Euler
    
    \[ e^{i5x} = \cos 5x + i \sin 5x \]
    
    et on peut écrire
    
    \[
        \int_{-\infty}^\infty \frac{\cos 5x}{x^2 + 1} \dx
        = \Rep \left[ \int_{-\infty}^\infty \frac{e^{i5x}}{x^2 + 1} \dx \right]
    \]
    
    On considère $\int_{-\infty}^\infty f(x) e^{i\alpha x} \dx$ où $\alpha = 5$ et $f(x) = \frac{1}{x^2 + 1}$.
    
    On a $p(x) = 1$ et $q(x) = x^2 + 1$.
    Conditions vérifiées: $q(x) \neq 0 \ \forall x \in \R$ et $\deg(p) - \deg(q) = 2$.
    
    Méthode des résidus avec $g(z) = f(z) e^{i5z} = \frac{e^{i5z}}{z^2 + 1}$.
    
    Singularités de $f \implies$ zéros de $q \implies q(z) = 0 \implies z^2 + 1 = 0 \implies z^2 = -1 \implies z_1 = i$ et $z_2 = -i$.
    Ce sont des pôles d'ordre 1 et $z^2 + 1 = (z - i)(z + i)$.
    
    Le seul pôle qui contribue est $z_1 = i$ situé dans le demi-plan supérieur.
    
    \[
        \Res_{z_1}(g)
        = \lim_{z \rightarrow z_1} (z - z_1) g(z)
        = \lim_{z \rightarrow i} (z - i) \frac{e^{i5z}}{(z - i)(z + i)}
        = \frac{e^{i5i}}{2i}
        = \frac{e^{-5}}{2i}
    \]
    
    Alors:
    
    \[
        \int_{-\infty}^\infty \frac{e^{i5x}}{x^2 + 1} \dx = 2\pi i \Res_i (g) = 2\pi i \frac{e^{-5}}{2i} = \frac{\pi}{e^{5}}
    \]
    
    et
    
    \[
        \int_{-\infty}^\infty \frac{\cos 5x}{x^2 + 1} \dx
        = \Rep \left[ \int_{-\infty}^\infty \frac{e^{i5x}}{x^2 + 1} \dx \right]
        = \Rep \left[ \frac{\pi}{e^{5}} \right]
        = \frac{\pi}{e^{5}}
    \]
    
    \textit{Autres exemples: ex. 2 et 3, série 11}
\end{example}

\end{enumerate}

\begin{proof}
    Preuve de
    
    \[ \lim_{r \rightarrow \infty} \int_{C_r} f(z) e^{i\alpha z} \dz = 0 \]
    
    Sur $C_r$, on a $z = r e^{i\theta}$ avec $\theta \in [0,\pi]$ et $\dz = i r e^{i\theta} \dth$
    
    \begin{align*}
        \implies \int_{C_r} f(z) e^{i\alpha z} \dz
        &= \int_0^\pi f(r e^{i\theta}) e^{i\alpha r e^{i\theta}} i r e^{i\theta} \dth
        \\&= ir \int_0^\pi f(r e^{i\theta}) e^{i\alpha r (\cos \theta + i \sin \theta)} e^{i\theta} \dth
        \\&= ir \int_0^\pi f(r e^{i\theta}) e^{-\alpha r\sin \theta}e^{i\alpha r \cos \theta} e^{i\theta} \dth
        \\&= ir \int_0^\pi f(r e^{i\theta}) e^{-\alpha r\sin \theta}e^{i \left(\alpha r \cos \theta + \theta\right)} \dth
    \end{align*}
    
    Alors
    
    \[
        \left| \int_{C_r} f(z) e^{i\alpha z} \dz \right|
        \leq r \int_0^\pi \left| f(r e^{i\theta}) \right| e^{-\alpha r\sin \theta} \left| e^{i \left(\alpha r \cos \theta + \theta\right)} \right| \dth
    \]
    
    \begin{enumerate}
        \item 
        Puisque $f(z) = \frac{p(z)}{q(z)}$ avec $\deg(q) - \deg(p) \geq 2$, alors on a que $\left|f(z)\right| \leq \frac{C}{|z|^2}$ pour tout $z \in \Cx$ avec $|z|$ suffisamment grand, $C \in \R^\ast_+$ est une constante.
        
        \item 
        Comme $\alpha \geq 0$, $r > 0$ et $\sin \theta \geq 0$ pour $\theta \in [0, \pi]$, on a que $-\alpha r \sin \theta \leq 0$.
        
        \[ 0 \leq e^{-\alpha r \sin \theta} \leq 1 \]
        
        \item 
        De plus, $\left| e^{i(\theta + \alpha r \cos \theta)} \right| = 1$ car $\left|e^{ix}\right| = 1$ pour tout $x \in \R$.
        Donc on obtient:
        
        \[
            \left| \int_{C_r} f(z) e^{i\alpha z} \dz \right|
            \leq r \int_0^\pi \frac{C}{r^2} \dth = \frac{C}{r} \int_0^\pi \dth = \frac{C\pi}{r}
        \]
        
        Puisque 
        
        \[
            \lim_{r \rightarrow \infty} \frac{C\pi}{r} = 0 \implies \lim_{r \rightarrow \infty} \int_{C_r} f(z) e^{i\alpha z} \dz = 0
        \]
    \end{enumerate}
\end{proof}

\begin{remark}[1]
    Pour $z = r e^{i\theta} = r(cos \theta + i \sin \theta)$, on a $\module{e^{i\alpha z}} = e^{-\alpha r \sin \theta}$.
    Lorsque $\alpha \leq 0$, on a $-\alpha r \sin \theta \leq 0$ et $\module{e^{i\alpha z}} < 1$ si $\theta \in [\pi, 2\pi]$ et il faut donc choisir le demi-cercle $C_r$ dans le demi-plan pour avoir
    
    \[ \lim_{r \rightarrow \infty} \int_{C_r} f(z) e^{i\alpha z} \dz = 0 \]
    
    Résultat: pour calculer des intégrales généralisées $\int_{-\infty}^\infty f(x) e^{i\alpha x} \dx$ avec $\alpha \in \R_-$, on applique la même méthode en considérant les singularités de $f$ situées dans le \textit{demi-plan inférieur}.
    
    \textbf{Attention} à l'orientation positive de $\gamma_r = L_r \cup C_r$.
    On obtient l'intégrale suivante:
    
    \[ \int_{\infty}^{-\infty} f(x) e^{i\alpha x} \dx \]
\end{remark}

\begin{remark}[2]
    La méthode permet de calculer la transformée de Fourier d'une fonction $f$ du type quotient de polynômes vérifiant les conditions demandées.
    
    Pour $\alpha \geq 0$
    
    \[ \hat{f}(-\alpha) := \ostpi \int_{-\infty}^\infty f(x)e^{i\alpha x} \dx = 2\pi i \sum_{k = 1}^m \Res_{z_k}(g) \]
    
    où $g(z) = f(z) e^{i\alpha z}$ et $z_k = 1, \ldots, m$ sont les singularités de $f$ situées dans le demi-plan supérieur.
    
    Pour $\alpha \leq 0$, on considère les singularités de $f$ dans le demi-plan inférieur et on obtient $-\hat{f}(-\alpha)$.
\end{remark}

% !TeX spellcheck = fr_FR
\chapter{Transformée de Laplace}


\section{Introduction}

\begin{motivation}
    Généralisation de la théorie de Fourier pour appliquer une étude de problèmes transitoires en électricité avec des conditions initiales.
\end{motivation}

\subsubsection*{Procédé heuristique induisant la définition de la transformée de Laplace:}

Soit $g: \R \rightarrow \R$, la transformée de Fourier $\TF (g): \R \rightarrow \Cx$ est définie par $\TF(g)(\alpha) = \ostpi \int_{-\infty}^{\infty} g(t) e^{-i\alpha t} \dt$.

On considère $g(t) = \left\{
\begin{array}{lcl}
f(t) & \textrm{ si } & t \geq 0\\
0    & \textrm{ si } & t < 0\\
\end{array}
\right.$ et on autorise la variable $\alpha \in \R$ à prendre des valeurs complexes.
En posant $\alpha = -i z$ avec $z \in \Cx$, on obtient:

\[ \sqrt{2\pi} \ \TF(g)(-iz) = \int_{-\infty}^{\infty} g(t) e^{-i(-iz)t} \dt = \int_{0}^{\infty} f(t) e^{-zt} \dt \]

Ceci est une nouvelle fonction qui dépend de la variable $z \in \Cx$ qui est appelée la transformée de Laplace de $f$.

\section{Transformée de Laplace d'une fonction}

\subsection{Définition}

\begin{definition}
    Soit $f: \R_+ \rightarrow \R$ une fonction continue par morceaux et soit $\gamma_0 \in \R$ tels que $\int_0^\infty \module{f(t)} e^{-\gamma_0 t} \dt < \infty$.
    
    La \textbf{transformée de Laplace} de $f$ est la fonction notée $\TL(f)$ où $F: \Cx \rightarrow \Cx$ définie par:
    
    \[ z \longmapsto \TL(f)(z) = F(z) = \int_0^\infty f(t) e^{-zt} \dt \quad \forall z \in \Cx : \Rep z \geq \gamma_0 \]
    
    $\gamma_0$ s'appelle l'abscisse de convergence de $f$.
\end{definition}

\begin{remark}
    Si $\Rep z \geq \gamma_0$, alors $\TL(f)(z)$ est bien définie.
    En effet, comme:
    
    \[ \module{e^{-zt}} = \module{e^{-(\Rep z + i \Imp z) t}} = e^{-t \Rep z} \module{e^{-i t \Imp z}} = e^{-t\Rep z} \leq e^{-t\gamma_0} \]
    
    pour $t \geq 0$ si $\Rep z \geq \gamma_0$.
    Alors:
    
    \[ \module{\TL(f)(z)} = \module{\int_0^\infty f(t) e^{-zt} \dt} \leq \int_0^\infty \module{f(t)} \module{e^{-zt}} \dt \leq \int_0^\infty \module{f(t)} e^{-\gamma_0 t} \dt < \infty \]
\end{remark}

\subsection{Exemples}

\begin{example}[1]
    Calculer la transformée de Laplace de la fonction $f: \R_+ \rightarrow \R$, où $f(t) = 1$.

Puisque

\begin{align*}
    \int_0^\infty \module{f(t)} e^{-\gamma_0 t} \dt
    &= \int_0^\infty e^{-\gamma_0 t} \dt
    = \left.-\frac{e^{-\gamma_0 t}}{\gamma_0}\right|_0^\infty
    \\&= \frac{1}{\gamma_0} \left[1 - \lim_{t \rightarrow \infty} e^{-\gamma_0 t}\right]
    = \left\{
    \begin{array}{clc}
    \frac{1}{\gamma_0} & \textrm{ si } \gamma_0 > 0\\
    +\infty & \textrm{ si } \gamma_0 < 0
    \end{array}\right.
\end{align*}

Alors n'importe quel $\gamma_0 \in \R_+^\ast$ est abscisse de convergence de $f$.

\[
    \TL(f)(z)
    = \int_0^\infty \module{f(t)} \module{e^{-zt}} \dt
    = \int_0^\infty \module{e^{-zt}} \dt
    = \left.-\frac{e^{-z t}}{z}\right|_0^\infty
    = \frac{1}{z} \left[1 - \lim_{t \rightarrow \infty} e^{-z t}\right]
\]

Or,

\[
    \lim_{t \rightarrow \infty} \module{e^{-zt}}
    = \lim_{t \rightarrow \infty} \module{e^{-(x + iy)t}}
    = \lim_{t \rightarrow \infty} e^{-xt} \module{e^{-iyt}}
    = 0
\]

si $x = \Rep z > 0$.

Résultat: $\TL(f): \Cx \rightarrow \Cx$, où:

\[
    z \longmapsto \TL(f)(z) = \frac{1}{z} \textrm{ si } \Rep z > 0
\]
\end{example}

\begin{example}[2]
Calculer la transformée de Laplace de la fonction $f: \R_+ \rightarrow \R$, où $f(t) = e^{at}$ avec $a \in \R$:

Comme 

\begin{align*}
    \int_0^\infty \module{f(t)} e^{-\gamma_0 t} \dt 
    &= \int_0^\infty e^{(a - \gamma_0)t}
    = \left.
    \frac{e^{(a - \gamma_0)t}}{a - \gamma_0}
    \right|_0^\infty
    \\& = \frac{1}{a - \gamma_0} \left[ \lim_{t \rightarrow \infty} e^{(a - \gamma_0)t} - 1 \right]
    = \left\{
    \begin{array}{clc}
    +\infty & \textrm{ si } a \geq \gamma_0\\
    \frac{1}{\gamma_0 - a} & \textrm{ si } a < \gamma_0
    \end{array}\right.
\end{align*}

Alors, n'importe quel $\gamma_0 > a$ est abscisse de convergence de $f$.

\[
    \TL(f)(z)
    = \int_0^\infty f(t) e^{-z t} \dt 
    = \int_0^\infty e^{(a - z)t}
    = \left.
    \frac{e^{(a - z)t}}{a - z}
    \right|_0^\infty
    = \frac{1}{a - z} \left[ \lim_{t \rightarrow \infty} e^{(a - z)t} - 1 \right]
\]

Or,

\[
    \lim_{t \rightarrow \infty} \module{e^{(a - z)t}}
    = \lim_{t \rightarrow \infty} \module{e^{(a - x - iy)t}}
    = \lim_{t \rightarrow \infty} e^{(a - x)t} \module{e^{- iyt}}
    = 0
\]

si $a - x < 0$.

Résultat: $\TL(f): \Cx \rightarrow \Cx$, où:

\[
z \longmapsto \TL(f)(z) = \frac{1}{z - a} \textrm{ si } \Rep z > a
\]

\textit{Autres exemples: ex.1, série 12}

\end{example}


\section{Propriétés de la transformée de Laplace}

On considère deux fonctions, $f,g: \R_+ \rightarrow \R$ continues par morceaux et $\gamma_0 \in \R$ tels que $\int_{0}^{\infty} \module{f(t)} e^{-\gamma_0 t} \dt < \infty$ et $\int_{0}^{\infty} \module{g(t)} e^{-\gamma_0 t} \dt < \infty$.

On note $\TL(f) = F$ et $\TL(g) = G$ les transformées de Laplace de $f$ et de $g$.

\subsection{Linéarité et décalage}

\begin{itemize}
    \item 
    $\TL(af + bg) = a \TL(f) + b \TL(g)$ avec $a, b \in \R$
    \item 
    Si $a \in \R_+^\ast$ et $b \in \R$ et $h(t) = e^{-bt} f(at)$ alors
    
    \[ \TL(h)(z) = \frac{1}{a} \TL(f)\left(\frac{z + b}{a}\right) \]
    
    pour tout $z \in \Cx$ tel que $\Rep\left(\frac{z + b}{a}\right) \geq \gamma_0$
\end{itemize}

\subsection{Transformée de Laplace du produit de convolution}

Si

    \[
        (f \ast g)(t)
        = \int_{-\infty}^\infty f(t - s)g(s) \ds
        = \int_0^t f(t-s)g(s) \ds
    \]

est le produit de convolution de $f$ et de $g$, alors:

\[
    \TL(f \ast g)(z) = \TL(f)(z)\TL(g)(z) \quad \forall z \in \Cx : \Rep z \geq \gamma_0
\]

\subsection{Holomorphie et dérivation de la transformée de Laplace}

$\TL(f)$ est holomorphe dans le domaine $D = \{ z \in \Cx | \Rep z > \gamma_0 \}$.
De plus, $\forall z \in D$, on a:

\[ \TL(f)'(z) = - \int_0^\infty t f(t) e^{-zt} \dt = - \TL(h)(z) \]

où $h: \R_+ \rightarrow \R$ est définie par $h(t) = tf(t)$.

\subsection{Transformée de Laplace de la dérivée d'une fonction}

Si de plus $f \in C^1(\R_+)$ et $\int_0^\infty \module{f'(t)} e^{-\gamma_0 t} \dt < \infty$, alors $\forall z \in D$, on a:

\[
    \TL(f')(z) = z \TL(f)(z) - f(0)
\]

Plus généralement: si $f \in C^n (R_+)$ et $\int_0^\infty \module{f^{(k)}(t)} e^{-\gamma_0 t} \dt < \infty$ pour $k = 1, 2, \ldots, n$, alors:

\[
    \TL(f^{(n)})(z) = z^n \TL(f)(z) - z^{n-1} f(0) - z^{n-2} f'(0)
    - \ldots - z f^{(n-2)}(0) - f^{(n-1)}(0) \quad \forall z \in D
\]

\subsection{Transformée de Laplace d'une primitive d'une fonction}

Si de plus $f \in C(\R_+)$ avec $\gamma_0 \geq 0$ et si $\varphi(t) = \int_0^t f(s) \ds$ est une primitive de $f$, alors $\forall z \in D$, on a:

\[
    \TL(\varphi)(z) = \frac{1}{z} \TL(f)(z)
\]

\subsection{Esquisse des démonstrations des propriétés}

\textit{Cf. ex.3, série 12}

\subsection{Exemples d'utilisation des propriétés}

\begin{example}
    Calculer la transformée de Laplace de la fonction $f: \R_+ \rightarrow \R$ définie par $f(t) = t^2$.
    
    Méthode: utiliser la propriété 6.3.4 de la deuxième dérivée de $f$:
    
    On a $\TL(f'')(z) = z^2 \TL(f)(z) - zf(0) - f'(0)$.
    Comme $f(t) = t^2, \enspace f'(t) = 2t$ et $f''(t) = 2 \implies f(0) = 0$ et $f'(0) = 0$, et $f''(t) = 2g(t)$ avec $g(t) = 1$ si $t \geq 0$.
    
    \[ \implies \TL(f'')(z) = 2 \TL(g)(z) = 2 \frac{1}{z} \]
    
    Donc on obtient $\frac{2}{z} = z^2 \TL(f)(z) \implies \TL(f)(z) = \frac{2}{z^3}$
    
    Résultat, pour $\TL(f): \Cx \rightarrow \Cx$:
    
    \[ z \longmapsto \TL(f)(z) = \frac{2}{z^3} \]
    
    \textit{Autres exemples: ex. 2, série 12}
\end{example}


\section{La formule d'inversion de la transformée de Laplace}

\subsection{Théorème de la transformée de Laplace inverse}

\begin{theorem}
    Soit $f: \R_+ \rightarrow \R$ une fonction (étendue à $\R$ en posant $f(t) = 0$ pour $t < 0$) continue pour $t > 0$ et soit $\gamma_0 \in \R$ tel que $\int_0^\infty \module{f(t)}e^{-\gamma_0 t}\dt < \infty$.
    
    Si la transformée de Laplace $F = \TL(f)$ de $f$ est telle que $\int_{-\infty}^\infty F(\gamma + is) \ds < \infty$ pour un certain $\gamma > \gamma_0$, alors on a la formule d'inversion de $\TL$:
    
    \[ \TLi(F)(t) := \frac{1}{2\pi} \int_{-\infty}^\infty F(\gamma + is) e^{(\gamma + is) t} \ds = f(t) \quad \forall t > 0 \]
    
    $\TLi$ s'appelle la transformée de Laplace inverse.
    L'intégrale de la définition est indépendante de $\gamma$.
\end{theorem}

\subsection{Exemples d'utilisation}

\begin{example}[1]
    Trouver une fonction $f: \R_+ \rightarrow \R$ telle que sa transformée de Laplace soit:
    
    \[ F(z) = \frac{1}{(z-a)(z-b)} \]
    
    où $a, b \in \R, a \neq b$.
    Autrement dit: trouver $\TLi(F)$.
    
    On décompose $F$ en élément simples.
    
    \begin{align*}
        F(z) &= \frac{\alpha}{z - a} + \frac{\beta}{z - b}
        = \frac{\alpha (z - b) + \beta (z-a)}{(z-a)(z-b)}
        = \frac{(\alpha + \beta) z - (\alpha b + \beta a)}{(z-a)(z-b)}
        \\&\implies \beta = \frac{1}{b - a}, \alpha = \frac{1}{a - b}
    \end{align*}
    
    On a donc:
    
    \[ F(z) = \frac{1}{a - b} \left[\frac{1}{z - a} - \frac{1}{z - b}\right] = \frac{1}{a - b} \left[G(z) - H(z)\right] \]
    
    avec $G(z) = \frac{1}{z - a}$ et $H(z) = \frac{1}{z - b}$.
    Alors:
    
    \begin{align*}
        f(t) &= \TLi(F)(t)
        = \TLi\left(\frac{1}{a - b} \left[G(z) - H(z)\right]\right)(t)
        \\&= \frac{1}{a - b} \left[ \TLi(G)(t) - \TLi(H)(t) \right]
        = \frac{1}{a - b} \left[ e^{at} - e^{bt} \right]
    \end{align*}
    
    \textit{Voir exemple 2 §6.2.2}
\end{example}


\begin{example}[2]
    Trouver une fonction $f: \R_+ \rightarrow \R$ telle que sa transformée de Laplace soit:
    
    \[ F(z) = \frac{z^2}{(z^2+1)^2} \]
    
    Autrement dit: trouver $\TLi(F)$.
    
    On remarque:
    
    \begin{align*}
    F(z) &= \frac{z^2}{(z^2+1)^2}
    = \frac{z^2 - 1 + z^2 + 1}{2(z^2+1)^2}
    = \frac{z^2 - 1}{2(z^2+1)^2} + \frac{z^2 + 1}{2(z^2+1)^2}
    \\&= \frac{z^2 - 1}{2(z^2+1)^2} + \frac{1}{2(z^2+1)}
    = \frac{1}{2} G(z) + \frac{1}{2} H(z)
    \end{align*}
    
    avec $G(z) = \frac{z^2 - 1}{(z^2+1)^2}$ et $H(z) = \frac{1}{z^2+1}$.
    
    Alors:
    
    \begin{align*}
    f(t) &= \TLi(F)(t)
    = \TLi\left(\frac{1}{2} \left[G(z) + H(z)\right]\right)(t)
    \\&= \frac{1}{2} \left[ \TLi(G)(t) + \TLi(H)(t) \right]
    = \frac{1}{2} \left[ t \cos t +  \sin t \right]
    \end{align*}
\end{example}


\begin{example}[3]
    Trouver une fonction $f: \R_+ \rightarrow \R$ telle que sa transformée de Laplace soit:
    
    \[ F(z) = \frac{z^2}{(z+1)(z-2)^2} \]
    
    Autrement dit: trouver $\TLi(F)$.
    
    \textbf{Méthode:} utiliser le théorème des résidus pour la fonction $h$ définie par:
    
    \[ h(z) = F(z) e^{zt} \]
    
    avec $t > 0$ fixé.
    
    \begin{enumerate}
        \item 
        On cherche les singularités de $h(z) = \frac{z^2 e^{zt}}{(z+1)(z-2)^2}$.
        $z = -1$ est un pôle d'ordre 1 et $z = 2$ est un pôle d'ordre 2 de $h$.
        
        \item 
        On choisit $\gamma \in \R$ tel que toutes les singularités de $h$ se trouvent à gauche de la droite $\Rep z = \gamma$.
        
        \item 
        On choisit $r > 0$ assez grand pour que le cercle $C_r$ de rayon $r$ et centré à l'origine intercepte la droite $\Rep z = \gamma$ et que toutes les singularités de $h$ soient à l'\textit{intérieur} de la courbe fermée $\gamma_r = C'_r \cup L_r$ où
        
        \[ C'_r = \left\{ z \in \Cx : |z| = r \ \textrm{ et } \Rep z < \gamma \right\} \]
        
        \[ L_r = \left\{ z \in \Cx : \Rep z = \gamma \ \textrm{ et } \module{\Imp z} \leq \sqrt{r^2 - \gamma^2} \right\} \]
        
        \item 
        On applique le théorème des résidus à $h$ et $\gamma_r$.
        
        \[ \int_{\gamma_r} \frac{z^2 e^{zt}}{(z+1)(z-2)^2} \dz = 2 \pi i \left[ \Res_{-1}(h) + \Res_2(h) \right] \]
        
        On a:
        
        \[
            \Res_{-1}(h)
            = \lim_{z \rightarrow -1} (z + 1) h(z)
            = \lim_{z \rightarrow -1} \frac{z^2 e^{zt}}{(z-2)^2}
            = \frac{e^{-t}}{9}
        \]
        
        \begin{align*}
            \Res_2(h)
            &= \lim_{z \rightarrow 2} \frac{\dd}{\dz} \left[ (z - 2)^2 h(z) \right]
            = \lim_{z \rightarrow 2} \frac{\dd}{\dz} \frac{z^2 e^{zt}}{z+1}
            \\&= \frac{1}{9} \left( 8 e^{2t} + 12t e^{2t} \right)
            = (8 + 12t) \frac{e^{2t}}{9}
        \end{align*}
        
        Donc:
        
        \[ \int_{\gamma_r} \frac{z^2 e^{zt}}{(z+1)(z-2)^2} \dz = \frac{2 \pi i}{9} \left[ e^{-t} + (8 + 12t) e^{2t} \right] \]
        
        \item 
        On a aussi:
        
        \[
            \int_{\gamma_r} \frac{z^2 e^{zt}}{(z+1)(z-2)^2} \dz
            = \int_{C'_r} F(z) e^{zt} \dz + \int_{L_r} F(z) e^{zt} \dz
        \]
        
        Par ailleurs:
        \begin{itemize}
            \item 
            Si $\module{F(z)} \leq \frac{C}{|z|^k}$ avec $C \in \R$ et $k > 0$ pour $z \in C'_r$ avec $|z|$ suffisamment grand, alors on peut montrer que $\lim\limits_{r \rightarrow \infty} \int_{C'_r} F(z)e^{zt} \dz = 0$.
            
            Pour $F(z) = \frac{z^2}{(z+1)(z-2)^2}$, la condition est vérifiée avec $k = 1$.
            
            \item 
            \begin{align*}
                \lim_{r \rightarrow \infty} \int_{L_r} F(z) e^{zt} \dz
                &= \lim_{r \rightarrow \infty}
                i \int_{-\sqrt{r^2 - \gamma^2}}^{\sqrt{r^2 - \gamma^2}} F(\gamma + is) e^{(\gamma + is)t} \ds
                \\&= i \int_{-\infty}^\infty F(\gamma + is) e^{(\gamma + is)t} \ds
                = 2 \pi i \TLi(F)(t)
            \end{align*}
            
            avec $z = \gamma + is$, $\dz = i \ds$.
            On obtient la formule d'inversion de la transformée de Laplace.
            
            \item 
            \[
                \lim\limits_{r \rightarrow \infty} \int_{\gamma_r} \frac{z^2 e^{zt}}{(z+1)(z-2)^2} \dz
                = \frac{2 \pi i}{9} \left[ e^{-t} + (8 + 12t) e^{2t} \right]
            \]
        \end{itemize}
    \end{enumerate}

Finalement, lorsque $r \rightarrow \infty$, on obtient:

\[
    2 \pi i \TLi(F)(t) = \frac{2 \pi i}{9} \left[ e^{-t} + (8 + 12t) e^{2t} \right]
\]

D'après le théorème de la transformée de Laplace inverse, on a:

\[
    f(t) = \TLi(F)(t)
    = \frac{1}{9} \left[ e^{-t} + (8 + 12t) e^{2t} \right]
\]
\end{example}

\textit{Autres exemples: ex 1-2, série 13}


\section{Quelques applications de la transformée de Laplace}

\begin{enumerate}[label=\alph{enumi})]
    \item 
    Trouver une solution d'équations intégrales du type produit de convolution \textit{(cf ex.3, série 13)}.
    
    \item 
    Trouver une solution d'équations différentielles du type
    
    \[ a y''(t) + b y'(t) + c y(t) = f(t) \]
    
    pour $t > 0$ avec les conditions initiales $y(0)$ et $y'(0)$ données.
    
    \begin{example}
        Trouver une solution $y$ de:
        
        \[ y''(t) + 4 y'(t) + 3 y(t) = 0 \]
        
        pour $t > 0$ avec conditions initiales $y(0) = 3$ et $y'(0) = 1$.
        
        On écrit:
        
        \[ \TL(y'' + 4 y' + 3 y)(t) = \TL(f)(z) \]
        
        avec $f(t) = 0$ pour $t > 0$.
        
        \[ \implies \TL(y'')(z) + 4 \TL(y')(z) + 3 \TL(y)(z) = 0 \]
        
        car $\TL(f)(z) = 0$.
        En écrivant $\TL(y)(z) = Y(z)$, on a:
        
        \[ \implies \left[ z^2 Y(z) - z y(0) - y'(0) \right] + 4 \left[ z Y(z) - y(0) \right] + 3 Y(z) = 0 \]
        
        \[ \implies \left[ z^2 Y(z) - 3z - 1 \right] + 4 \left[ z Y(z) - 3 \right] + 3 Y(z) = 0 \]
        
        \[ \implies \left[ z^2 + 4z + 3 \right] Y(z) - 3z - 13 = 0 \]
        
        \[ \implies Y(z) = \frac{3z + 13}{z^2 + 4z + 3} = \frac{3z + 13}{(z+3)(z+1)} \]
        
        On calcule la transformée de Laplace inverse de $Y(z)$.
        
        Méthode des résidus (conditions vérifiées avec $k = 1$) avec:
        
        \[ h(z) = Y(z) e^{zt} = \frac{(3z + 13)e^{zt}}{(z+3)(z+1)} \]
        
        Deux pôles d'ordre 1: $z = -3$ et $z = -1$
        
        \[ \Res_{-3}(h) = \lim_{z \rightarrow -3} (z + 3)h(z)
        = \lim_{z \rightarrow -3} \frac{(3z + 13)e^{zt}}{z+1}
        = \frac{4 e^{-3t}}{-2} = -2 e^{-3t} \]
        
        \[ \Res_{-1}(h) = \lim_{z \rightarrow -1} (z + 1)h(z)
        = \lim_{z \rightarrow -1} \frac{(3z + 13)e^{zt}}{z+3}
        = \frac{10 e^{-t}}{2} = 5 e^{-t} \]
        
        La solution est
        
        \[ y(t) = \TLi(Y)(t) = \Res_{-3}(h) + \Res_{-1}(h) = 5 e^{-t} - 2 e^{-3t} \]
        
        \textit{Autre exemple: ex. 4, série 13}
    \end{example}
\end{enumerate}



\begin{thebibliography}{9}
    \bibitem{mainbook} 
    Bernard Dacogna et Chiara Tanteri. 
    \textit{Analyse avancée pour ingénieurs}. 
    PPUR, 2017.
\end{thebibliography}

Toutes les références (numéro de théorème, définition, etc.) sont faites à ce livre.


\section*{Contributeurs}

\begin{itemize}
    \item Robin Mamié (IN)
    \item Eric Jollès (SC)
    \item Yves Zumbach (IN)
    \item Victor Cochard (IN)
    \item Ghali Chraibi (SC)
    \item Charline Montial (IN)
    \item Amine Chaouachi (SC)
    \item Xiaoyan Zou (SC)
\end{itemize}
\end{document}