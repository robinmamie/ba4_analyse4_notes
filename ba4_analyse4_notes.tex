\documentclass[a4paper, 11pt]{report}

\usepackage[french]{babel}
\usepackage[a4paper]{geometry}
\usepackage{amsmath}
\usepackage{amsthm}
\usepackage{amssymb}
\usepackage{mathtools}


\newtheorem*{theorem}{Théorème}

\theoremstyle{definition}
\newtheorem*{definition}{Définition}
\newtheorem*{idea}{Idée}

\theoremstyle{remark}
\newtheorem*{terminology}{Terminologie}
\newtheorem*{note}{Note}
\newtheorem*{remark}{Remarque}
\newtheorem*{example}{Exemple}

\newcommand{\R}{\mathbb{R}}
\newcommand{\C}{\mathbb{C}}
\newcommand{\dx}{\mathrm{d}x}
\newcommand{\dt}{\mathrm{d}t}
\newcommand{\dal}{\mathrm{d}\alpha}
\newcommand{\TF}{\mathfrak{F}}
\newcommand{\TFi}{\TF^{-1}}

\title{\textbf{Analyse IV} \\
Transcript du cours du Pr. Michel \bsc{Cibils}}
\author{Robin \bsc{Mamié}}
\date{Printemps 2018}


\begin{document}

\maketitle

\tableofcontents

% !TeX spellcheck = fr_FR
\chapter{Transformées de Fourier}


\section{Introduction}


\subsection{Définitions et résultats préliminaires}

\begin{definition}
    Une fonction $f: \R \rightarrow \R$ est dite \textbf{T-périodique} s'il existe $T > 0$ tel que $f(x + T) = f(x) \enspace \forall x \in \R$.
    
    L'intervalle $[0, T]$ caractérise complètement la fonction.
\end{definition}


\begin{definition}[14.1.i, p.103]

    Une fonction $f: \R \rightarrow \R$ est dite \textbf{continue par morceaux} sur l'intervalle $[a, b]$ s'il existe des points $ \{x\}_{i = 0}^{n + 1} \subset [a, b]$ avec $a = x_0 < x_1 < \cdots < x_n < x_{n + 1} = b$ tels que pour $i = 0, 1, ..., n$ on ait:
    
    \begin{enumerate}

        \item
        $f$ est continue sur chaque intervalle ouvert $]x_i, x_{i + 1}[$

        \item
        la limite à droite $f(x_i + 0) := \lim\limits_{\substack{t \rightarrow x_i \\ t > x_i}} f(t)$ et la limite à gauche $f(x_{i + 1} - 0) := \lim\limits_{\substack{t \rightarrow x_i \\ t < x_i}} f(t)$ existent et sont finies.

    \end{enumerate}

\end{definition}


\begin{terminology}
    On dit qu'une fonction T-périodique est continue par morceaux si elle l'est sur l'intervalle $[0, T]$ qui la caractérise.
\end{terminology}


\begin{definition}[14.2, p.104]
    Soit $f: \R \rightarrow \R$ une fonction T-périodique continue par morceaux.
    Pour $N \in \mathbb{N}$, la \textbf{série de Fourier partielle d'ordre N} de $f$ est:
    
    \[
    F_N f(x) = \sum_{n = -N}^{N} c_n e^{i \frac{2 \pi n}{T} x}
    \]
    
    où les \textbf{coefficients de Fourier} $c_n$ sont des nombres complexes donnés par:
    
    \[
    c_n = \frac{1}{T} \int_{0}^{T} f(x) e^{-i \frac{2 \pi n}{T} x} \dx
    \]
    
    On appelle \textbf{série de Fourier de f} (en notation complexe) la limite lorsque $N \longrightarrow \infty$ de la série de $F_N f(x)$.
    On écrit:
    
    \[ F f(x) := \lim_{N \rightarrow +\infty} F_N f(x) = \sum_{-\infty}^{\infty} c_n e^{i \frac{2 \pi n}{T} x} \]
\end{definition}


\begin{theorem}[de Dirichlet -- Résultat de convergence; 14.3, p.104]
    Soit $f: \R \rightarrow \R$ une fonction T-périodique telle que $f$ et $f'$ soient continues par morceaux.
    Alors $\forall x \in \R$:
    
    \[
    F f(x) = \lim\limits_{N \rightarrow \infty} F_N f(x) \textrm{ existe et } F f(x) = \frac{f(x + 0) + f(x - 0)}{2}
    \]
    
    En particulier, si $f$ est continue en $x$, alors $f(x + 0) = f(x - 0) = f(x)$ et on a $F f(x) = f(x)$.
\end{theorem}


\begin{note}
    Utilisation de la formule d'Euler $e^{ix} = \cos x + i \sin x$ \textit{(cf. ex. 1-2, série 1)}.
\end{note}


\subsection{Motivation}

\begin{description}

    \item[Série de Fourier]
    développement des fonctions \textit{périodiques} comme somme infinie de fonctions trigonométriques.
    
    \item[Transformée de Fourier]
    étude de fonctions \textit{non} périodiques.

\end{description}


\begin{idea}
    Soit $T > 0$ et $f_T$ une fonction T-périodique définie par
    
    \[
    f_T (x) =
    \left \{
    \begin{array}{r c l}
        0 & \mathrm{si} & x \in \enspace ]-\frac{T}{2}, -1[ \\
        1 & \mathrm{si} & x \in [-1, 1] \\
        0 & \mathrm{si} & x \in \enspace ]1, \frac{T}{2}[ \\
    \end{array}
    \right .
    \]
    
    Lorsque la période $T \rightarrow \infty$, on a:
    
    \[
    \lim_{T \rightarrow \infty} f_T (x) = f(x) =
    \left \{
    \begin{array}{r c l}
        1 & \mathrm{si} & x \in [-1, 1] \\
        0 & \mathrm{si} & x \notin [-1, 1] \\
    \end{array}
    \right .
    \]
    
    qui n'est plus une fonction périodique.
    
\end{idea}


\begin{idea}
    considérer des fonctions comme limites de fonctions périodiques dont la période $T$ tend vers $+\infty$.
\end{idea}


\subsection{Raisonnement heuristique}

Soit $f_T: \R \rightarrow \R$ une fonction \textit{continue}, T-périodique telle que $f'_T$ soit continue par morceaux.
Alors la série de Fourier de $f_T$ est:

\[
F f_T(y) = \sum_{-\infty}^{+\infty} c_n e^{i \frac{2 \pi n}{T} y}
\]

pour $y \in \R$, où

\[
c_n = \frac{1}{T} \int_{0}^{T} f_T(x) e^{-i \frac{2 \pi n}{T} x} \dx
= \frac{1}{T} \int_{-\frac{T}{2}}^{\frac{T}{2}} f_T(x) e^{-i \frac{2 \pi n}{T} x} \dx
\]

En écrivant
$\Delta \alpha = \frac{2 \pi}{T}$
et
$\alpha_n = n \cdot \Delta \alpha$,
on a
$\frac{1}{T} = \frac{\Delta \alpha}{2 \pi}$.

\[
c_n =
\frac{\Delta \alpha}{2 \pi}
\int_{-\frac{T}{2}}^{\frac{T}{2}} f_T(x) e^{-i \alpha_n x} \dx
\]

\[
\Rightarrow
F f_T(y) =
\sum_{-\infty}^{+\infty}
\left[
\frac{\Delta \alpha}{2 \pi}
\int_{-\frac{T}{2}}^{\frac{T}{2}}
f_T(x) e^{-i \alpha_n x} \dx
\right]
e^{i \alpha_n y}
\]

Échange de la somme infinie et de l'intégrale:

\[
F f_T(y) =
\frac{1}{2 \pi}
\int_{-\frac{T}{2}}^{\frac{T}{2}}
f_T(x)
\left[
\Delta \alpha
\sum_{-\infty}^{+\infty}
e^{-i \alpha_n (x - y)}
\right]
\dx
\]

On découvre une somme de Riemann qui permet de définir une intégrale.
En effet:

\[
\Delta \alpha \sum_{-\infty}^{+\infty}
e^{-i \alpha_n (x - y)}
\underbrace{=}_{\Delta\alpha = \alpha_n - \alpha_{n - 1}}
\sum_{-\infty}^{+\infty}
e^{-i \alpha_n (x - y)} (\alpha_n - \alpha_{n - 1})
\]

\[
=
\int_{-\infty}^{+\infty}
e^{-i \alpha (x - y)}
\dal
\]

Donc on obtient:

\begin{align*}
F f_T(y) &=
\frac{1}{2 \pi}
\int_{-\frac{T}{2}}^{\frac{T}{2}}
f_T(x)
\left[
\int_{-\infty}^{+\infty}
e^{-i \alpha (x - y)}
\dal
\right]
\dx
\\&=
\frac{1}{\sqrt{2 \pi}}
\int_{-\infty}^{+\infty}
\left[
\frac{1}{\sqrt{2 \pi}}
\int_{-\frac{T}{2}}^{\frac{T}{2}}
f_T(x) e^{-i \alpha x}
\dx
\right]
e^{i \alpha y}
\dal
\end{align*}

Comme $f_T$ est continue, alors on a $f_T(y) = F f_T(y)$ et donc lorsque $T$ tend vers $+\infty$, on a
$\lim\limits_{T \rightarrow +\infty} f_T(y) =
\lim\limits_{T \rightarrow +\infty} F f_T(y) \iff
f(y) = \lim\limits_{T \rightarrow +\infty} F f_T(y)$.

\[
\iff
f(y) =
\frac{1}{\sqrt{2 \pi}}
\int_{-\infty}^{+\infty}
\underbrace{
\left[
\frac{1}{\sqrt{2 \pi}}
\int_{-\infty}^{+\infty}
f(x) e^{-i \alpha x}
\dx
\right]
}_{\substack{
\textrm{Nouvelle fonction qui dépend de la}\\
\textrm{variable $\alpha$, qui est appelée la}\\
\textrm{transformée de Fourier de $f$ et}\\
\textrm{notée $\TF(f)$ ou $\hat{f}$}
}}
e^{i \alpha y}
\dal
\]

On écrit:

\[
\TF(f)(\alpha) = \hat{f}(\alpha) =
\frac{1}{\sqrt{2 \pi}}
\int_{-\infty}^{+\infty}
f(x) e^{-i \alpha x}
\dx
\]

\begin{remark}
    On a que:
    
    \[
    f(y) =
    \frac{1}{\sqrt{2 \pi}}
    \int_{-\infty}^{+\infty}
    \hat{f}(\alpha)
    e^{i \alpha y}
    \dal
    \]
\end{remark}



\section{Transformée de Fourier d'une fonction}


\subsection{Définition}

\begin{definition}[15.1, p.113]
    Soit $f: \R \rightarrow \R$ une fonction continue par morceaux et telle que $\int_{-\infty}^{+\infty} | f(x) | \dx < \infty$.
    
    La \textbf{transformée de Fourier} de $f$ est la fonction notée $\TF(f)$ ou $\hat{f}: \R \rightarrow \C$ définie par:
    
    \[
    \alpha \longmapsto
    \TF(f)(\alpha) = \hat{f}(\alpha) =
    \frac{1}{\sqrt{2 \pi}}
    \int_{-\infty}^{+\infty}
    f(x) e^{-i \alpha x}
    \dx
    \]
\end{definition}


\subsection{Exemples}

\begin{example}
    Calculer la transformée de Fourier de la fonction:
    
    \[
    f:
    \begin{array}{r c l}
        \R & \rightarrow & \R_+^* \\
        x  & \mapsto & f(x) = e^{-|x|} =
        \left \{
        \begin{array}{r c l}
            e^{-x} & \textrm{si} & x \geq 0 \\
            e^x    & \textrm{si} & x < 0
        \end{array}
        \right.
    \end{array}
    \]
    
    \begin{align*}
    \hat{f}(\alpha) & =
    \frac{1}{\sqrt{2 \pi}}
    \int_{-\infty}^{+\infty}
    f(x)
    e^{-i \alpha x}
    \dx
    =
    \frac{1}{\sqrt{2 \pi}}
    \int_{-\infty}^{+\infty}
    e^{-|x|}
    e^{-i \alpha x}
    \dx
    \\&=
    \frac{1}{\sqrt{2 \pi}}
    \left[
    \int_{-\infty}^{0}
    e^{x}
    e^{-i \alpha x}
    \dx
    +
    \int_{0}^{+\infty}
    e^{-x}
    e^{-i \alpha x}
    \dx
    \right]
    \\&=
    \frac{1}{\sqrt{2 \pi}}
    \left[
    \int_{-\infty}^{0}
    e^{(1 - i \alpha) x}
    \dx
    +
    \int_{0}^{+\infty}
    e^{-(1 + i \alpha) x}
    \dx
    \right]
    \\&=
    \frac{1}{\sqrt{2 \pi}}
    \left[
    \left.
    \frac{e^{(1 - i \alpha) x}}
    {1 - i \alpha}
    \right|_{-\infty}^0
    -
    \left.
    \frac{e^{-(1 + i \alpha) x}}
    {1 + i \alpha}
    \right|_0^{+\infty}
    \right]
    \\&=
    \frac{1}{\sqrt{2 \pi}}
    \left[
    \frac{1}
    {1 - i \alpha}
    \left(
    1 - \lim\limits_{x \rightarrow -\infty} e^{(1 - i \alpha) x}
    \right)
    \right.
    -
    \left.
    \frac{1}
    {1 + i \alpha}
    \left(
    \lim\limits_{x \rightarrow + \infty} e^{-(1 + i \alpha) x} - 1
    \right)
    \right]
    \\&=
    \frac{1}{\sqrt{2 \pi}}
    \left(
    \frac{1}{1 - i \alpha}
    +
    \frac{1}{1 + i \alpha}
    \right)
    \\&=
    \frac{1}{\sqrt{2 \pi}}
    \frac{1 + i \alpha + 1 - i \alpha}
    {(1 - i \alpha) (1 + i \alpha)}
    =
    \frac{1}{\sqrt{2 \pi}}
    \frac{2}{1 + \alpha^2}
    \end{align*}
    
    \textbf{Résultat:}
    
    \[
    \hat{f}:
    \begin{array}{r c l}
    \R & \rightarrow & \R_+^* \\
    \alpha & \mapsto & \hat{f}(\alpha) =
    \sqrt{\frac{2}{\pi}} \frac{1}{1 + \alpha^2}
    \end{array}
    \]
\end{example}


\begin{remark}
    Pour calculer
    $\lim\limits_{x \rightarrow -\infty} e^{(1 - i \alpha) x}$:
    
    \begin{align*}
    \left|e^{(1 - i \alpha) x}\right|
    &= \left|e^{- i \alpha x} e^x\right|
    = \underbrace{\left|e^{- i \alpha x}\right|}_{= 1} \left|e^x\right|
    = e^x\\
    \Rightarrow \lim\limits_{x \rightarrow -\infty} \left|e^{(1 - i \alpha) x}\right|
    &= \lim\limits_{x \rightarrow -\infty} e^x
    = 0\\
    \Rightarrow \lim\limits_{x \rightarrow -\infty} e^{(1 - i \alpha) x}
    &= 0 + i0
    = 0
    \end{align*}
    
    On a aussi $\lim\limits_{x \rightarrow +\infty} e^{-(1 + i \alpha) x} = 0$.
\end{remark}


\begin{example}
    Soit la fonction 
    
    \[
    f:
    \begin{array}{r c l}
    \R & \rightarrow & \R \\
    x  & \mapsto & f(x) = 
    \left \{
    \begin{array}{r c l}
    1 & \textrm{si} & x \in [-1, 1] \\
    0   & \textrm{sinon} &
    \end{array}
    \right.
    \end{array}
    \]
    
    Calcul de la transformée de Fourier de $f$.
    
    \begin{align*}
    \hat{f}(\alpha) & =
    \frac{1}{\sqrt{2 \pi}}
    \int_{-\infty}^{+\infty}
    f(x)
    e^{-i \alpha x}
    \dx
    =
    \frac{1}{\sqrt{2 \pi}}
    \int_{-1}^{1}
    e^{-i \alpha x}
    \dx
    \\&\overbrace{=}^{\alpha \neq 0}
    -
    \frac{1}{\sqrt{2 \pi}}
    \left.
    \frac{e^{-i \alpha x}}{i \alpha}
    \right|_{-1}^{1}
    =
    \frac{1}{\sqrt{2 \pi}}
    \frac{e^{i \alpha} - e^{-i \alpha}}{i \alpha}
    \\&=
    \frac{2}{\sqrt{2 \pi}}
    \frac{1}{\alpha}
    \frac{e^{i \alpha} - e^{-i \alpha}}{2i}
    =
    \sqrt{\frac{2}{\pi}}
    \frac{\sin \alpha}{\alpha}
    \quad
    \text{si } \alpha \neq 0
    \end{align*}
    
    Pour $\alpha = 0$ :
    
    \begin{align*}
    \hat{f}(0) &=
    \frac{1}{\sqrt{2 \pi}}
    \int_{-\infty}^{+\infty}
    f(x)
    e^{-i 0 x}
    \dx
    =
    \frac{1}{\sqrt{2 \pi}}
    \int_{-1}^{1}
    \dx
    =
    \left.
    \frac{1}{\sqrt{2 \pi}}
    x
    \right|_{-1}^{1}
    =
    \sqrt{\frac{2}{\pi}}
    \end{align*}
    
    \textbf{Résultat:}
    
    \[
    \hat{f}:
    \begin{array}{r c l}
    \R & \rightarrow & \R \\
    \alpha & \mapsto & \hat{f}(\alpha) =
    \left\{
    \begin{array}{l c l}
    \sqrt{\frac{2}{\pi}}
    \frac{\sin \alpha}{\alpha} & \mathrm{si} & \alpha \neq 0\\
    \sqrt{\frac{2}{\pi}} & \mathrm{si} & \alpha = 0
    \end{array}
    \right.
    \end{array}
    \]
    
    On remarque que $\lim\limits_{\alpha \rightarrow 0} \hat{f}(\alpha) = \lim\limits_{\alpha \rightarrow 0} \sqrt{\frac{2}{\pi}} \frac{\sin \alpha}{\alpha} = \sqrt{\frac{2}{\pi}} = \hat{f}(0)$.
    
    $\implies \hat{f}$ est aussi continue en $\alpha = 0$.
    
    \textit{Autres exemples: ex. 3-4, série 1}
    
\end{example}



\section{Transformée de Fourier inverse}


\subsection{Définition}

\begin{definition}
    Soit $g: \R \rightarrow \C$ une fonction continue par morceaux telle que $\int_{-\infty}^{+ \infty} |g(t)| \dt < \infty$.
    La \textbf{transformée de Fourier inverse} de $g$ est notée:
    
    \[
    \TFi(g):
    \begin{array}{r c l}
    \R & \rightarrow & \C \\
    x & \rightarrow & \TFi(g)(x)
    \end{array}, \textrm{ où}
    \enspace
    \TFi(g)(x) =
    \frac{1}{\sqrt{2 \pi}} \int_{-\infty}^{+\infty} g(t) e^{itx} \dt.
    \]
\end{definition}


\subsection{Théorème de réciprocité (formule d'inversion)}

\begin{theorem}[15.3.i, p.115]
    Soit $f: \R \rightarrow \R$ une fonction telle que $f$ et $f'$ soient continues par morceaux avec $\int_{-\infty}^{+ \infty} |f(x)| dx < \infty$ et $\int_{-\infty}^{+ \infty} |\hat{f}(\alpha)| \dal < \infty$.
    Alors $\forall x \in \R$, on a:
    
    \[
    \TFi (\hat{f})(x) =
    \frac{1}{\sqrt{2 \pi}}\int_{-\infty}^{+ \infty} \hat{f}(\alpha) e^{i\alpha x}\dal =
    \frac{f(x+0)+f(x-0)}{2}
    \]
    
    En particulier si $f$ est continue en $x$, on a $\frac{1}{2}[f(x+0) + f(x-0)] = f(x)$ et alors:
    
    \[
    f(x) = \TFi (\hat{f})(x) =
    \frac{1}{\sqrt{2 \pi}}
    \int_{-\infty}^{+ \infty}
    \hat{f}(\alpha) e^{i\alpha x}
    \dal
    \]
    
    Autrement dit, on a $\TFi(\TF(f)) = f$.
    La transformée de Fourier peut être vue comme une \og transformation $\TF$ \fg{} inversible (une bijection) qui \og agit\fg{} sur la fonction $f$:
    
    \[
    f \enspace
    \xrightarrow{\enspace \TF \enspace}
    \enspace \hat{f} \enspace
    \xrightarrow{\TFi}
    \enspace f
    \]
\end{theorem}


\subsection{Exemple d'utilisation}

\begin{example}
    Soit $
    f:
    \begin{array}{r c l}
    \R & \rightarrow & \R_+^* \\
    x  & \mapsto & f(x) = e^{-|x|} =
    \left \{
    \begin{array}{r c l}
    e^{-x} & \textrm{si} & x \geq 0 \\
    e^x    & \textrm{si} & x < 0
    \end{array}
    \right.
    \end{array}
    $.
    
    La transformée de Fourier de $f$ est (\textit{exemple 1, §1.2.2})
    
    \[
    \hat{f}:
    \begin{array}{r c l}
    \R & \rightarrow & \R_+^* \\
    \alpha & \mapsto & \hat{f}(\alpha) =
    \sqrt{\frac{2}{\pi}} \frac{1}{1 + \alpha^2}
    \end{array}
    \]

    On remarque que $f'(x) =
    \left\{
    \begin{array}{r c l}
    -e^{-x} & \textrm{si} & x \geq 0 \\
    e^x    & \textrm{si} & x < 0
    \end{array}
    \right.$
    est continue par morceaux car:
    
    \[\lim\limits_{\substack{x \rightarrow 0 \\ x > 0}} f'(x) = -1 \textrm{ et } \lim\limits_{\substack{x \rightarrow 0 \\ x < 0}} f'(x) = 1\]
    
    De plus, $\int_{-\infty}^{+\infty} |\hat{f}|\dx = \sqrt{\frac{2}{\pi}} \int_{-\infty}^{+\infty} \frac{\dal}{1 + \alpha^2} \dal < \infty$.
    $f$ est continue $\forall x \in \R \implies$ en appliquant le \textit{théorème de réciprocité}, on a que $f(x) = \frac{1}{\sqrt{2 \pi}} \int_{-\infty}^{+\infty} \hat{f}(\alpha) e^{i\alpha x}$, i.e.:
    
    \[
    e^{-|x|} =
    \frac{1}{\sqrt{2 \pi}} \sqrt{\frac{2}{\pi}} \int_{-\infty}^{+\infty} \frac{e^{i\alpha x}}{1 + \alpha^2}\dal
    \quad \forall x \in \R
    \]
    
    En particulier, lorsque $x = 0$, on trouve:
    
   \[1 = \frac{1}{\pi} \int_{-\infty}^{+\infty} \frac{\dal}{1 + \alpha^2} \implies \int_{-\infty}^{+\infty} \frac{\dal}{1 + \alpha^2} = \pi.
   \]
    
    En particulier, lorsque $x = 1$, on trouve:
    
    \begin{align*}
    e^{-1} &=
    \frac{1}{\pi} \int_{-\infty}^{+\infty} \frac{e^{i \alpha}}{1 + \alpha^2}\dal =
    \frac{1}{\pi} \left[ \int_{-\infty}^{+\infty} \frac{\cos \alpha}{1 + \alpha^2} \dal +
    i \overbrace{\int_{-\infty}^{+\infty} \frac{\sin \alpha}{1 + \alpha^2} \dal}^{\substack{\textrm{fonction impaire} \\ \textrm{intégrée sur tout l'axe réel } = 0}} \right]
    \\&\implies
    \int_{-\infty}^{+\infty} \frac{\cos \alpha}{1 + \alpha^2} \dal = \frac{\pi}{e}
    \end{align*}

\end{example}

\textbf{Conclusion:} Le \textit{théorème de réciprocité} permet de calculer la valeur d'intégrales généralisées.

\textit{Autre exemple: ex. 1, série 2}



\section{Propriétés de la transformée de Fourier}


On considère $f$ et $g: \R \rightarrow \R$ continues par morceaux telles que $\int_{-\infty}^{+\infty} |f(x)| \dx < \infty$ et $\int_{-\infty}^{+\infty} |g(x)| \dx < \infty$.
On note indifféremment $\TF(f) \doteq \hat{f}$ et $\TF(g) \doteq \hat{g}$ les transformées de Fourier de $f$ et de $g$.

\begin{note}
    Les prochains résultats sont décrits dans les \textbf{théorèmes 15.2 et 15.3} aux pages 113 à 115 du livre du cours.
\end{note}


\subsection{Continuité et linéarité}

\begin{itemize}
\item
$\TF(f)$ est continue $\forall \alpha \in \R$ et $\lim\limits_{\alpha \rightarrow \pm \infty} |\TF(f)(\alpha)| = 0$.
\item
$\TF$ linéaire: $\TF(af + bg) = a\,\TF(f) + b\,\TF(g) \quad \forall a,b \in \R$.
\end{itemize}


\subsection{Transformée de Fourier du produit de convolution}

\begin{definition}
    Le \textbf{produit de convolution} de deux fonctions $f$ et $g$ est la fonction notée $f \ast g: \R \rightarrow \R$ définie par:
    
    \[(f \ast g)(x) := \int_{-\infty}^{+\infty} f(x - t) g(t) \dt\]
\end{definition}

\begin{remark}
    On peut aussi écrire $(f \ast g)(x) := \int_{-\infty}^{+\infty} f(t') g(x - t') \dt'$, via un changement de variable.
\end{remark}

\textbf{Résultat:} on a que $\TF(f \ast g) = \sqrt{2\pi} \enspace \TF(f) \cdot \TF(g)$.

La transformée de Fourier du produit de convolution de deux fonctions est égale au \textbf{produit} des transformées de Fourier de chaque fonction.

\textit{Exemples: ex. 2-3, série 2}


\subsection{Transformée de Fourier de la dérivée d'une fonction}

Si de plus $f \in C^1(\R)$ et $\int_{-\infty}^{+\infty} |f'(x)| \dx < \infty$, alors on a :

\[\TF(f')(\alpha) = i\alpha \enspace \TF(f)(\alpha) \quad \forall \alpha \in \R\]

On écrit aussi $\widehat{f'}(\alpha) = i\alpha \enspace \hat{f}(\alpha)$.

La transformée de Fourier de la dérivée de $f$ s'obtient en \textbf{multipliant par} $i\alpha$ la transformée de Fourier de $f$.

Plus généralement, si $f\in C^n(\R)$ et $\int_{-\infty}^{+\infty} |f^{(k)}(x)| \dx < \infty$ pour $k=1,2,...,n$, alors on a:

\[\TF(f^{(k)})(\alpha) = (i\alpha)^k \enspace \TF(f)(\alpha) \quad \forall \alpha \in \R, k = 1,2,...,n\]

On écrit aussi $\widehat{f^{(k)}}(\alpha) = (i\alpha)^k \enspace \hat{f}(\alpha)$.

\textit{Exemple: ex.3, série 2}


\subsection{Décalage}

Si $a \in \R^*, b \in \R$ et $h(x) = e^{-ibx}f(ax)$, alors:

\[\TF(h)(\alpha) = \frac{1}{|a|} \TF(f)(\frac{\alpha + b}{a})\]


\subsection{Identité de Plancherel}

Si de plus $\int_{-\infty}^{+\infty} [f(x)]^2 \dx < \infty$, alors on a:

\[\int_{-\infty}^{+\infty} [f(x)]^2 \dx = \int_{-\infty}^{+\infty} |\TF(f)(\alpha)|^2 \dal\]


\subsection{Transformée de Fourier en sinus et cosinus}

Si la fonction $f$ est paire (i.e. $f(-x) = f(x) \enspace \forall x \in \R$), alors on a:

\[
\TF(f)(\alpha) =
\sqrt{\frac{2}{\pi}}
\int_{0}^{+\infty} f(x)\cos(\alpha x) \dx
\]

qui est la transformée de Fourier en \textbf{cosinus} de $f$.

Si la fonction $f$ est impaire (i.e. $f(-x) = -f(x) \enspace \forall x \in \R$), alors on a:

\[
\TF(f)(\alpha) =
- i
\sqrt{\frac{2}{\pi}}
\int_{0}^{+\infty} f(x)\sin(\alpha x) \dx
\]

qui est la transformée de Fourier en \textbf{sinus} de $f$.

\textit{Exemples: ex.4, série 2 et ex.1, série 3}

\begin{remark}
    Si de plus $f'$ est continue par morceaux et $\int_{-\infty}^{+\infty} |\hat{f}(\alpha)| \dal < \infty$, alors, d'après le \textit{théorème de réciprocité}, on a:
    
    \begin{align*}
    f(x) &=
    \sqrt{\frac{2}{\pi}}
    \int_{0}^{+\infty} \hat{f}(\alpha)\cos(\alpha x) \dal
    \quad
    \textrm{ lorsque } f \textrm{ est paire}\\
    f(x) &=
    i
    \sqrt{\frac{2}{\pi}}
    \int_{0}^{+\infty} \hat{f}(\alpha)\sin(\alpha x) \dal
    \quad
    \textrm{ lorsque } f \textrm{ est impaire}\\
    \end{align*}
\end{remark}


\begin{thebibliography}{9}
    \bibitem{mainbook} 
    Bernard Dacogna et Chiara Tanteri. 
    \textit{Analyse avancée pour ingénieurs}. 
    PPUR, 2017.
\end{thebibliography}

Toutes les références (numéro de théorème, définition, etc.) sont faites à ce livre.
\end{document}