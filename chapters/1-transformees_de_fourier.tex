% !TeX spellcheck = fr_FR
\chapter{Transformées de Fourier}


\section{Introduction}


\subsection{Définitions et résultats préliminaires}


\begin{definition}

    Une fonction $f: \R \rightarrow \R$ est dite \textbf{T-périodique} s'il existe $T > 0$ tel que $f(x + T) = f(x) \enspace \forall x \in \R$.
    L'intervalle $[0, T]$ caractérise complètement la fonction.

\end{definition}


\begin{definition}

    Une fonction $f: \R \rightarrow \R$ est dite \textbf{continue par morceaux} sur l'intervalle $[a, b]$ s'il existe des points $ \{x\}_{i = 0}^{n + 1} \subset [a, b]$ avec $a = x_0 < x_1 < \cdots < x_n < x_{n + 1} = b$ tels que pour $i = 0, 1, ..., n$ on ait:
    
    \begin{enumerate}

        \item
        $f$ est continue sur chaque intervalle ouvert $]x_i, x_{i + 1}[$

        \item
        la limite à droite $f(x_i + 0) := \lim\limits_{\substack{t \rightarrow x_i \\ t > x_i}} f(t)$ et la limite à gauche $f(x_{i + 1} - 0) := \lim\limits_{\substack{t \rightarrow x_i \\ t < x_i}} f(t)$ existent et sont finies.

    \end{enumerate}

\end{definition}


\begin{terminology}
    On dit qu'une fonction T-périodique est continue par morceaux si elle l'est sur l'intervalle $[0, T]$ qui la caractérise.
\end{terminology}


\begin{definition}
    Soit $f: \R \rightarrow \R$ une fonction T-périodique continue par morceaux.
    Pour $N \in \mathbb{N}$, la \textbf{série de Fourier partielle d'ordre N} de $f$ est:
    
    \[
    F_N f(x) = \sum_{n = -N}^{N} c_n e^{i \frac{2 \pi n}{T} x}
    \]
    
    où les \textbf{coefficients de Fourier} $c_n$ sont des nombres complexes donnés par:
    
    \[
    c_n = \frac{1}{T} \int_{0}^{T} f(x) e^{-i \frac{2 \pi n}{T} x} \dx
    \]
    
    On appelle \textbf{série de Fourier de f} (en notation complexe) la limite lorsque $N \longrightarrow \infty$ de la série de $F_N f(x)$.
    On écrit:
    
    \[ F f(x) := \lim_{N \rightarrow \infty} F_N f(x) = \sum_{n = -\infty}^{\infty} c_n e^{i \frac{2 \pi n}{T} x} \]
\end{definition}


\begin{theorem*}[de Dirichlet -- Résultat de convergence]
    Soit $f: \R \rightarrow \R$ une fonction T-périodique telle que $f$ et $f'$ soient continues par morceaux.
    Alors $\forall x \in \R : F f(x) = \lim\limits_{N \rightarrow \infty} F_N f(x)$ existe et $F f(x) = \frac{f(x + 0) + f(x - 0)}{2}$.
    En particulier, si $f$ est continue en $x$, alors $f(x + 0) = f(x - 0) = f(x)$ et on a $F f(x) = f(x)$.
\end{theorem*}


\begin{note}
    Utilisation de la formule d'Euler $e^{ix} = \cos x + i \sin x$ (cf. ex. 1-2, série 1).
\end{note}



\subsection{Motivation}



\begin{description}

    \item[Série de Fourier]
    développement des fonctions \textit{périodiques} comme somme infinie de fonctions trigonométriques.
    
    \item[Transformée de Fourier]
    étude de fonctions \textit{non} périodiques.

\end{description}


\begin{idea}
    Soit $T > 0$ et $f_T$ une fonction T-périodique définie par
    
    \[
    f_T (x) =
    \left \{
    \begin{array}{r c l}
        0 & \mathrm{si} & x \in \enspace ]-\frac{T}{2}, -1[ \\
        1 & \mathrm{si} & x \in [-1, 1] \\
        0 & \mathrm{si} & x \in \enspace ]1, \frac{T}{2}[ \\
    \end{array}
    \right .
    \]
    
    Lorsque la période $T \rightarrow \infty$, on a:
    
    \[
    \lim_{T \rightarrow \infty} f_T (x) = f(x) =
    \left \{
    \begin{array}{r c l}
        1 & \mathrm{si} & x \in [-1, 1] \\
        0 & \mathrm{si} & x \notin [-1, 1] \\
    \end{array}
    \right .
    \]
    
    qui n'est plus une fonction périodique.
    
\end{idea}


\begin{idea}
    considérer des fonctions comme limites de fonctions périodiques dont la périodiques dont la période $T$ tend vers $+\infty$.
\end{idea}


\subsection{Raisonnement heuristique}

Soit $f_T: \R \rightarrow \R$ une fonction \textit{continue}, T-périodique telle que $f'_T$ soit continue par morceaux.
Alors la série de Fourier de $f_T$ est:

\[
F f_T(y) = \sum_{n = -\infty}^{+\infty} c_n e^{i \frac{2 \pi n}{T} y}
\]

pour $y \in \R$, où

\[
c_n = \frac{1}{T} \int_{0}^{T} f_T(x) e^{-i \frac{2 \pi n}{T} x} \dx
= \frac{1}{T} \int_{-\frac{T}{2}}^{\frac{T}{2}} f_T(x) e^{-i \frac{2 \pi n}{T} x} \dx
\]

En écrivant
$\Delta \alpha = \frac{2 \pi}{T}$
et
$\alpha_n = n \cdot \Delta \alpha$,
on a
$\frac{1}{T} = \frac{\Delta \alpha}{2 \pi}$.

\[
c_n =
\frac{\Delta \alpha}{2 \pi}
\int_{-\frac{T}{2}}^{\frac{T}{2}} f_T(x) e^{-i \alpha_n x} \dx
\]

\[
\Rightarrow
F f_T(y) =
\sum_{n = -\infty}^{+\infty}
\left[
\frac{\Delta \alpha}{2 \pi}
\int_{-\frac{T}{2}}^{\frac{T}{2}}
f_T(x) e^{-i \alpha_n x} \dx
\right]
e^{i \alpha_n y}
\]

Échange de la somme infinie et de l'intégrale:

\[
F f_T(y) =
\frac{1}{2 \pi}
\int_{-\frac{T}{2}}^{\frac{T}{2}}
f_T(x)
\left[
\Delta \alpha
\sum_{n = -\infty}^{+\infty}
e^{-i \alpha_n (x - y)}
\right]
\dx
\]

On décrouvre une somme de Riemann qui permet de définir une intégrale.
En effet:

\[
\Delta \alpha \sum_{n = -\infty}^{+\infty}
e^{-i \alpha_n (x - y)}
\underbrace{=}_{\Delta\alpha = \alpha_n - \alpha_{n - 1}}
\sum_{n = -\infty}^{+\infty}
e^{-i \alpha_n (x - y)} (\alpha_n - \alpha_{n - 1})
\]

\[
=
\int_{-\infty}^{+\infty}
e^{-i \alpha (x - y)}
\dal
\]

Donc on obtient:

\[
F f_T(y) =
\frac{1}{2 \pi}
\int_{-\frac{T}{2}}^{\frac{T}{2}}
f_T(x)
\left[
\int_{-\infty}^{+\infty}
e^{-i \alpha (x - y)}
\dal
\right]
\dx
\]

\[
=
\frac{1}{\sqrt{2 \pi}}
\int_{-\infty}^{+\infty}
\left[
\frac{1}{\sqrt{2 \pi}}
\int_{-\frac{T}{2}}^{\frac{T}{2}}
f_T(x) e^{-i \alpha x}
\dx
\right]
e^{i \alpha y}
\dal
\]

Comme $f_T$ est continue, alors on a $f_T(y) = F f(y)$ et donc lorsque $T$ tend vers $+\infty$, on a
$\lim\limits_{T \rightarrow \infty} f_T(y) =
\lim\limits_{T \rightarrow \infty} F f_T(y) \iff
f(y) = \lim\limits_{T \rightarrow \infty} F f_T(y)$.

\[
\iff
f(y) =
\frac{1}{\sqrt{2 \pi}}
\int_{-\infty}^{+\infty}
\underbrace{
\left[
\frac{1}{\sqrt{2 \pi}}
\int_{-\infty}^{+\infty}
f(x) e^{-i \alpha x}
\dx
\right]
}_{\substack{
\textrm{Nouvelle fonction qui dépend de la}\\
\textrm{variable $\alpha$, qui est appelée la}\\
\textrm{transformée de Fourier de $f$ et}\\
\textrm{notée $\mathfrak{F}(f)$ ou $\hat{f}$}
}}
e^{i \alpha y}
\dal
\]

On écrit:

\[
\mathfrak{F}(f)(\alpha) = \hat{f}(\alpha) =
\frac{1}{\sqrt{2 \pi}}
\int_{-\infty}^{+\infty}
f(x) e^{-i \alpha x}
\dx
\]

\begin{remark}
    On a que:
    
    \[
    f(y) =
    \frac{1}{\sqrt{2 \pi}}
    \int_{-\infty}^{+\infty}
    \hat{f}(\alpha)
    e^{i \alpha y}
    \dal
    \]
\end{remark}



\section{Transformée de Fourier d'une fonction}



\subsection{Définitions}



\begin{definition}
    Soit $f: \R \rightarrow \R$ une fonction continue par morceaux et telle que $\int_{-\infty}^{+\infty} | f(x) | \dx < \infty$.
    La \textbf{transformée de Fourier} de $f$ est la fonction notée $\mathfrak{F}(f)$ ou $\hat{f}: \R \rightarrow \C$ définie par:
    
    \[
    \alpha \longmapsto
    \mathfrak{F}(f)(\alpha) = \hat{f}(\alpha) =
    \frac{1}{\sqrt{2 \pi}}
    \int_{-\infty}^{+\infty}
    f(x) e^{-i \alpha x}
    \dx
    \]
\end{definition}



\subsection{Définitions}



\begin{example}
    Calculer la transformée de Fourier de la fonction:
    
    \[
    f:
    \begin{array}{r c l}
        \R & \rightarrow & \R_+^* \\
        x  & \rightarrow & f(x) = e^{-|x|} =
        \left \{
        \begin{array}{r c l}
            e^{-x} & \textrm{si} & x \geq 0 \\
            e^x    & \textrm{si} & x < 0
        \end{array}
        \right.
    \end{array}
    \]
    
    \begin{align*}
    \hat{f}(\alpha) & =
    \frac{1}{\sqrt{2 \pi}}
    \int_{-\infty}^{+\infty}
    f(x)
    e^{-i \alpha x}
    \dx
    =
    \frac{1}{\sqrt{2 \pi}}
    \int_{-\infty}^{+\infty}
    e^{-|x|}
    e^{-i \alpha x}
    \dx
    \\&=
    \frac{1}{\sqrt{2 \pi}}
    \left[
    \int_{-\infty}^{0}
    e^{x}
    e^{-i \alpha x}
    \dx
    +
    \int_{0}^{+\infty}
    e^{-x}
    e^{-i \alpha x}
    \dx
    \right]
    \\&=
    \frac{1}{\sqrt{2 \pi}}
    \left[
    \int_{-\infty}^{0}
    e^{(1 - i \alpha) x}
    \dx
    +
    \int_{0}^{+\infty}
    e^{-(1 + i \alpha) x}
    \dx
    \right]
    \\&=
    \frac{1}{\sqrt{2 \pi}}
    \left[
    \left.
    \frac{e^{(1 - i \alpha) x}}
    {1 - i \alpha}
    \right|_{-\infty}^0
    -
    \left.
    \frac{e^{-(1 + i \alpha) x}}
    {1 + i \alpha}
    \right|_0^{+\infty}
    \right]
    \\=&
    \frac{1}{\sqrt{2 \pi}}
    \left[
    \frac{1}
    {1 - i \alpha}
    \left(
    1 - \lim\limits_{x \rightarrow -\infty} e^{(1 - i \alpha) x}
    \right)
    \right.
    \\& -
    \left.
    \frac{1}
    {1 + i \alpha}
    \left(
    \lim\limits_{x \rightarrow + \infty} e^{-(1 + i \alpha) x} - 1
    \right)
    \right]
    \\&=
    \frac{1}{\sqrt{2 \pi}}
    \left(
    \frac{1}{1 + i \alpha}
    +
    \frac{1}{1 + i \alpha}
    \right)
    \\&=
    \frac{1}{\sqrt{2 \pi}}
    \frac{1 + i \alpha + 1 - i \alpha}
    {(1 - i \alpha) (1 + i \alpha)}
    =
    \frac{1}{\sqrt{2 \pi}}
    \frac{2}{1 + \alpha^2}
    \end{align*}
    
    Résultat:
    
    \[
    \hat{f}:
    \begin{array}{r c l}
    \R & \rightarrow & \R_+^* \\
    \alpha & \rightarrow & \hat{f}(\alpha) =
    \sqrt{\frac{2}{\pi}} \frac{1}{1 + \alpha^2}
    \end{array}
    \]
    
    Autres exemples: ex. 3-4, série 1
\end{example}


\begin{remark}
    Pour calculer
    $\lim\limits_{x \rightarrow -\infty} e^{(1 - i \alpha) x}$:
    
    \begin{align*}
    \left|e^{(1 - i \alpha) x}\right|
    &= \left|e^{- i \alpha x} e^x\right|
    = \underbrace{\left|e^{- i \alpha x}\right|}_{= 1} \left|e^x\right|
    = e^x\\
    \Rightarrow \lim\limits_{x \rightarrow -\infty} \left|e^{(1 - i \alpha) x}\right|
    &= \lim\limits_{x \rightarrow -\infty} e^x
    = 0\\
    \Rightarrow \lim\limits_{x \rightarrow -\infty} e^{(1 - i \alpha) x}
    &= 0 + i0
    = 0
    \end{align*}
    
    On a aussi $\lim\limits_{x \rightarrow +\infty} e^{-(1 + i \alpha) x} = 0$.
\end{remark}


















