% !TeX spellcheck = fr_FR
\chapter{Séries de Laurent, pôles et résidus}


\section{Polynôme et série de Taylor d'une fonction holomorphe}

\subsection{Définitions et résultats}

\begin{hypothesis}
    Soit un ouvert $D \subset \Cx$ et $f: \begin{array}{ccc}
    D & \longrightarrow & \Cx\\
    z & \longmapsto & f(z)
    \end{array}$ une fonction holomorphe dans $D$ et $z_0 \in D$.
\end{hypothesis}

\begin{definition}
    Pour $N \in \N$, le \textbf{polynôme de Taylor} de $f$ de degré $N$ en $z_0$ est:
    
    \[
    T_N f(z) = \sum_{n = 0}^N \frac{f^{(n)}(z_0)}{n!}(z - z_0)^n
    \]
\end{definition}

\begin{result}[séries de Taylor]
    Soit $R > 0$ et $D_R(z_0) = \{z \in \Cx : \module{z - z_0} < R \}$ le plus grand disque de rayon $R$ centré en $z_0$ contenu dans $D$.
    
    Convention: si $D = \Cx \implies R = +\infty$ et $D_R(z_0) = \Cx$
    
    Alors:
    
    \begin{enumerate}[label=\arabic{enumi})]
        \item 
        \[
        T f(z) = \lim_{N \rightarrow +\infty} T_N f(z) = \sum_{n = 0}^{+\infty} \frac{f^{(n)}(z_0)}{n!}(z - z_0)^n
        \]
        
        existe et est finie $\forall z \in D_R(z_0)$. L'expression $T f(z)$ s'appelle \textbf{la série de Taylor} de $f$ en $z_0$.
        
        \item 
        De plus, on a $f(z) = T f(z) \quad \forall z \in D_R(z_0)$
        
        $R$ est appelé \textbf{le rayon de convergence} de la série de Taylor.
        
        \item 
        Les coefficients de la série de Taylor sont reliés à la formule de Cauchy par le corollaire du §3.4.
        On a:
        
        \[ \frac{f^{(n)}(z_0)}{n!} = \frac{1}{2\pi i} \int_\gamma \frac{f(\xi)}{(\xi - z_0)^{n+1}} \dxi \]
        
        où $\gamma \subset D_R(z_0)$ est une courbe simple fermée régulière orientée positivement telle que $z_0 \in \inte \gamma$.
    \end{enumerate}
\end{result}

\subsection{Exemples}

\begin{example}[1]
    \[f(z) = e^z\]
    
    est holomorphe dans $\Cx$.
    On a $f^{(n)}(z) = e^z$ et $f^{(n)}(0) = 1 \quad \forall n \in N$.
    
    Donc:
    
    \[ e^z = \sum_{n=0}^{+\infty} \frac{z^n}{n!} \quad \forall z \in \Cx \]
\end{example}

\begin{example}[2]
    \[f(z) = \frac{1}{1-z}\]
    
     est holomorphe dans $D = \Cx \setminus \{1\}$.
    
    Le plus grand disque centré en $z_0 = 0$ contenu dans $D$ est $D_1(0) = \{ z \in \Cx : \module{z} < 1 \}$.
    
    On a $f^{(n)}(z) = \frac{n!}{(1-z)^{n+1}}$ et $f^{(n)}(0) = n! \ \forall n \in \N$.
    
    Donc:
    
    \[
    \frac{1}{1-z} = \sum_{n=0}^{+\infty} z^n \quad \forall z \in \Cx, \module{z} < 1
    \]
    
    \textbf{\og Série géométrique \fg{}} avec rayon de convergence $R = 1$.
\end{example}

\begin{example}[3]
    \[f(z) = \frac{1}{1 + z^2}\]
    
    est holomorphe dans $D = \Cx \setminus \{-i; i\}$.
    
    Le plus grand disque centré en $z_0 = 0$ et contenu dans $D$ est $D_1(0) = \{z \in \Cx : \module{z} < 1\}$.
    
    On a:
    
    \[
    \frac{1}{1 + z^2}
    = \frac{1}{1 - \left(-z^2\right)}
    \overset{\textrm{Ex. 2}}{=}
    \sum_{n=0}^{+\infty} (-z^2)^n
    = \sum_{n=0}^{+\infty} (-1)^n z^{2n} \quad \forall z \in \Cx, \module{z} < 1
    \]
    
    Le rayon de convergence $R = 1$
\end{example}

\textit{Autre exemple: ex. 5, série 7}

\subsection{Applications}

\begin{enumerate}[label=\arabic{enumi})]
    \item Règle de l'Hôpital
    
    Soit $z_0 \in \Cx$ et $f, g$ deux fonctions holomorphes au voisinage de $z_0$ telles que $f(z_0) = 0$, $g(z_0) = 0$ et $g'(z_0) \neq 0$.
    Alors:
    
    \[ \lim_{z \rightarrow z_0} \frac{f(z)}{g(z)} = \lim_{z \rightarrow z_0} \frac{f'(z)}{g'(z)} = \frac{f'(z_0)}{g'(z_0)} \]
    
    \textit{Preuve: ex.4, série 8}
    
    \item Théorème de Liouville
    
    Soit $f: \Cx \rightarrow \Cx$ une fonction \textit{bornée} et holomorphe \textit{dans} $\Cx$, alors $f$ est constante.
    
    \textit{Preuve: corrigé de l'ex. 18, p.248 (§11.3)}
\end{enumerate}


\section{Développement et série de Laurent d'une fonction holomorphe}

\subsection{Problématique, définitions et résultats}


\begin{motivation}\hfill
    
    Le développement de Taylor d'une fonction $f$ donne une série en puissances \textbf{positives} de $z - z_0$ au voisinage d'un point $z_0$ où $f$ \textbf{est} holomorphe.
    
    \textbf{But:} généralisation avec un développement en puissances \textbf{positives} et \textbf{négatives} de $z - z_0$ où $z_0$ peut être une \textbf{singularité} de $f$.
\end{motivation}


\begin{hypothesis}\hfill
    
    Soit $D \subset \Cx$ un domaine simplement connexe, $z_0 \in D$ et $f: D \setminus \{z_0\} \rightarrow \Cx$ une fonction holomorphe dans $D \setminus \{z_0\}$.
\end{hypothesis}

\begin{definition}[11.1, p.78]
    Pour $N \in \N$, le développement de Laurent de $f$ de degré $N$ au voisinage de $z_0$ est:
    
    \[ L_N f(z) = \sum_{n = -N}^{N} c_n (z - z_0)^n \]
    
    \[ = \frac{c_ {-N}}{(z - z_0)^N} + \ldots + \frac{c_ {-1}}{z - z_0} + c_0 + c_1 (z - z_0) + \ldots + c_n (z - z_0)^N \]
    
    avec
    
    \[ c_n = \oipi \int_\gamma \frac{f(\xi)}{(\xi - z_0)^{n + 1}} \dxi \]
    
    où $\gamma \subset D$ est une courbe simple fermée régulière (par morceaux) orientée positivement telle que $z_0 \in \inte \gamma$.
\end{definition}

\begin{result}[11.2, p.80]
    Soit $R > 0$ et $D_R(z_0) = \{ z \in \Cx : \module{z - z_0} < R \}$ le plus grand disque de rayon $R$ centré en $z_0$ et contenu dans $D$.
    Alors:
    
    \begin{enumerate}[label=\arabic{enumi})]
        \item 
        \[ Lf(z) = \lim_{N \rightarrow \infty} L_N f(z) = \sum_{n = -\infty}^\infty c_n (z - z_0)^n \]
        
        existe et est finie pour tout $z \in D_R(z_0) \setminus \{z_0\}$.
        L'expression $Lf(z)$ s'appelle la \textbf{série de Laurent} de $f$ au voisinage de $z_0$.
        
        \item 
        De plus, on a $f(z) = Lf(z) \quad \forall z \in D_R(z_0) \setminus \{z_0\}$ et $R$ est appelé le \textbf{rayon de convergence} de la série de Laurent.
    \end{enumerate}
\end{result}

\begin{remark}\hfill
    
    \begin{enumerate}[label=\alph*)]
        \item 
        La série de Laurent de $f$ peut s'écrire sous la forme suivante:
        
        \[ Lf(z) = \sum_{n = -\infty}^{-1} c_n(z - z_0)^n + \sum_{n = 0}^{\infty} c_n(z - z_0)^n \]
        
        \begin{itemize}
        \item 
        La première série
        
        \[
        \sum_{n = -\infty}^{-1} c_n(z - z_0)^n =
        \sum_{n = 1}^{\infty} c_{-n}(z - z_0)^{-n} =
        \frac{c_{-1}}{z - z_0} + \frac{c_{-2}}{(z - z_0)^2} + \ldots
        \]
        
        s'appelle la \textbf{partie singulière} de la série de Laurent.
        
        \item 
        La deuxième série
        
        \[
        \sum_{n = 0}^{\infty} c_n(z - z_0)^n =
        c_0 + c_1(z - z_0) + c_2(z - z_0)^2 + \ldots
        \]
        
        s'appelle la \textbf{partie régulière} de la série de Laurent.
        \end{itemize}

        \item 
        \textit{Si} $f: D \rightarrow \Cx$ est holomorphe en $z_0$, alors la série de Laurent coïncide avec la série de Taylor.
        
        En effet, la partie singulière de la série de Laurent est nulle puisque:
        
        Pour $n = 1, 2, 3, \ldots$, on a:
        
        \[ c_{-n} =  \oipi \int_\gamma \frac{f(\xi)}{(\xi - z_0)^{-n + 1}} \dxi
        = \oipi \int_\gamma f(\xi)(\xi - z_0)^{n - 1} \dxi
        = 0 \]
        
        par le Théorème de Cauchy (§3.2), car $f(\xi)(\xi - z_0)^{n - 1}$ est holomorphe dans $D$.
        
        Les coefficients de la partie régulière donnent la série de Taylor car:
        
        Pour $n = 1, 2, 3, \ldots$, on a:
        
        \[ c_{n} =  \oipi \int_\gamma \frac{f(\xi)}{(\xi - z_0)^{n + 1}} \dxi
        = \frac{f^{(n)}(z_0)}{n!}\]
        
        par le corollaire de la formule intégrale de Cauchy (§3.4), car $f(\xi)$ est holomorphe dans $D$.
    \end{enumerate}
\end{remark}

\subsection{Définitions issues de la série de Laurent}

\begin{note}
    Ces définitions sont issues de la \textbf{Définition 11.2} page 80 du livre de référence.
\end{note}

\begin{definition}[1]
    $z_0 \in \Cx$ est un \textbf{point régulier} de $f \iff$ la partie singulière de la série de Laurent au voisinage de $z_0$ est nulle.
    
    C'est-à-dire:
    
    \[ Lf(z) = Tf(z) = \sum_{n = 0}^{\infty} \frac{f^{(n)}(z_0)}{n!} (z - z_0)^n \]
\end{definition}

\begin{definition}[2]
    Soit $m \in \N^\ast$, $z_0 \in \Cx$ est \textbf{un pôle d'ordre $m$} de $f \iff c_{-m} \neq 0$ et $c_{-k} = 0 \quad \forall k \geq m + 1$.
    
    C'est-à-dire:
    
    \[ Lf(z) = \frac{c_{-m}}{(z-z_0)^m} + \ldots + \frac{c_{-1}}{z-z_0} + \sum_{n = 0}^{\infty} c_n(z - z_0)^n \]
\end{definition}

\begin{definition}[3]
    $z_0 \in \Cx$ est une \textbf{singularité essentielle} (isolée) de $f \iff c_{-n} \neq 0$ pour une infinité d'indices $n$.
    
    C'est-à-dire:
    
    \[ Lf(z) = \sum_{n = 1}^{\infty} \frac{c_{-n}}{(z - z_0)^{n}} + \sum_{n = 0}^{\infty} c_n(z - z_0)^n \]
\end{definition}

\begin{definition}[4]
    Le \textbf{résidu} de $f$ en $z_0$, noté $\Res_{z_0}(f)$, est la valeur du coefficient $c_{-1}$ de la série de Laurent de $f$ au voisinage de $z_0$.
    
    C'est-à-dire:
    
    \[ \Res_{z_0}(f) := c_{-1} = \oipi \int_\gamma f(\xi) \dxi \]
    
    où $\gamma \subset D$ avec $z_0 \in \inte \gamma$
\end{definition}

\subsection{Exemples}

\begin{example}[1]
    Soit
    
    \[f(z) = \frac{1}{z}\]
    
    ($f$ holomorphe dans $\Cx \setminus \{0\}$)
    
    \begin{enumerate}[label=\alph*)]
        \item 
        Au voisinage de $z_0 = 0$, on a $Lf(z) = \frac{1}{z} + 0$
        
        $c_{-1} = 1$ et $c_{-n} = 0$ pour $n \geq 2 \implies z_0$ est un pôle d'ordre 1 de $f$ et $\Res_0(f) := c_{-1} = 1$
        
        \item 
        Au voisinage de $z_0 = 1$, on a:
        
        \[ \frac{1}{z} = \frac{1}{1 - (1 - z)} = \sum_{n=0}^\infty (1-z)^n = \sum_{n=0}^\infty (-1)^n (z-1)^n = Tf(z) = Lf(z) \]
        
        \textit{(cf. série géométrique, donc le rayon de convergence est de 1)}
        
        Partie singulière nulle $\implies z_0 = 1$ est un point régulier de $f$ et $\Res_1(f) = 0$
    \end{enumerate}
\end{example}

\begin{example}[2]
    \[ f(z) = \frac{1}{z^3} + \frac{2}{z}\]
    
    ($f$ holomorphe dans $\Cx \setminus \{0\}$).
    Au voisinage de $z_0 = 0$, on a:
    
    \[ Lf(z) = \frac{1}{z^3} + \frac{2}{z} + 0 \]
    
    On a $c_{-1} = 2$, $c_{-2} = 0$, $c_{-3} = 1$ et $c_{-n} = 0$ pour $n \geq 4$.
    
    $\implies z_0 = 0$ est un pôle d'ordre 3 de $f$ et $\Res_0(f) := c_{-1} = 2$
\end{example}

\begin{example}[3]
    Soit
    
    \[ f(z) = \frac{1}{z^2 + z} \]
    
    ($f$ holomorphe dans $\Cx \setminus \{-1,0\}$).
    Au voisinage de $z_0 = 0$, on a:
    
    \begin{align*}
        \frac{1}{z^2 + z} &= \frac{1}{z(z+1)} = \frac{1}{z} - \frac{1}{z + 1} = \frac{1}{z} - \frac{1}{1 - (-z)}
        \\&= \frac{1}{z} - \sum_{n = 0}^\infty (-z)^n
        = \frac{1}{z} - \sum_{n = 0}^\infty (-1)^n z^n
        = \frac{1}{z} + \sum_{n = 0}^\infty (-1)^{n+1} z^n
    \end{align*}
    
    $\implies c_{-1} = 1$ et $c_{-n} = 0$ pour $n \geq 2 \implies z_0 = 0$ est un pôle d'ordre 1 de $f$ et $\Res_0(f) = 1$
\end{example}

\begin{example}[4]
    \[ f(z) = \frac{\sin z}{z},\ g(z) = \frac{\cos z}{z} \]
    
    ($f$ et $g$ holomorphes dans $\Cx \setminus \{0\}$).
    Au voisinage de $z_0 = 0$, on a:
    
    \begin{enumerate}[label=\alph*)]
        \item 
        \begin{align*}
        \frac{\sin z}{z} &= \frac{1}{z} \sin z = \frac{1}{z} \sum_{n = 0}^\infty (-1)^n \frac{z^{2n + 1}}{(2n + 1)!} = \sum_{n = 0}^\infty (-1)^n \frac{z^{2n}}{(2n + 1)!}
        \\&= 0 + 1 - \frac{z^2}{3!} + \frac{z^4}{5!} + \ldots = Lf(z)
        \end{align*}
        
        $\implies z_0 = 0$ est un \textbf{point régulier} de $f$ et $\Res_0(f) = c_{-1} = 0$ ($z_0 = 0$ est une singularité éliminable).
        
        \item 
        \begin{align*}
        \frac{\cos z}{z} &= \frac{1}{z} \cos z = \frac{1}{z} \sum_{n = 0}^\infty (-1)^n \frac{z^{2n}}{(2n)!} = \sum_{n = 0}^\infty (-1)^n \frac{z^{2n - 1}}{(2n)!}
        \\&= \frac{1}{z} + \sum_{n = 1}^\infty (-1)^n \frac{z^{2n - 1}}{(2n)!}
        = \frac{1}{z} - \frac{z}{2!} + \frac{z^3}{4!} - \ldots = Lf(z)
        \end{align*}
        
        $\implies z_0 = 0$ est un pôle d'ordre 1 de $g$ et $\Res_0(g) = 1$
    \end{enumerate}
\end{example}

\begin{example}[5]
    \[ f(z) = e^{\frac{1}{z}} \]
    
    ($f$ holomorphe dans $\Cx \setminus \{0\}$).
    Au voisinage de $z_0 = 0$, on a:
    
    \[
    e^{\frac{1}{z}} = \sum_{n=0}^\infty \frac{1}{n!} \left(\frac{1}{z}\right)^n = 1 + \sum_{n=1}^\infty \frac{1}{n!}z^{-n} = Lf(z)
    \]
    
    $c_{-n} = \frac{1}{n!} \neq 0 \ \forall n \geq 1 \implies z_0$ est une singularité essentielle de $f$ et $\Res_0(f) = c_{-1} = \frac{1}{1!} = 1$.
\end{example}


\section{Étude des pôles d'une fonction et calcul des résidus}

\subsection{Méthodes de détection des pôles}

\begin{note}
    Ces résultats sont issus de la \textbf{Proposition 11.3} page 81 du livre de référence.
\end{note}

\begin{definition}
    Soient $n \in N^\ast$ et $z_0 \in \Cx$. $z_0$ est un \textbf{zéro d'ordre} $n$ de $f$ lorsque:
    
    \[ f(z_0) = f^{(1)}(z_0) = f^{(2)}(z_0) = \ldots = f^{(n-1)}(z_0) = 0 , \quad \textrm{mais } f^{(n)}(z_0) \neq 0 \]
\end{definition}

\begin{convention}
    Si $z_0$ n'est pas un zéro de $f$, alors $f(z_0) \neq 0$ et puisque $f^{(0)}(z_0) = f(z_0) \neq 0$, en posant $n=0$, on dit que \og $z_0$ est un zéro d'ordre 0 \fg{}.
\end{convention}

\begin{method}\hfill
    
    \begin{enumerate}[label=\alph*)]
    \item 
    Soit $f(z) = \frac{p(z)}{q(z)}$ où $p$ et $q$ sont des fonctions holomorphes au voisinage de $z_0 \in \Cx$ qui est un zéro d'ordre $k$ de $p$ et un zéro d'ordre $\ell$ de $q$.
    Deux cas sont possibles:
    \begin{definition}
        \item[Cas 1:]
        si $\ell > k$, alors $z_0$ est un pôle d'ordre $\ell - k$ de $f$.
        \item[Cas 2:]
        si $\ell \leq k$, alors $z_0$ est un point régulier de $f$.
        On dit que $z_0$ est une \textbf{singularité éliminable} en posant $f(z_0) = \lim\limits_{z \rightarrow z_0} \frac{p(z)}{q(z)}$.
    \end{definition}

    \item 
    Soit $f$ une fonction holomorphe dans $D \setminus \{z_0\}$, soit $m \in \N^\ast$ et
    
    \[ L = \lim\limits_{z \rightarrow z_0} \left[ (z - z_0)^m f(z) \right] \]
    
    Si $L$ est finie et $L \neq 0$, alors $z_0$ est un pôle d'ordre $m$ de $f$.
    \end{enumerate}
\end{method}

\begin{example}\hfill

    \begin{enumerate}[label=\arabic{enumi})]
        \item 
        \[ f(z) =  \frac{\sin z}{z}, \quad z_0 = 0 \]
        
        Avec $p(z) = \sin z$ et $q(z) = z$, on a $p(0) = \sin(0) = 0$, $p'(0) = \cos(0) = 1$, $q(0) = 0$, $q'(0) = 1$.
        Alors $k = \ell = 1$.
        Donc $z_0$ est un point régulier.
        C'est une singularité éliminable en posant:
        
        \[
        f(z) =
        \left\{
        \begin{array}{cc}
        \frac{\sin z}{z} & \textrm{pour } z \neq 0\\
        \lim\limits_{z \rightarrow 0} \frac{\sin z}{z} = 1 & \textrm{pour } z = 0\\
        \end{array}
        \right.
        \]
        
        \item 
        \[ f(z) =  \frac{z}{\sin^2 z}, \quad z_0 = 0 \]
        
        Avec $p(z) = z$ et $q(z) = \sin^2 z$, on a $p(0) = 0$, $p'(0) = 1 \neq 0$, $q(0) = 0$, $q'(z) \Big|_{z=0} = 2 \sin z \cos z \Big|_{z=0} = 0$, $q''(z) \Big|_{z=0} = 2\cos^2 z - 2\sin^2 z \Big|_{z=0} = 2 \neq 0$.
        Alors $k = 1$ et $l = 2$.
        Donc $z_0 = 0$ est un pôle d'ordre $\ell - k = 1$ de $f$.
        
        \item 
        \[ f(z) = \frac{\sin (z - \pi)}{(z - \pi)^3} \]
        
        $z_0 = \pi$ est un pôle d'ordre 2 de $f$ car
        
        \[ \lim\limits_{z \rightarrow z_0} \left[ (z - \pi)^2 f(z) \right]
        = \lim\limits_{z \rightarrow z_0} \left[ (z - \pi)^2 \frac{\sin (z - \pi)}{(z - \pi)^3} \right]
        = \lim\limits_{z \rightarrow z_0} \frac{\sin (z - \pi)}{z - \pi} = 1\]
        
        Par ailleurs, on constate
        
        \[ \lim\limits_{z \rightarrow z_0} \left[ (z - \pi)^3 f(z) \right]
        = \lim\limits_{z \rightarrow z_0} \left[ (z - \pi)^3 \frac{\sin (z - \pi)}{(z - \pi)^3} \right]
        = \lim\limits_{z \rightarrow z_0} \sin (z - \pi) = 0 \]
        
        \[ \lim\limits_{z \rightarrow z_0} \left[ (z - \pi) f(z) \right]
        = \lim\limits_{z \rightarrow z_0} \left[ (z - \pi) \frac{\sin (z - \pi)}{(z - \pi)^3} \right]
        = \lim\limits_{z \rightarrow z_0} \frac{1}{z - \pi} \lim\limits_{z \rightarrow z_0} \frac{\sin (z - \pi)}{z - \pi} = \infty \]
    \end{enumerate}
\end{example}

\begin{remark}
    Les preuves des critères a) et b) découlent du développement en série de Laurent et de la définition de pôle (§4.2.1 et §4.2.2).
\end{remark}

\subsection{Formules de calcul du résidu d'une fonction}

\begin{note}
    Ces résultats sont issus des \textbf{Propositions 11.4, 11.5} page 81 du livre de référence.
\end{note}

\begin{method}\hfill
    
    \begin{enumerate}[label=\alph*)]
    \item 
    Soit $f$ une fonction holomorphe dans $D\setminus \{ z_0 \}$, soit $m \in \N^\ast$.
    Si $z_0$ est un pôle d'ordre m de $f$, alors:
    
    \[ \Res_{z_0}(f) = \frac{1}{(m - 1)!} \lim\limits_{z \rightarrow z_0} \frac{\dd^{m-1}}{\dz^{m - 1}} \left[ (z - z_0)^m f(z) \right] \]
    
    \item 
    Soit $f(z) = \frac{p(z)}{q(z)}$ où $p$ et $q$ sont des fonctions holomorphes au voisinage de $z_0 \in \Cx$ telles que $z_0$ est un zéro d'ordre 1 de $q(z_0)$ et $p(z_0) \neq 0$.
    Alors:
    
    \[ \Res_{z_0}(f) = \frac{p(z_0)}{q'(z_0)} \]
    \end{enumerate}
\end{method}

\begin{example}\hfill
    
    \begin{enumerate}[label=\arabic{enumi})]
    \item 
    \[ f(z) = \frac{3z^2}{z + 2} \]
    
    alors $z_0 = -2$ est un pôle d'ordre 1 de $f$.
    
    \[ \implies \Res_{-2}(f) = \lim\limits_{z \rightarrow -2} (z + 2) \frac{3z^2}{z + 2} = 12 \]
    
    \item 
    \[ f(z) = \frac{e^z}{(z - 5)^3} \]
    
    alors $z_0 = 5$ est un pôle d'ordre 3 de $f$.
    
    \[ \implies \Res_{5}(f) = \frac{1}{2!} \lim\limits_{z \rightarrow 5} \frac{\dd^2}{\dz^2} \left[ (z - 5)^3 \frac{e^z}{(z - 5)^3}  \right] = \frac{1}{2} \lim\limits_{z \rightarrow 5} e^z = \frac{e^5}{2} \]
    
    \item 
    \[ f(z) = \frac{\sin z}{z^2 + 1} \]
    
    comme $z^2 + 1 = (z - i)(z + i)$, alors $z_0 = i$ et $z_0 = -i$ sont deux pôles d'ordre 1 de $f$.
    
    \[ \implies \Res_{i}(f) = \lim\limits_{z \rightarrow i} (z - i) \frac{\sin z}{(z - i)(z + i)} = \frac{\sin i}{2i} \]
    
    \[ \implies \Res_{-i}(f) = \lim\limits_{z \rightarrow -i} (z + i) \frac{\sin z}{(z - i)(z + i)} = \frac{\sin (-i)}{-2i} = \frac{\sin i}{2i} \]
    
    \item 
    \[ f(z) = \frac{3z^2}{z + 2} \]
    
    et $z_0 = -2$.
    Avec $p(z) = 3z^2$ et $q(z) = z + 2$, on a que $z_0 = -2$ est un zéro d'ordre 1 de $q$ avec $p(-2) = 12$ et $q'(-2) = 1$.
    
    \[ \implies \Res_{-2}(f) = \frac{12}{1} = 12 \]
    \end{enumerate}
    
\end{example}

\subsection{Démonstration des formules}

\begin{proof}
    
    \begin{enumerate}[label=\alph*)]
    \item 
    Si $m = 1$, la série de Laurent de $f$ au voisinage de $z_0$ donne:
    
    \[ f(z) = \frac{c_{-1}}{z - z_0} + \sum_{n = 0}^\infty c_n (z - z_0)^n \]
    
    Alors $(z - z_0) f(z) = c_{-1} + F(z)$ avec $F(z) = \sum\limits_{n = 0}^\infty c_n (z - z_0)^{n + 1}$
    
    \[ \implies \lim_{z \rightarrow z_0} \left[ (z - z_0) f(z) \right] = c_{-1} + \lim_{z \rightarrow z_0} F(z) = c_{-1} =: \Res_{z_0}(f) \]
    
    Si $m = 2$, la série de Laurent de f au voisinage de $z_0$ donne:
    
    \[ f(z) = \frac{c_{-2}}{(z - z_0)^2} + \frac{c_{-1}}{z - z_0} + \sum_{n = 0}^\infty c_n (z - z_0)^n \]
    
    \[ (z - z_0)^2f(z) = c_{-2} + c_{-1}(z - z_0) + \sum_{n = 0}^\infty c_n (z - z_0)^{n + 2} \]
    
    et
    
    \begin{align*}
    \frac{\dd}{\dz}\left[ (z - z_0)^2 f(z) \right]
    &= \frac{\dd}{\dz}\left[ c_{-2} + c_{-1}(z - z_0) + \sum_{n = 0}^\infty c_n (z - z_0)^{n + 2} \right]
    \\&= 0 + c_{-1} + \sum_{n = 0}^\infty c_n (n + 2) (z - z_0)^{n + 1}
    \\&= c_{-1} + G(z)
    \end{align*}
    
    où
    
    \[ G(z) = \sum_{n = 0}^\infty c_n (n + 2) (z - z_0)^{n + 1} \]
    
    \newpage
    
    Donc
    
    \[
    \lim_{z \rightarrow z_0} \frac{\dd}{\dz}\left[ (z - z_0)^2 f(z) \right] = c_{-1} + \lim_{z \rightarrow z_0} G(z) = c_{-1} =: \Res_{z_0}(f)
    \]
    
    Si $m \geq 3$, raisonnement analogue et preuve par récurrence qui fait apparaître le terme $\frac{1}{(m - 1)!}$ dans la formule
    
    \item 
    On applique la formule a) à $f(z) = \frac{p(z)}{q(z)}$ où $z_0$ est un pôle d'ordre 1 de $f$.
    
    \textit{Cf. ex. 5, série 9}
    \end{enumerate}
\end{proof}
