% !TeX spellcheck = fr_FR
\chapter{Transformée de Laplace}


\section{Introduction}

\begin{motivation}
    Généralisation de la théorie de Fourier pour appliquer une étude de problèmes transitoires en électricité avec des conditions initiales.
\end{motivation}

\subsubsection*{Procédé heuristique induisant la définition de la transformée de Laplace:}

Soit $g: \R \rightarrow \R$, la transformée de Fourier $\TF (g): \R \rightarrow \Cx$ est définie par $\TF(g)(\alpha) = \ostpi \int_{-\infty}^{\infty} g(t) e^{-i\alpha t} \dt$.

On considère $g(t) = \left\{
\begin{array}{lcl}
f(t) & \textrm{ si } & t \geq 0\\
0    & \textrm{ si } & t < 0\\
\end{array}
\right.$ et on autorise la variable $\alpha \in \R$ à prendre des valeurs complexes.
En posant $\alpha = -i z$ avec $z \in \Cx$, on obtient:

\[ \sqrt{2\pi} \ \TF(g)(-iz) = \int_{-\infty}^{\infty} g(t) e^{-i(-iz)t} \dt = \int_{0}^{\infty} f(t) e^{-zt} \dt \]

Ceci est une nouvelle fonction qui dépend de la variable $z \in \Cx$ qui est appelée la transformée de Laplace de $f$.

\section{Transformée de Laplace d'une fonction}

\subsection{Définition}

\begin{definition}
    Soit $f: \R_+ \rightarrow \R$ une fonction continue par morceaux et soit $\gamma_0 \in \R$ tels que $\int_0^\infty \module{f(t)} e^{-\gamma_0 t} \dt < \infty$.
    
    La \textbf{transformée de Laplace} de $f$ est la fonction notée $\TL(f)$ où $F: \Cx \rightarrow \Cx$ définie par:
    
    \[ z \longmapsto \TL(f)(z) = F(z) = \int_0^\infty f(t) e^{-zt} \dt \quad \forall z \in \Cx : \Rep z \geq \gamma_0 \]
    
    $\gamma_0$ s'appelle l'abscisse de convergence de $f$.
\end{definition}

\begin{remark}
    Si $\Rep z \geq \gamma_0$, alors $\TL(f)(z)$ est bien définie.
    En effet, comme:
    
    \[ \module{e^{-zt}} = \module{e^{-(\Rep z + i \Imp z) t}} = e^{-t \Rep z} \module{e^{-i t \Imp z}} = e^{-t\Rep z} \leq e^{-t\gamma_0} \]
    
    pour $t \geq 0$ si $\Rep z \geq \gamma_0$.
    Alors:
    
    \[ \module{\TL(f)(z)} = \module{\int_0^\infty f(t) e^{-zt} \dt} \leq \int_0^\infty \module{f(t)} \module{e^{-zt}} \dt \leq \int_0^\infty \module{f(t)} e^{-\gamma_0 t} \dt < \infty \]
\end{remark}

\subsection{Exemples}

\begin{example}[1]
    Calculer la transformée de Laplace de la fonction $f: \R_+ \rightarrow \R$, où $f(t) = 1$.

Puisque

\[
    \int_0^\infty \module{f(t)} e^{-\gamma_0 t} \dt
    = \int_0^\infty e^{-\gamma_0 t} \dt
    = \left.-\frac{e^{-\gamma_0 t}}{\gamma_0}\right|_0^\infty
    = \frac{1}{\gamma_0} \left[1 - \lim_{t \rightarrow \infty} e^{-\gamma_0 t}\right]
    = \left\{
    \begin{array}{clc}
    \frac{1}{\gamma_0} & \textrm{ si } \gamma_0 > 0\\
    +\infty & \textrm{ si } \gamma_0 < 0
    \end{array}\right.
\]

Alors n'importe quel $\gamma_0 \in \R_+^\ast$ est abscisse de convergence de $f$.

\[
    \TL(f)(z)
    = \int_0^\infty \module{f(t)} \module{e^{-zt}} \dt
    = \int_0^\infty \module{e^{-zt}} \dt
    = \left.-\frac{e^{-z t}}{z}\right|_0^\infty
    = \frac{1}{z} \left[1 - \lim_{t \rightarrow \infty} e^{-z t}\right]
\]

Or,

\[
    \lim_{t \rightarrow \infty} \module{e^{-zt}}
    = \lim_{t \rightarrow \infty} \module{e^{-(x + iy)t}}
    = \lim_{t \rightarrow \infty} e^{-xt} \module{e^{-iyt}}
    = 0
\]

si $x = \Rep z > 0$.

Résultat: $\TL(f): \Cx \rightarrow \Cx$, où:

\[
    z \longmapsto \TL(f)(z) = \frac{1}{z} \textrm{ si } \Rep z > 0
\]
\end{example}

\begin{example}[2]
Calculer la transformée de Laplace de la fonction $f: \R_+ \rightarrow \R$, où $f(t) = e^{at}$ avec $a \in \R$:

Comme 

\begin{align*}
    \int_0^\infty \module{f(t)} e^{-\gamma_0 t} \dt 
    &= \int_0^\infty e^{(a - \gamma_0)t}
    = \left.
    \frac{e^{(a - \gamma_0)t}}{a - \gamma_0}
    \right|_0^\infty
    \\& = \frac{1}{a - \gamma_0} \left[ \lim_{t \rightarrow \infty} e^{(a - \gamma_0)t} - 1 \right]
    = \left\{
    \begin{array}{clc}
    +\infty & \textrm{ si } a \geq \gamma_0\\
    \frac{1}{\gamma_0 - a} & \textrm{ si } a < \gamma_0
    \end{array}\right.
\end{align*}

Alors, n'importe quel $\gamma_0 > a$ est abscisse de convergence de $f$.

\[
    \TL(f)(z)
    = \int_0^\infty f(t) e^{-z t} \dt 
    = \int_0^\infty e^{(a - z)t}
    = \left.
    \frac{e^{(a - z)t}}{a - z}
    \right|_0^\infty
    = \frac{1}{a - z} \left[ \lim_{t \rightarrow \infty} e^{(a - z)t} - 1 \right]
\]

Or,

\[
    \lim_{t \rightarrow \infty} \module{e^{(a - z)t}}
    = \lim_{t \rightarrow \infty} \module{e^{(a - x - iy)t}}
    = \lim_{t \rightarrow \infty} e^{(a - x)t} \module{e^{- iyt}}
    = 0
\]

si $a - x < 0$.

Résultat: $\TL(f): \Cx \rightarrow \Cx$, où:

\[
z \longmapsto \TL(f)(z) = \frac{1}{z - a} \textrm{ si } \Rep z > a
\]

\textit{Autres exemples: ex.1, série 12}

\end{example}


\section{Propriétés de la transformée de Laplace}

On considère deux fonctions, $f,g: \R_+ \rightarrow \R$ continues par morceaux et $\gamma_0 \in \R$ tels que $\int_{0}^{\infty} \module{f(t)} e^{-\gamma_0 t} \dt < \infty$ et $\int_{0}^{\infty} \module{g(t)} e^{-\gamma_0 t} \dt < \infty$.

On note $\TL(f) = F$ et $\TL(g) = G$ les transformées de Laplace de $f$ et de $g$.

\subsection{Linéarité et décalage}

\begin{itemize}
    \item 
    $\TL(af + bg) = a \TL(f) + b \TL(g)$ avec $a, b \in \R$
    \item 
    Si $a \in \R_+^\ast$ et $b \in \R$ et $h(t) = e^{-bt} f(at)$ alors
    
    \[ \TL(h)(z) = \frac{1}{a} \TL(f)(\frac{z + b}{a}) \]
    
    pour tout $z \in \Cx$ tel que $\Rep\left(\frac{z + b}{a}\right) \geq \gamma_0$
\end{itemize}

\subsection{Transformée de Laplace du produit de convolution}

Si

    \[
        (f \ast g)(t)
        = \int_{-\infty}^\infty f(t - s)g(s) \ds
        = \int_0^t f(t-s)g(s) \ds
    \]

est le produit de convolution de $f$ et de $g$, alors:

\[
    \TL(f \ast g)(z) = \TL(f)(z)\TL(g)(z) \quad \forall z \in \Cx : \Rep z \geq \gamma_0
\]

\subsection{Holomorphie et dérivation de la transformée de Laplace}

$\TL(f)$ est holomorphe dans le domaine $D = \{ z \in \Cx | \Rep z > \gamma_0 \}$.
De plus, $\forall z \in D$, on a:

\[ \TL(f)'(z) = - \int_0^\infty t f(t) e^{-zt} \dt = - \TL(h)(z) \]

où $h: \R_+ \rightarrow \R$ est définie par $h(t) = tf(t)$.

\subsection{Transformée de Laplace de la dérivée d'une fonction}

Si de plus $f \in C^1(\R_+)$ et $\int_0^\infty \module{f'(t)} e^{-\gamma_0 t} \dt < \infty$, alors $\forall z \in D$, on a:

\[
    \TL(f')(z) = z \TL(f)(z) - f(0)
\]

Plus généralement: si $f \in C^n (R_+)$ et $\int_0^\infty \module{f^{(k)}(t)} e^{-\gamma_0 t} \dt < \infty$ pour $k = 1, 2, \ldots, n$, alors:

\[
    \TL(f^{(n)})(z) = z^n \TL(f)(z) - z^{n-1} f(0) - z^{n-2} f'(0)
    - \ldots -  - z f^{(n-2)}(0) -  - f^{(n-1)}(0) \enspace \forall z \in D
\]

\subsection{Transformée de Laplace d'une primitive d'une fonction}

Si de plus $f \in C(\R_+)$ avec $\gamma_0 \geq 0$ et si $\varphi(t) = \int_0^t f(s) \ds$ est une primitive de $f$, alors $\forall z \in D$, on a:

\[
    \TL(\varphi)(z) = \frac{1}{z} \TL(f)(z)
\]

\subsection{Esquisse des démonstrations des propriétés}

\textit{Cf. ex.3, série 12}

\subsection{Exemples d'utilisation des propriétés}

\begin{example}
    Calculer la transformée de Laplace de la fonction $f: \R_+ \rightarrow \R$ définie par $f(t) = t^2$.
    
    Méthode: utiliser la propriété 6.3.4 de la deuxième dérivée de $f$:
    
    On a $\TL(f'')(z) = z^2 \TL(f)(z) - zf(0) - f'(0)$.
    Comme $f(t) = t^2, \enspace f'(t) = 2t$ et $f''(t) = 2 \implies f(0) = 0$ et $f'(0) = 0$, et $f''(t) = 2g(t)$ avec $g(t) = 1$ si $t \geq 0$.
    
    \[ \implies \TL(f'')(z) = 2 \TL(g)(z) = 2 \frac{1}{z} \]
    
    Donc on obtient $\frac{2}{z} = z^2 \TL(f)(z) \implies \TL(f)(z) = \frac{2}{z^3}$
    
    Résultat, pour $\TL(f): \Cx \rightarrow \Cx$:
    
    \[ z \longmapsto \TL(f)(z) = \frac{2}{z^3} \]
    
    \textit{Autres exemples: ex. 2, série 12}
\end{example}
