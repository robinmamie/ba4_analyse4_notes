% !TeX spellcheck = fr_FR
\chapter{Fonctions holomorphes et équations de Cauchy-Riemann}


\section{Introduction}

\subsection{Motivation}

\begin{description}
    \item[But:] étendre l'étude de fonctions réelles (du type $f: \R \rightarrow \R$) à des fonctions qui dépendent d'\textbf{une} variable complexe qui sont à valeurs complexes (du type $f: \C \rightarrow \C$ où $\C$ est l'ensemble des nombres complexes).
    
    \item[Rôle:] établir les notions de limite, de continuité, de dérivabilité et d'intégration dans $\C$.
    
    \item[Intérêt:] méthodes puissantes qui permettent de calculer facilement des intégrales \textbf{réelles} compliquées.
\end{description}

\textit{Cf. ex. 4, série 3}
